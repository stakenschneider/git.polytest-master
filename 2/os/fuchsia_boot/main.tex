\documentclass[14pt,a4paper]{article}
\usepackage[a4paper, mag=1000, left=2cm, right=2cm, top=2cm, bottom=2cm, headsep=0.7cm, footskip=cm]{geometry}
\usepackage[utf8x]{inputenc}
\usepackage[T1,T2A]{fontenc}
\usepackage[english,russian]{babel}
\usepackage{indentfirst}
\usepackage[dvipsnames]{xcolor}
\usepackage[colorlinks]{hyperref}
\usepackage{listings} 
\usepackage{fancyhdr}
\usepackage{caption}
\usepackage{subfig}
\usepackage{color}
\usepackage{here}
\usepackage{array}
\usepackage{multirow}
\usepackage{subcaption}
\usepackage{amsmath}
\usepackage{latexsym}
\usepackage{graphicx}
\hypersetup{
	colorlinks = true,
	linkcolor  = black
}

\DeclareCaptionFont{white}{\color{white}} 

\sloppy

% Listing description

\usepackage{listings}
\usepackage{caption}
\DeclareCaptionFont{white}{\color{white}}
\DeclareCaptionFormat{listing}{\colorbox{gray}{\parbox{\dimexpr\textwidth-1.72\fboxsep\relax}{#1#2#3}}}
\captionsetup[lstlisting]{format=listing,labelfont=white,textfont=white,margin=0pt}
\lstset{language=C,
	basicstyle=\footnotesize,
	keepspaces=true,
	tabsize=4,               
	frame=single,                           % Single frame around code
	rulecolor=\color{black},
	captionpos=b,
	showstringspaces=false,	
	abovecaptionskip=-0.9pt,
	xleftmargin=3.4pt,
	xrightmargin=2.6pt,
	breaklines=true,
	postbreak=\raisebox{0ex}[0ex][0ex]{\ensuremath{\color{black}\hookrightarrow\space}},
	xleftmargin=3.2pt,
	literate={а}{{\selectfont\char224}}1
	{~}{{\textasciitilde}}1
	{б}{{\selectfont\char225}}1
	{в}{{\selectfont\char226}}1
	{г}{{\selectfont\char227}}1
	{д}{{\selectfont\char228}}1
	{е}{{\selectfont\char229}}1
	{ё}{{\"e}}1
	{ж}{{\selectfont\char230}}1
	{з}{{\selectfont\char231}}1
	{и}{{\selectfont\char232}}1
	{й}{{\selectfont\char233}}1
	{к}{{\selectfont\char234}}1
	{л}{{\selectfont\char235}}1
	{м}{{\selectfont\char236}}1
	{н}{{\selectfont\char237}}1
	{о}{{\selectfont\char238}}1
	{п}{{\selectfont\char239}}1
	{р}{{\selectfont\char240}}1
	{с}{{\selectfont\char241}}1
	{т}{{\selectfont\char242}}1
	{у}{{\selectfont\char243}}1
	{ф}{{\selectfont\char244}}1
	{х}{{\selectfont\char245}}1
	{ц}{{\selectfont\char246}}1
	{ч}{{\selectfont\char247}}1
	{ш}{{\selectfont\char248}}1
	{щ}{{\selectfont\char249}}1
	{ъ}{{\selectfont\char250}}1
	{ы}{{\selectfont\char251}}1
	{ь}{{\selectfont\char252}}1
	{э}{{\selectfont\char253}}1
	{ю}{{\selectfont\char254}}1
	{я}{{\selectfont\char255}}1
	{А}{{\selectfont\char192}}1
	{Б}{{\selectfont\char193}}1
	{В}{{\selectfont\char194}}1
	{Г}{{\selectfont\char195}}1
	{Д}{{\selectfont\char196}}1
	{Е}{{\selectfont\char197}}1
	{Ё}{{\"E}}1
	{Ж}{{\selectfont\char198}}1
	{З}{{\selectfont\char199}}1
	{И}{{\selectfont\char200}}1
	{Й}{{\selectfont\char201}}1
	{К}{{\selectfont\char202}}1
	{Л}{{\selectfont\char203}}1
	{М}{{\selectfont\char204}}1
	{Н}{{\selectfont\char205}}1
	{О}{{\selectfont\char206}}1
	{П}{{\selectfont\char207}}1
	{Р}{{\selectfont\char208}}1
	{С}{{\selectfont\char209}}1
	{Т}{{\selectfont\char210}}1
	{У}{{\selectfont\char211}}1
	{Ф}{{\selectfont\char212}}1
	{Х}{{\selectfont\char213}}1
	{Ц}{{\selectfont\char214}}1
	{Ч}{{\selectfont\char215}}1
	{Ш}{{\selectfont\char216}}1
	{Щ}{{\selectfont\char217}}1
	{Ъ}{{\selectfont\char218}}1
	{Ы}{{\selectfont\char219}}1
	{Ь}{{\selectfont\char220}}1
	{Э}{{\selectfont\char221}}1
	{Ю}{{\selectfont\char222}}1
	{Я}{{\selectfont\char223}}1,
	extendedchars=true
}

\DeclareCaptionFormat{hfillstart}{\hfill#1#2#3\par}
\captionsetup[table]{format=hfillstart,labelsep=newline,justification=centering,skip=-10pt,textfont=bf}


\usepackage{float}

\begin{document}

\begin{titlepage}
    \centering
    \textsc{Санкт-Петербургский политехнический университет Петра Великого}\\[3mm]
    \textsc{Институт компьютерных наук и технологий}\\[3mm]
    \textsc{Кафедра компьютерных систем и программных технологий}
	
	\vfill
	
	\textbf{Отчёт по лабораторной работе }\\[3mm]
	по курсу «Проектирование ОС и компонентов»\\[3mm]
	по теме «Системные вызовы»\\[41mm]
	
    \begin{flushright}
	\begin{minipage}{.35\textwidth}
		Выполнил студент гр. 13541/3:\\
		Волкова М.Д.\\[3mm]
		Проверил преподаватель:\\
		Душутина Е. В.
	\end{minipage}
    \end{flushright}
	
	\vfill

	Санкт-Петербург\\
	\the\year\ г.
\end{titlepage}


\renewcommand\contentsname{\centerline{Содержание}}
\tableofcontents
\newpage


\section{Цель работы}
\par Необходимо провести запуск операционной системы Fuchsia, опробовать ее функциональность и попытаться провести трассировку системных вызовов.


\section{Ход работы}

\par Операционная система Fuchsia на данный момент все еще находится в разработке и никакого стабильного релиза все еще нет. Однако, ее код находится в открытом репозитории и можно попробовать собрать ее из исходного кода и запустить.\\

\subsection{Создание }

\par Процедура начальной загрузки требует, чтобы у вас был установлен Go 1.6 или новее, а также Git и ваш PATH.

\par Этот скрипт загрузит среду разработки, создав сначала каталоги fuchsia. \\

\begin{lstlisting}[language=c caption={}]
curl -s "https://fuchsia.googlesource.com/fuchsia/+/master/scripts/bootstrap?format=TEXT" | base64 --decode | bash
\end{lstlisting}

\subsection{Настройка переменных среды}

\par Fuchsia использует инструмент jiri для управления git-хранилищами https://fuchsia.googlesource.com/jiri. Этот инструмент управляет набором репозиториев, указанных в манифесте.\\

\par В случае успеха скрипт начальной загрузки должен напечатать сообщение, рекомендующее добавить каталог .jiri\_root / bin в вашу переменную PATH.\\

\par Другим инструментом в .jiri\_root / bin является fx, который помогает конфигурировать, собирать, запускать и отлаживать Fuchsia.\\

\subsection{Настройка fx workflows}

Для установки fx необходимо запустить данный скрипт:

\begin{lstlisting}[language=c caption={}]
# In your fuchsia checkout:
$ cd fuchsia
# Add a configuration to your shell to include fx-env.sh
$ echo "source \"$PWD/scripts/fx-env.sh\"" >> "$HOME/.$(basename "$SHELL")rc"
# If you would like additional convenience tools from the Fuchsia scripts, also
# optionally run the following:
$ echo "fx-update-path" >> "$HOME/.$(basename "$SHELL")rc"
# Restart your shell
$ exec "$SHELL"
\end{lstlisting}


\subsection{Сборка исходного кода}

После того, как переменные среды настроены, то можно выполнить сборку через fx:

\begin{lstlisting}[language=c caption={}]
fx set core.x64 --with //bundles:kitchen_sink
fx build
\end{lstlisting}

\par Первая команда выбирает конфигурацию сборки, которую необходимо собрать, и сама генерирует систему сборки в выходном каталоге (например, out / x64). Фуксия может загружать пакеты по сети; здесь мы используем флаг --available, чтобы сделать доступными для загрузки необходимые пакеты. \\

\par Вторая команда фактически выполняет сборку, преобразовывая исходный код в продукты сборки. Если было измненение в исходном коде, то можно сделать инкрементную сборку, повторно выполнив одну только команду fx build. fx -i build запускает наблюдателя и автоматически делает перестроение каждый раз, когда файл изменяется.

\par Далее необходимо произвести загрузку на USB носитель. Для этого используется следующая команда:

\begin{lstlisting}
fx mkzedboot <usb_drive_device_path>
\end{lstlisting}

После комплиляции всех файлов проекта, в логе сборки можно прочесть инициализацию отдельных частей ядра:


\subsection{Запуск Fuchsia}

Fuchsia на данный момент может быть установлена только на следующих устройствах: 

\begin{itemize}
    \item Acer Switch Alpha 12
    \item Intel NUC
    \item Google Pixelbook
\end{itemize}

Поэтому запуск будет производиться через эмулирование в QEMU. Fuchsia включает в себя готовые файлы для QEMU в buildtools / qemu. \\
Команда fx run запустит Zircon в QEMU, используя локально созданный образ диска:

\begin{lstlisting}
fx run
\end{lstlisting}

Существуют различные флаги для запуска fx для управления конфигурацией QEMU:

\begin{itemize}
    \item -m устанавливает размер памяти QEMU в МБ.
    \item -g включает графику.
    \item -N включает сети.
    \item -k включает ускорение KVM в Linux
\end{itemize}

Для полной сборки fuchsia и ее запуска был написан скрипт: 
\begin{lstlisting}

#!/bin/sh
clear

# Preparing the build environment
# https://github.com/fuchsia-mirror/manifest#prerequisites
# https://github.com/fuchsia-mirror/magenta/blob/master/docs/getting_started.md#preparing-the-build-environment
sudo apt-get install golang git-all curl texinfo libglib2.0-dev autoconf libtool libsdl-dev build-essential

# Creating a new checkout (https://github.com/fuchsia-mirror/manifest#creating-a-new-checkout)
curl -s https://raw.githubusercontent.com/vanadium/go.jiri/master/scripts/bootstrap_jiri | bash -s fuchsia
cd fuchsia
export PATH=`pwd`/.jiri_root/scripts:$PATH
jiri import fuchsia https://fuchsia.googlesource.com/manifest
jiri update

# Building Fuchsia (https://github.com/fuchsia-mirror/manifest#building-fuchsia)
./packages/gn/gen.py
./buildtools/ninja -C out/debug-x86-64

# Build Magenta (https://github.com/fuchsia-mirror/magenta/blob/master/docs/getting_started.md#build-magenta)
cd magenta
make -j32 magenta-pc-x86-64

# Boot Fuchsia (https://github.com/fuchsia-mirror/manifest#building-fuchsia)
# Updated 21 August 2016 to run with a graphical framebuffer
./scripts/run-magenta-x86-64 -g -x ../out/debug-x86-64/user.bootfs
\end{lstlisting}

После комплиляции всех файлов проекта, в логе сборки можно прочесть инициализацию системы во время запуска:

\begin{lstlisting}
welcome to lk/MP

boot args 0x1 0x80 0x0 0x0
INIT: cpu 0, calling hook 0xffffffff801306c0 (version) at level 0x3ffff, flags 0x1
version:
	arch:     X86
	platform: PC
	target:   PC_X86
	project:  MAGENTA_PC_X86_64
	buildid:  _LOCAL
INIT: cpu 0, calling hook 0xffffffff8014f26c (vm_preheap) at level 0x3ffff, flags 0x1
initializing heap
calling constructors
INIT: cpu 0, calling hook 0xffffffff8014f2b4 (vm_post_ctors) at level 0x40000, flags 0x1
INIT: cpu 0, calling hook 0xffffffff8014f2bc (vm) at level 0x50000, flags 0x1
INIT: cpu 0, calling hook 0xffffffff801009e0 (acpi_tables) at level 0x50001, flags 0x1
INIT: cpu 0, calling hook 0xffffffff801047d0 (display_memtype) at level 0x50001, flags 0x1
INIT: cpu 0, calling hook 0xffffffff80103658 (hpet) at level 0x50002, flags 0x1
INIT: cpu 0, calling hook 0xffffffff801037c0 (apic) at level 0x50002, flags 0x1
INIT: cpu 0, calling hook 0xffffffff801050cc (timer) at level 0x50003, flags 0x1
calibrate_apic_timer:231: Calibrating APIC with HPET
initializing mp
initializing threads
initializing timers
initializing ports
INIT: cpu 0, calling hook 0xffffffff8011c810 (intel_hda_init) at level 0x60000, flags 0x1
INIT: cpu 0, calling hook 0xffffffff801256c4 (debuglog) at level 0x6ffff, flags 0x1
creating bootstrap completion thread
[00000.003] K top of bootstrap2()
[00000.003] K INIT: cpu 0, calling hook 0xffffffff80157b90 (global_prng) at level 0x70000, flags 0x1
[00000.004] K WARNING: System has no entropy source.  It is completely unsafe to use this system for any cryptographic applications.
[00000.004] K INIT: cpu 0, calling hook 0xffffffff801604b0 (magenta) at level 0x70000, flags 0x1
[00000.007] K INIT: cpu 0, calling hook 0xffffffff801d7134 (dpc) at level 0x70000, flags 0x1
[00000.011] K initializing platform
[00000.011] K INIT: cpu 0, calling hook 0xffffffff8011eb68 (pcie_early_init) at level 0x8ffff, flags 0x1
[00000.012] K Found 4 cpus
[00000.013] K creating idle thread for cpu 1
[00000.015] K creating idle thread for cpu 2
[00000.016] K creating idle thread for cpu 3
running final init tasks on cpu 1
entering scheduler on cpu 1
running final init tasks on cpu 2
entering scheduler on cpu 2
running final init tasks on cpu 3
entering scheduler on cpu 3
[00000.835] K acpi_ec_init:98: Failed to find EC: 0
[00000.857] K initializing target
[00000.858] K calling apps_init()
[00000.858] K INIT: cpu 0, calling hook 0xffffffff801003a8 (pixel2_quirks) at level 0xa000a, flags 0x1
[00000.860] K INIT: cpu 0, calling hook 0xffffffff8012f9d4 (userboot) at level 0xaffff, flags 0x1
[00000.860] K userboot: console init
[00000.861] K userboot: bootfs          2347008 @ 0xffffffff80242000
[00000.861] K userboot: ramdisk       175804416 @ 0xffffff8015827000
[00001.080] K userboot: userboot-rodata       0 @ [0x1000000,0x1002000)
[00001.080] K userboot: userboot-code    0x2000 @ [0x1002000,0x1004000)
[00001.080] K userboot: vdso-rodata           0 @ [0x1004000,0x1006000)
[00001.080] K userboot: vdso-code        0x2000 @ [0x1006000,0x1007000)
[00001.081] K userboot: entry point             @ 0x1002740
[00001.088] U userboot: searching bootfs for "bin/devmgr"
[00001.091] U userboot: bin/devmgr has PT_INTERP "lib/ld.so.1"
[00001.092] U userboot: searching bootfs for "lib/ld.so.1"
[00001.107] U Loaded at [0x1500000,0x153b000): <application>
[00001.108] U Loaded at [0x1000000,0x10a8000): libc.so
[00001.110] U devmgr: main()
[00001.112] U devmgr: vfs init
[00001.114] U devmgr: bootfs #0 contains 70 files
[00001.114] U devmgr: bootfs #1 contains 48 files
[00001.118] U devmgr: load drivers
[00001.120] U devfs: vn 0x1000000a100 destroyed
[00001.120] U console_init()
[00001.122] U debug_reader()
[00001.125] U devmgr: remote(0x1000000abc0) for 'devhost:pci#1:1234:1111'
[00001.126] U devmgr: launch: devhost:pci#1:1234:1111 pci=1 0x1534090
[00001.213] U devmgr: remote(0x1000000b800) for 'devhost:pci#2:8086:100e'
[00001.214] U devmgr: launch: devhost:pci#2:8086:100e pci=2 0x15342d0
[00001.223] U dlsvc: debug: Loaded at [0x1500000,0x153b000): /boot/bin/devmgr
[00001.224] U dlsvc: debug: Loaded at [0x1000000,0x10a8000): libc.so
[00001.230] U devhost: pci host 1
[00001.272] U devmgr: remote(0x1000000b940) for 'devhost:pci#4:8086:2922'
[00001.273] U devmgr: launch: devhost:pci#4:8086:2922 pci=4 0x1534318
[00001.286] U dlsvc: debug: Loaded at [0x1500000,0x153b000): /boot/bin/devmgr
[00001.287] U dlsvc: debug: Loaded at [0x1000000,0x10a8000): libc.so
[00001.291] U devhost: pci host 2
[00001.328] U i8042 device 0xab 0x41
[00001.332] U dlsvc: debug: Loaded at [0x1500000,0x153b000): /boot/bin/devmgr
[00001.332] U dlsvc: debug: Loaded at [0x1000000,0x10a8000): libc.so
[00001.336] U devhost: pci host 4
[00002.409] U i8042 device 0x00
[00002.417] U devmgr: shell startup
[00002.423] U devmgr: launch /boot/bin/mxsh on /dev/console
[00002.424] U eth: using legacy irq mode
[00002.432] U eth: mac: XX:XX:XX:XX:XX:XX
[00002.433] U initialized vc on display bochs_vbe, width=1024 height=768 stride=1024 format=1
[00002.441] U eth: iomem @0x10c8000 (phys b7be000)
[00002.491] U vc: input thread started for /dev/class/input/000
[00002.638] U devmgr: launch /boot/bin/netsvc on /dev/console
[00002.722] U dlsvc: debug: Loaded at [0x1500000,0x1507000): /boot/bin/mxsh
[00002.723] U dlsvc: debug: Loaded at [0x1600000,0x160d000): libmxio.so
[00002.725] U dlsvc: debug: Loaded at [0x1700000,0x1707000): liblaunchpad.so
[00002.726] U dlsvc: debug: Loaded at [0x1800000,0x1803000): libmagenta.so
[00002.726] U dlsvc: debug: Loaded at [0x1000000,0x10a8000): libc.so
\end{lstlisting}

\subsection{Обзор Fuchsia}

Чтобы выключить или перезагрузить Fuchsia, используйте команду dm:

\begin{lstlisting}
dm help
dm shutdown
\end{lstlisting}

Почти все, что существует в системе Fuchsia, хранится в пакете Fuchsia. Типичный рабочий процесс разработки включает повторную сборку и передачу пакетов Fuchsia на устройство разработки или виртуальное устройство QEMU.\\

\par Например, можно внести изменения в файл rolldice в файле garnet / bin / rolldice / src / main.rs. \\

\par Перестройте и вставьте пакет rolldice в работающее устройство Fuchsia с помощью:

\begin{lstlisting}
fx build-push rolldice
\end{lstlisting}

Из shell на устройстве Fuchsia теперь можно запустить обновленный компонент rolldice с помощью:

\begin{lstlisting}
rolldice
\end{lstlisting}

\subsection{Запуск тестов}
Скомпилированные тестовые двоичные файлы устанавливаются в / pkgfs / packages / и на них ссылается URI. Можно запустить тест, вызвав его в терминале. Например.

\begin{lstlisting}
run fuchsia-pkg://fuchsia.com/ledger_tests#meta/ledger_unittests.cmx
\end{lstlisting}

Если необходимо оставить Fuchsia включенной, перекомпилировать и повторно запустить тест, требуется запустить Fuchsia с включенной сетью на одном терминале, а затем на другом терминале запустить:

\begin{lstlisting}
fx run-test <имя теста> [<test args>]
\end{lstlisting}

Также командой runtests можно запустить все тесты, ниже приведен лог некоторых из них: 
\begin{lstlisting}
> runtests

CASE bad_syscall_tests                                  [STARTED] 
    bad_syscall_test                                    [RUNNING] [PASSED] 
CASE bad_syscall_tests                                  [PASSED]
SUCCESS!  All test cases passed!

====================================================
    CASES:  1     SUCCESS:  1     FAILED:  0   
====================================================
PASSED: bad-syscall-test passed

CASE mtx_tests                                          [STARTED] 
    test_initializer                                    [RUNNING] [PASSED] 
    test_mutexes                                        [RUNNING] [PASSED] 
    test_try_mutexes                                    [RUNNING] [PASSED] 
    test_static_initializer                             [RUNNING] [PASSED] 
CASE mtx_tests                                          [PASSED]
SUCCESS!  All test cases passed!

====================================================
    CASES:  1     SUCCESS:  1     FAILED:  0   
====================================================
PASSED: c11-mutex-test passed

CASE c11_thread_tests                                   [STARTED] 
    c11_thread_test                                     [RUNNING] [PASSED] 
CASE c11_thread_tests                                   [PASSED]
SUCCESS!  All test cases passed!

====================================================
    CASES:  1     SUCCESS:  1     FAILED:  0   
====================================================
PASSED: c11-thread-test passed

CASE call_once_tests                                    [STARTED] 
    call_once_main_thread_test                          [RUNNING] [PASSED] 
    call_once_two_thread_test                           [RUNNING] [PASSED] 
CASE call_once_tests                                    [PASSED]
SUCCESS!  All test cases passed!

====================================================
    CASES:  1     SUCCESS:  1     FAILED:  0   
====================================================
PASSED: call_once-test passed

CASE cleanup_tests                                      [STARTED] 
    cleanup_test                                        [RUNNING] [PASSED] 
CASE cleanup_tests                                      [PASSED]
SUCCESS!  All test cases passed!

====================================================
    CASES:  1     SUCCESS:  1     FAILED:  0   
====================================================
PASSED: cleanup-test passed

CASE cprng_tests                                        [STARTED] 
    cprng_test_draw_buf_too_large                       [RUNNING] [PASSED] 
    cprng_test_draw_bad_buf                             [RUNNING] [PASSED] 
    cprng_test_draw_success                             [RUNNING] [PASSED] 
    cprng_test_add_entropy_buf_too_large                [RUNNING] [PASSED] 
    cprng_test_add_entropy_bad_buf                      [RUNNING] [PASSED] 
CASE cprng_tests                                        [PASSED]
SUCCESS!  All test cases passed!

====================================================
    CASES:  1     SUCCESS:  1     FAILED:  0   
====================================================
PASSED: cprng-test passed

CASE ctors                                              [STARTED] 
    check_ctor                                          [RUNNING] [PASSED] 
    check_initializer                                   [RUNNING] [PASSED] 
CASE ctors                                              [PASSED]
SUCCESS!  All test cases passed!

====================================================
    CASES:  1     SUCCESS:  1     FAILED:  0   
====================================================
PASSED: ctor-test passed

CASE data_pipe_tests                                    [STARTED] 
    create_destroy_test                                 [RUNNING] [PASSED] 
    simple_read_write                                   [RUNNING] [PASSED] 
    loop_write_full                                     [RUNNING] [PASSED] 
    write_read                                          [RUNNING] [PASSED] 
    begin_write_read                                    [RUNNING] [PASSED] 
    begin_write_read_large_request                      [RUNNING] [PASSED] 
    loop_write_read                                     [RUNNING] [PASSED] 
    loop_begin_write_read                               [RUNNING] [PASSED] 
    consumer_signals_when_producer_closed               [RUNNING] [PASSED] 
CASE data_pipe_tests                                    [PASSED]
SUCCESS!  All test cases passed!

====================================================
    CASES:  1     SUCCESS:  1     FAILED:  0   
====================================================
\end{lstlisting}

\subsection{Трассировка}
Fuchsia запускает программу при загрузке, которая называется «crashanalyzer», которая сообщает о сбоях программы и печатает обратную трассировку вызвавшего сбой потока.

\begin{lstlisting}
[11156.652165][1048][1187][klog] INFO: devmgr: crash_analyzer_listener: analyzing exception type 0x108
[11156.652215][1107][1121][klog] INFO: <== fatal exception: process crasher[42410] thread initial-thread[42424]
[11156.652218][1107][1121][klog] INFO: <== fatal page fault, PC at 0x1e1888dbbbd7
[11156.652223][1107][1121][klog] INFO:  CS:                   0 RIP:     0x1e1888dbbbd7 EFL:            0x10246 CR2:                  0
[11156.652226][1107][1121][klog] INFO:  RAX:                  0 RBX:                0x1 RCX:     0x721ad98697c6 RDX:     0x77accb36f264
[11156.652229][1107][1121][klog] INFO:  RSI:                  0 RDI:                  0 RBP:     0x2781c4816f90 RSP:     0x2781c4816f80
[11156.652232][1107][1121][klog] INFO:   R8:                  0  R9:                  0 R10:                  0 R11:              0x246
[11156.652235][1107][1121][klog] INFO:  R12:     0x773bf11dcda0 R13:     0x773bf11dcdd0 R14:               0x16 R15:         0x78050d69
[11156.652236][1107][1121][klog] INFO:  errc:               0x6
[11156.652237][1107][1121][klog] INFO: bottom of user stack:
[11156.652244][1107][1121][klog] INFO: 0x00002781c4816f80: f11dcda0 0000773b 9ccd2b38 000039b2 |....;w..8+...9..|
[11156.652248][1107][1121][klog] INFO: 0x00002781c4816f90: c4816fd0 00002781 88dbbba7 00001e18 |.o...'..........|
[11156.652252][1107][1121][klog] INFO: 0x00002781c4816fa0: 00000008 00000000 9ccd2b38 000039b2 |........8+...9..|
[11156.652255][1107][1121][klog] INFO: 0x00002781c4816fb0: f11dcf70 0000773b f11dcf70 0000773b |p...;w..p...;w..|
[11156.652260][1107][1121][klog] INFO: 0x00002781c4816fc0: cb36f570 000077ac f11dcdd0 0000773b |p.6..w......;w..|
[11156.652270][1107][1121][klog] INFO: 0x00002781c4816fd0: c4816ff0 00002781 cb2b0d0f 000077ac |.o...'....+..w..|
[11156.652277][1107][1121][klog] INFO: 0x00002781c4816fe0: 00000054 00000000 f11dcf70 0000773b |T.......p...;w..|
[11156.652281][1107][1121][klog] INFO: 0x00002781c4816ff0: f11dcfe0 0000773b 00000000 00000000 |....;w..........|
[11156.652282][1107][1121][klog] INFO: arch: x86_64
[11156.652550][1107][1121][klog] INFO: dso: id=a94c78564173530d51670b6586b1aa471e004f06 base=0x7d3506a49000 name=libfdio.so
[11156.652553][1107][1121][klog] INFO: dso: id=a61961ba9776a67a00fb322af9ebbdcfd1ce3f62 base=0x77accb297000 name=libc.so
[11156.652554][1107][1121][klog] INFO: dso: id=760f1e6e47d3dd8b6a19150aa47241279ec75a9c base=0x721ad9863000 name=<vDSO>
[11156.652558][1107][1121][klog] INFO: dso: id=b18462140c6784a53736105bbf3021852eeda68c base=0x1e1888dbb000 name=app:crasher
[11156.652637][1107][1121][klog] INFO: bt#01: pc 0x1e1888dbbbd7 sp 0x2781c4816f80 (app:crasher,0xbd7)
[11156.652750][1107][1121][klog] INFO: bt#02: pc 0x1e1888dbbba7 sp 0x2781c4816fa0 (app:crasher,0xba7)
[11156.652847][1107][1121][klog] INFO: bt#03: pc 0x77accb2b0d0f sp 0x2781c4816fe0 (libc.so,0x19d0f)
[11156.652978][1107][1121][klog] INFO: bt#04: pc 0 sp 0x2781c4817000
[11156.653027][1107][1121][klog] INFO: bt#05: end
\end{lstlisting}

Также через fx можно посмотреть системный лог fuchsia: 

\begin{lstlisting}
$ fx syslog | fx symbolize
[11156.652165][1048][1187][klog] INFO: devmgr: crash_analyzer_listener: analyzing exception type 0x108
... same output as "raw" backtrace ...
start of symbolized stack:
#01: blind_write at ../../zircon/system/uapp/crasher/crasher.c:21
#02: main at ../../zircon/system/uapp/crasher/crasher.c:137
#03: start_main at ../../zircon/third_party/ulib/musl/src/env/__libc_start_main.c:49
#04: unknown, can't find pc, sp or app/library in line
end of symbolized stack
\end{lstlisting}



\section{Вывод}
    В ходе выполнения лабораторной работы были разобраны особенности установки Fuchsia OS в ее настоящем виде. Был разобран вариант установки Fuchsia на устройство с usb носителя, а также через эмулирование в qemu. 
\end{document}