\documentclass[14pt,a4paper,report]{report}
\usepackage[a4paper, mag=1000, left=2.5cm, right=1cm, top=2cm, bottom=2cm, headsep=0.7cm, footskip=1cm]{geometry}
\usepackage[utf8]{inputenc}
\usepackage[english,russian]{babel}
\usepackage{indentfirst}
\usepackage[dvipsnames]{xcolor}
\usepackage[colorlinks]{hyperref}
\usepackage{listings} 
\usepackage{fancyhdr}
\usepackage{caption}
\usepackage{amsmath}
\usepackage{latexsym}
\usepackage{graphicx}
\usepackage{amsmath}
\usepackage{booktabs}
\usepackage{array}
\hypersetup{
 colorlinks = true,
 linkcolor  = black
}

\usepackage{listings}
\usepackage{color} %red, green, blue, yellow, cyan, magenta, black, white
\definecolor{MyDarkGreen}{rgb}{0.0,0.4,0.0}

\usepackage{titlesec}
\titleformat{\chapter}
{\Large\bfseries} % format
{}                % label
{0pt}             % sep
{\huge}           % before-code


\DeclareCaptionFont{white}{\color{white}} 

% Listing description
\usepackage{listings} 
\DeclareCaptionFormat{listing}{\colorbox{gray}{\parbox{\textwidth}{#1#2#3}}}
\captionsetup[lstlisting]{format=listing,labelfont=white,textfont=white}
\lstloadlanguages{Matlab}%
\lstset{language=Matlab,                        % Use MATLAB
        frame=single,                           % Single frame around code
        basicstyle=\small\ttfamily,             % Use small true type font
        keywordstyle=[1]\color{Blue}\bfseries,        % MATLAB functions bold and blue
        keywordstyle=[2]\color{Purple},         % MATLAB function arguments purple
        keywordstyle=[3]\color{Blue}\underbar,  % User functions underlined and blue
        identifierstyle=,                       % Nothing special about identifiers
                                                % Comments small dark green courier
        commentstyle=\usefont{T1}{pcr}{m}{sl}\color{MyDarkGreen}\small,
        stringstyle=\color{Purple},             % Strings are purple
        showstringspaces=false,                 % Don't put marks in string spaces
        tabsize=5,                              % 5 spaces per tab
        %
        %%% Put standard MATLAB functions not included in the default
        %%% language here
        morekeywords={xlim,ylim,var,alpha,factorial,poissrnd,normpdf,normcdf},
        %
        %%% Put MATLAB function parameters here
        morekeywords=[2]{on, off, interp},
        %
        %%% Put user defined functions here
        morekeywords=[3]{FindESS, homework_example},
        %
        morecomment=[l][\color{Blue}]{...},     % Line continuation (...) like blue comment
        numbers=left,                           % Line numbers on left
        firstnumber=1,                          % Line numbers start with line 1
        numberstyle=\tiny\color{Blue},          % Line numbers are blue
        stepnumber=5                            % Line numbers go in steps of 5
        }


\begin{document}

\def\contentsname{Содержание}

% Titlepage
\begin{titlepage}
 \begin{center}
  \textsc{Санкт-Петербургский Политехнический 
   Университет Петра Великого\\[5mm]
   Кафедра компьютерных систем и программных технологий}
  
  \vfill
  
  \textbf{Отчёт по лабораторной работе №1\\[3mm]
   Курс: «Методы оптимизации и принятия решений»\\[3mm]
   Тема: «Многокритериальная оптимизация»\\[35mm]
   }
 \end{center}
 
 \hfill
 \begin{minipage}{.5\textwidth}
  Выполнил студент:\\[2mm] 
  Волкова Мария Дмитриевна\\
  Группа: 13541/2\\[5mm]
  
  Проверил:\\[2mm] 
  Сиднев Александр Георгиевич
 \end{minipage}
 \vfill
 \begin{center}
  Санкт-Петербург\\ \the\year\ г.
 \end{center}
\end{titlepage}

% Contents
\tableofcontents
\clearpage

\chapter{Лабораторная работа №1}

\section{Цель работы}

Научиться решать задачи по многокритериальной оптимизации.

\section{Программа работы}

\begin{enumerate}
    \item Осуществить переход от многокритериальной задачи к однокритериальной с использованием следующих подходов:
    
\begin{itemize}
 \item Выделение главного критерия
 \item Свертка критериев (аддитивная и мультипликативная)
 \item Максимин или минимакс (он же метод максиминной свертки)
 \item Метод последовательных уступок
 \item fgoalattain
 \item Ведение метрики в пространстве критериев
\end{itemize}

    \item  Решить задачу стохастического программирования для одной из однокритериальных задач, превратив детерминированное ограничение в вероятностное по схеме
   
   $$ P(\sum_{j=1}^n a_{ij}x_{ij} - b_{ij} \leq 0) \geq a_i$$
   
Менять  $ a_i $  в следующем диапазоне $ 0.1 \leq a_i \leq 0.9 $.

Считать случайной величиной $b_i$  или элементы   $ {a_{ij}}i$-й строки матрицы  А  (по выбору).

Разрешается изменить формулировку исходной задачи, придумать собственную задачу, найти другую аналогичную задачу, которая могла бы быть сформулирована как многокритериальная.
\end{enumerate}

\section{Индивидуальное задание}

\subsubsection{Задача 5}



Фабрика производит два вида красок: первый – для наружных, а второй – для внутренних работ. 

Для производства красок используются два ингредиента: А и В. Максимально возможные суточные запасы этих ингредиентов составляют 6 и 8 т соответственно. Известны расходы А и В на 1 т соответствующих красок.

\begin{table}[h!]
\begin{tabular}{|l|l|l|l|}
\hline
\multirow{}{}{Ингридиент} & \multicolumn{2}{l|}{Расход. т} &\multirow{}{}{Запас, т} \\ \cline{2-3}
                    & Краска для наружных работ & Краска для внутренних работ     &               \\ \hline
А & 1 & 2 & 6 \\ \hline
Б & 2 & 1 & 8 \\  \hline
\end{tabular}
\end{table}

Изучение рынка сбыта показало, что суточный спрос на краску 1-го вида никогда не превышает спроса на краску 2-го вида более, чем на 4 т в сутки. Оптовая цена одной тонны краски равна 3000 рублей для первго вида и 2000 рублей за краску второго вида. Розничная цена одной тонны краски первого вида равно 5500 рублей, а второго вида - 3000 рублей. Выручка от  розничной продажи должна быть не менее 60 процентов общей выручки от розничной и оптовой продаж. 

Какое количество краски каждого ивда надо производить в условиях ограниченного количества ингридиентов А и В, чтобы доход от оптовой реализации продукции был максимальный? Какое количество краски каждого ивда надо производить, чтобы доход от розничной реализации продукции был максимальный? Какое количество краски каждого вида надо производить, чтобы остаток ингридиента А на складе был минимальный?


\subsection{Математическая модель задачи многокритериальной оптимизации}

\begin{itemize}
    \item $x_1$ - суточный объем краски для наружных работ для оптовой продажи
    
    \item $x_2$ - суточный объем краски  для внутренних работ для оптовой продажи 
    
    \item $x_3$ - суточный объем краски для наружных работ для розничной продажи
    
    \item $x_4$ - суточный объем краски  для внутренних работ для розничной продажи
\end{itemize}

\textbf{Ограничения:}
\begin{enumerate}
    \item Расход не превышает запасов $$ x_1 + 2x_2 + x_3 + 2x_4 \leq 6 $$ $$ 2x_1 + x_2 + 2x_3 + x_4 \leq 8 $$
    \item Объем производства не отрицательный $$ x_1 \geq 0 ,  x_2 \geq 0 ,  x_3 \geq 0 ,  x_4 \geq 0  $$
    \item Ограничение на спрос $$x_1+x_3 \leq x_2+x_4+4$$
    \item Ограничение по выручке $$ 5.5x_3 + 3x_4 \geq 0.6 (3x_1 + 2x_2 + 5.5x_3 + 3x_4) $$
\end{enumerate}

Запишем все в систему уравнений:

\begin{equation*}
 \begin{cases}
 x_1 + 2x_2 + x_3 + 2x_4 \leq 6 \\
  2x_1 + x_2 + 2x_3 + x_4 \leq 8 \\
  x_1+x_3 \leq x_2+x_4+4 \\
 5.5x_3 + 3x_4 \geq 0.6 (3x_1 + 2x_2 + 5.5x_3 + 3x_4) \\
  x_1 \geq 0 ,  x_2 \geq 0 ,  x_3 \geq 0 ,  x_4 \geq 0 
 \end{cases}
\end{equation*}



\textbf{Критерии:}
\begin{enumerate}
    \item Максимизация от оптовой продажи $$ F_1(x_1 , x_2) = 3x_1 + 2x_2 \rightarrow max $$
    
    \item Максимизация от розничной продажи $$ F_2(x_3 , x_4) = 5.5x_3 + 3x_4 \rightarrow max $$
    
    \item Минимизация остатка ингридиента А $$ F_3(x_1, x_2, x_3 , x_4) = 6 - x_1 - 2x_2 - x_3 - 2x_4  \rightarrow min $$
\end{enumerate}

\subsection{Поиск оптимумов частных критериев}
Найдем оптимумы каждой из целевых функций независимо от других. Для этого необходимо решить три задачи однокритериальной оптимизации: для $z_1, z_2, f_3$ при тех же ограничениях на $x_{1}, x_{2}, x_{3}, x_{4}$ что имеют место для задачи многокритериальной оптимизации.

Для решение данной задачи, был использован MATLAB.

\lstinputlisting{listings/0.m}

\lstinputlisting{listings/1.m}

\begin{enumerate}
    \item Максимизация дохода от оптовой продажи
    \begin{itemize}
        \item 1.4333 -объем краски  для наружных работ для оптовой продажи 
        \item 1.3333 - объем краски  для внутренних работ для оптовой продажи 
        \item 1.9 -  объем краски для наружных работ для розничной продажи
        \item 2.0221e-06 - объем краски  для внутренних работ для розничной продажи
        \item 6966.7 рублей дохода с оптовой продажи
    \end{itemize}
        \item Максимизация дохода от розничной продажи
            \begin{itemize}
        \item 7.2727e-08 -объем краски  для наружных работ для оптовой продажи 
        \item 1.3333e-07 - объем краски  для внутренних работ для оптовой продажи 
        \item 3.3333 -  объем краски для наружных работ для розничной продажи
        \item  1.3333 - объем краски  для внутренних работ для розничной продажи
        \item 22333 рублей дохода с розничной продажи
    \end{itemize}
    \item Минимизация остатка ингридиента А
    \begin{itemize}
        \item    0.48418 - -объем краски  для наружных работ для оптовой продажи 
        \item    0.7245- объем краски  для внутренних работ для оптовой продажи 
        \item 1.4535-  объем краски для наружных работ для розничной продажи
        \item   1.3066 - объем краски  для внутренних работ для розничной продажи
        \item 2e-06 - остаток ингридиента А (~0.000002 тонны = 2 грамма)
    \end{itemize}


\end{enumerate}

\section{Переход от многокритериальной задачи к однокритериальной}
\subsection{Выделение главного критерия}
%Один из критериев - главный - имеет существенно более высокий приоритет, чем все остальные. Пусть главный критерий - первый, тогда для двух оставшихся критериев составим ограничения.


Один из критериев - главный - имеет существенно более высокий приоритет, чем все остальные, но по остальным критериям вариант тоже не должен быть слишком плох. Пусть главный критерий - второй, следовательно, для оставшихся целевых функций необходимо указать нижние границы. Теперь, прибыль от оптовой продажи должна быть большее 5000 рублей, а остаток ингридиента А должен быть не больше 0.001 (1 кг).


$$ -x_1 -2x_2-x_3-2x_4 \leq 5.999 $$ 

$$ 3x_1+2x_2 \geq 5000 $$  


В соответствии с изменениями скрипт был дополнен ограничениями.
\lstinputlisting{listings/3.m}

После выполнения программы были получены следующие результаты:
\lstinputlisting{listings/4.m}

Таким образом было получено значение в 8233.3\$, что составляет 89\% от оптимума. 
Время использования станков - 1950 часов(упор в ограничение), доход на внешнем рынке - 1000\$(упор в ограничение).
\begin{itemize}
\item 0.77778 - объем краски для наружных работ для оптовой продажи
\item 1.3333 - объем краски  для внутренних работ для оптовой продажи 
\item 2.5556 - объем краски  для наружных работ для розничной продажи 
\item 0.000003 - объем краски  для внутренних работ для розничной продажи 
\item 14056 рублей доход от розничной продажи
\end{itemize}

Таким образом общий доход состовляет 19056 рублей, доход от оптовой торговли 5000 рублей, остаток ингридиента А 0.000014 тонны (14 грамм),




\subsection{Свертка критериев}
\subsubsection{Аддитивная свертка критериев}
Для использования метода аддитивной свертки необходимо выполнить нормировку критериев, с тем чтобы сделать их значения соизмеримыми, а единицы измерения – безразмерными. Выполним нормировку следующим образом:


    
    
\begin{equation}
\overline{z_1} = \frac{z_1}{|z_1^{min}|} =
-\frac{3x_1 + 2x_2}{6.966}
\end{equation}

\begin{equation}
\overline{z_2} = \frac{f_2}{|f_2^{min}|} = \frac{5.5x_3 + 3x_4} {22.333} 
\end{equation}

\begin{equation}
\overline{f_3} = \frac{z_3}{|z_3^{min}|} = \frac{6 - x_1 - 2x_2 - x_3 - 2x_4 }{0.000002} 
\end{equation}

Формула аддитивной свертки имеет вид:
\begin{equation}
F(x) = \sum_{i=1}^{r}\lambda_i f_i(x), 0<\lambda_i<1, \sum_i^{}\lambda_i=1,
\end{equation}
где $f_i(x)$ - критерии оптимальности, $r$ – их общее число, а $\lambda_i$ - параметры важности. 

\lstinputlisting{listings/5.m}

После выполнения программы были получены следующие результаты:
\lstinputlisting{listings/6.m}


Метод аддитивной свертки позволил получить решение:
\begin{itemize}
\item $f_1$ - 5744.6 сумма дохода от оптовой продажи
\item $f_2$ - 12248 сумма дохода от розничной продажи
\item $f_3$ - 0.000000000048547 остаток ингридиента А
\end{itemize}






\subsubsection{Мультипликативная свертка критериев}
Формула мультипликативной свертки имеет вид:
\begin{equation}
F(x) = \prod_{i=1}^{r}f_i(x)^{\lambda_i}
\end{equation}
где $f_i(x)$ - критерии оптимальности, $r$ - их общее число, а $\lambda_i$ - показатели важности. Нормировка уже была произведена в аддитивной свертки, в итоге получим следующую задачу однокритериальной оптимизации:

\begin{equation}
f = \overline{z_1}^{0.3}*\overline{z_2}^{0.3}*\overline{f_3}^{0.4}
\end{equation}

\lstinputlisting{listings/7.m}

После выполнения программы были получены следующие результаты:
\lstinputlisting{listings/8.m}


Результат:
\begin{itemize}
\item 5000 доход от оптовой продажи
\item 7500 доход от розничной продажи
\item 0.00000004 тонны остатка на скалде
\end{itemize}

Эта свертка привела к очень плохому результату f3, что, возможно, являеться результатом расставления коэффициентов $\lambda$ .





\subsection{Минимакс (максимин)}
Максиминную свертку представим в следующем виде: $$ C_i(a)= \text{min } w_i C_i(a) $$

Решение $a^*$ является наилучшим, если для всех $a$ выполняется условие $$C(a^*) \geq C(a) $$, или $$ a^* = \text{arg max } C(a) = \text{arg max min } w_i C_i (a) $$ .

Для реализации максиминной свертки необходимо в fminimax передавать функции обратные целевым (функция funminmax). Так как оцениваемые показатели разновелики, необходимо нормировать критерии. Что было произведено ранее.


\lstinputlisting{listings/10.m}

После выполнения программы были получены следующие результаты:
\lstinputlisting{listings/9.m}


Результат:
\begin{itemize}
\item 5012 доход от оптовой продажи
\item 7541.1 доход от розничной продажи
\item 0.0000000000000004409 тонны остатка на скалде
\end{itemize}

Процентное соотношение первого и второго критерия относительно оптимума примерно равное, второй критерий по сути игнорируется.





\subsection{Метод последовательных уступок}
Для решения данной задачи была выбрана уступка = 10\%. 	Предположим, что критерии пронумерованы в следующем порядке важности:
\begin{center}
$z_1>z_2>f_3$
\end{center} 
Для первого критерия уже решена задача поиска оптимального значения в п 1.2.1. То есть:
\begin{center}
$6966.7 * 0.9 = 6270.03$
\end{center}
То ограничения критерия выглядит следующим образом:
\begin{center}
$-0*x_{1}-2*x_{2}\leq -6270.03$
\end{center}


Запишем ограничения в скрипт
\lstinputlisting{listings/11.m}

После выполнения программы были получены следующие результаты:
\lstinputlisting{listings/12.m}


Как и ожидалось при минимизации функции $z_3$ учитывалось и ограничения для $z_1$, что существенное уменьшило результат для $z_3$.

В соответствии с полученным значением введем ограничение для второго критерия.

\begin{center}
$ 0.00000040083 * 0.9 = 3.60747 × 10^-7 $
\end{center}

Ограничения критерия выглядит следующим образом:

\begin{center}
$-x_1-2x_2-x_3-2x_4\leq -3.60747 * 10^-7$
\end{center}

\lstinputlisting{listings/14.m}

После выполнения программы были получены следующие результаты:
\lstinputlisting{listings/13.m}


Результат:
\begin{itemize}
\item 6966.7 доход от оптовой продажи
\item 20478 доход от розничной продажи
\item 0.00000040108 тонны остатка на скалде
\end{itemize}

По результатам видно, что за счет последовательности, при оптимизации каждого следующего критерия учитываются и уступка предыдущих критериев.















\subsection{Метод достижения цели (fgoalattain)}
Fgoalattain решает задачу достижения цели, которая является одной из формулировок задач для векторной оптимизации.
x = fgoalattain(fun, $x_0$, goal, weight, ...):
\begin{itemize}
\item fun – целевая функция,
\item $x_0$ – начальные значения,
\item goal – целевые значения,
\item weight – веса.
\end{itemize}
Пусть goal = ($z_1^{min}, f_2^{min}, z_3^{min}$), w = ($|z_1^{min}|, |f_2^{min}|, |z_3^{min}|$). Тогда скрипт для решения задачи будет выглядеть следующим образом:

\lstinputlisting{listings/16.m}

После выполнения программы были получены следующие результаты:
\lstinputlisting{listings/15.m}

Значение переменной af, говорит о том, что полученное решение на 51.62\% хуже цели. 


Результат:
\begin{itemize}
\item 3473.9 доход от оптовой продажи
\item 10805 доход от розничной продажи
\item 0.00000079426 тонны остатка на скалде
\end{itemize}






\subsection{Введение метрики в пространстве критериев}

Для перехода к однокритериальной задаче оптимизации методом введения метрики в пространстве целевых функций необходимо определить координаты «идеальной» точки $a=(f_1^*, f_2^*, ..., f_r^*)$,  где $f_i = min f_i(x)$. Эти значения
уже были получены в п. 1.2.1, и поэтому:



\begin{center}
$a = (-6966.7, 22333, 0.0000002)$
\end{center}

Введем в пространстве критериев метрику в виде евклидова расстояния:
\begin{equation}
p(y, a) = [\sum_{i=1}^r(a_i-y_i)^2]^{\frac{1}{2}} 
\end{equation}
Тогда за целевую функцию (обобщенный критерий), с учётом необходимости нормировки, можно взять выражение:
\begin{equation}
f=\sum_{i=1}^r(\frac{a_i-f_i}{f_i^*})^2=\sum_{i=1}^r(1-\frac{f_i}{f_i^*})^2
\end{equation}
Таким образом, получаем следующую задачу оптимизации:
\begin{equation}
f=(1-\frac{z_1}{z_1^{min}})^2+(1-\frac{z_2}{z_2^{min}})^2+(1-\frac{f_3}{f_3^{min}})^2
\end{equation}

\lstinputlisting{listings/18.m}

После выполнения программы были получены следующие результаты:
\lstinputlisting{listings/17.m}

Результат:
\begin{itemize}
\item 6891 доход от оптовой продажи
\item 10337 доход от розничной продажи
\item 0.0000019501 тонны остатка на скалде
\end{itemize}








\section{Оценка Парето-оптимальности полученных решений}
Для того чтобы уменьшить количество альтернатив, среди которых лицо, принимающее решение (ЛПР), должно сделать выбор, можно выделить множество Парето среди всех полученных решений. Для этого была составлена таблица и построен график.

\begin{table}[h!]
\begin{tabular}{|l|l|l|l|l|l|}
\hline
Метод & f_1 & f_2 & f_3 & f_1 + f_2  \\ \hline
Выделение главного критерия & 5000 & 14055 & 0.0000004 & 19056   \\ \hline
 Аддитивная свертка & 5744.6 & 12248 & 0.000000000048547 &  17992,6  \\ \hline
 Мультипликативная свертка & 5000 & 7500 & 0.00000004 & 12500    \\ \hline
Минимакс & 5012.1 & 7541.1 & 0.0000000000000004409 & 12553.2   \\ \hline
Метод последовательных уступок & 6966.7 & 20478 & 0.00000040108 &  27444,7  \\ \hline
fgoalattain & 3473.9 & 10805 & 0.00000079426 & 14278,9   \\ \hline
 Введение метрики в пространстве критериев & 6981 & 10337 & 0.0000019501 & 17318   \\ \hline
\end{tabular}
\end{table}


Парето-оптимальным являются решение в точке (19056;0.0000004). 


\section{Решение задачи стохастического программирования}
Рассмотрим задачу стохастического программирования на основе задачи однокритериальной оптимизации, которая была получена из исходной методом введения метрики в пространстве критериев.

Преобразуем второе ограничение системы:
\begin{center}
$2x_1+x_2+2x_3+x_4 \leq 8$
\end{center}

 в вероятностное, тогда:
\begin{center}
$P(\alpha_{1}x_{1}+\alpha_{2}x_{2}+\alpha_{3}x_{3}+\alpha_{4}x_{4} \leq 8)\geq \alpha$
\end{center}

где все $a_{i}$ нормально распределены и имеют следующие математические ожидания и дисперсии:

\begin{center}
$M(\alpha_{1})=2, M(\alpha_{2})=1, M(\alpha_{3})=2, M(\alpha_{4})=1$
$D(\alpha_{1})=1, D(\alpha_{2})=1, D(\alpha_{3})=1, D(\alpha_{4})=1$
\end{center}

Где СКО равно половине математического ожидания. По таблице функции распределения стандартного нормального закона  находим $K_a (0.1\leq \alpha\leq 0.9)$

\begin{center}
$K_{0.1}=-1.2816, 
K_{0.2}=-0.8416, 
K_{0.3}=-0.5244, 
K_{0.4}=-0.2533, 
K_{0.5}=0$
$K_{0.6}=0.2533, 
K_{0.7}=0.5244, 
K_{0.8}=0.8416, 
K_{0.9}=1.2816$
\end{center}

Таким образом, вероятностное ограничение становится эквивалентно детерминированному неравенству:	
\begin{equation}
2x_{1}+x_{2}+2*x_{3}+x_{4}+K_{\alpha}*\sqrt{x_{1}^2+x_{2}^2+x_{3}^2+x_{4}^2} \leq 8
\end{equation}

Решение задачи представлено как скрипт в программе Matlab

\lstinputlisting{listings/20.m}

Результаты выполнения программы приведены в таблице:

\begin{table}[h!]
\begin{tabular}{|l|l|l|l|l|l|l|l|}
\hline
a & x_1 & x_2 & x_3 & x_4 & f_1 & f_2 & f_3\\ \hline
0.1&1.6668&0.0000&1.6547& 1.339&5.00&13&0.00\\ \hline 
0.2&1.4141&1.3244&1.9069& 0.015&6.89&11&0.00\\ \hline
0.3&1.5812&1.0823&1.7464& 0.254&6.91&10&0.00\\ \hline
0.4&2.2318&0.0003&1.0981& 1.335&6.70&10&0.00\\ \hline
0.5&1.6511&0.4623&1.6797& 0.872&5.88&12&0.00\\ \hline
0.6&1.4688&0.7653&1.4689& 0.766&5.94&10&0.00\\ \hline
0.7&1.2848&0.8575&1.2848& 0.858&5.57&10&0.00\\ \hline
0.8&1.0915&0.9543&1.0915& 0.954&5.18&9&0.00\\ \hline
0.9&0.8401&1.0800&0.8401& 1.080&4.68&8&0.00  \\ \hline
\end{tabular}
\end{table}


Задача чувствительна к выбранному ограничению, т.к. для различных K получились разные результаты. 

Увеличение доверительной вероятности $\alpha$ приводит к ухудшению оценок получаемых решений по первому и второму критерию. 

При $\alpha < 0.5$ квантиль приобретает отрицательное значение. За счет чего ограничения \textbf{ослабевают}(их выполнение становится менее важным), и например решение при $\alpha = 0.5$ практические совпадает с оптимум для первого и второго критерия одновременно.

А при $\alpha > 0.5$ ограничения становяться более \textbf{жесткими}, что уменьшает общий доход.

Значение третьего критерия получить не удалось, так как решение чувствительно до 4 знаков после запятой.


\end{document}