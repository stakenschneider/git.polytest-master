\documentclass[14pt,a4paper,report]{report}
\usepackage[a4paper, mag=1000, left=2.5cm, right=1cm, top=2cm, bottom=2cm, headsep=0.7cm, footskip=1cm]{geometry}
\usepackage[utf8]{inputenc}
\usepackage[english,russian]{babel}
\usepackage{indentfirst}
\usepackage[dvipsnames]{xcolor}
\usepackage[colorlinks]{hyperref}
\usepackage{listings} 
\usepackage{fancyhdr}
\usepackage{caption}
\usepackage{amsmath}
\usepackage{latexsym}
\usepackage{graphicx}
\usepackage{amsmath}
\usepackage{booktabs}
\usepackage{array}
\hypersetup{
	colorlinks = true,
	linkcolor  = black
}

\usepackage{titlesec}
\titleformat{\chapter}
{\Large\bfseries} % format
{}                % label
{0pt}             % sep
{\huge}           % before-code


\DeclareCaptionFont{white}{\color{white}} 

% Listing description
\usepackage{listings} 
\DeclareCaptionFormat{listing}{\colorbox{gray}{\parbox{\textwidth}{#1#2#3}}}
\captionsetup[lstlisting]{format=listing,labelfont=white,textfont=white}
\lstset{ 
	% Listing settings
	inputencoding = utf8,			
	extendedchars = \true, 
	keepspaces = true, 			  	 % Поддержка кириллицы и пробелов в комментариях
	language = Matlab,            	 	 % Язык программирования (для подсветки)
	basicstyle = \small\sffamily, 	 % Размер и начертание шрифта для подсветки кода
	numbers = left,               	 % Где поставить нумерацию строк (слева\справа)
	numberstyle = \tiny,          	 % Размер шрифта для номеров строк
	stepnumber = 1,               	 % Размер шага между двумя номерами строк
	numbersep = 5pt,              	 % Как далеко отстоят номера строк от подсвечиваемого кода
	backgroundcolor = \color{white}, % Цвет фона подсветки - используем \usepackage{color}
	showspaces = false,           	 % Показывать или нет пробелы специальными отступами
	showstringspaces = false,    	 % Показывать или нет пробелы в строках
	showtabs = false,           	 % Показывать или нет табуляцию в строках
	frame = single,              	 % Рисовать рамку вокруг кода
	tabsize = 2,                  	 % Размер табуляции по умолчанию равен 2 пробелам
	captionpos = t,             	 % Позиция заголовка вверху [t] или внизу [b] 
	breaklines = true,           	 % Автоматически переносить строки (да\нет)
	breakatwhitespace = false,   	 % Переносить строки только если есть пробел
	escapeinside = {\%*}{*)}      	 % Если нужно добавить комментарии в коде
}

\begin{document}

\def\contentsname{Содержание}

% Titlepage
\begin{titlepage}
	\begin{center}
		\textsc{Санкт-Петербургский Политехнический 
			Университет Петра Великого\\[5mm]
			Кафедра компьютерных систем и программных технологий}
		
		\vfill
		
		\textbf{Отчёт по лабораторной рабте №3\\[3mm]
			Курс: «Методы оптимизации и принятия решений»\\[3mm]
			Тема: «Марковские модели принятия решений»\\[35mm]
			}
	\end{center}
	
	\hfill
	\begin{minipage}{.5\textwidth}
		Выполнил студент:\\[2mm] 
		Волкова Мария Дмитриевна\\
		Группа: 13541/3\\[5mm]
		
		Проверил:\\[2mm] 
		Сиднев Александр Георгиевич
	\end{minipage}
	\vfill
	\begin{center}
		Санкт-Петербург\\ \the\year\ г.
	\end{center}
\end{titlepage}

% Contents
\tableofcontents
\clearpage

\chapter{Лабораторная работ №3}

\section{Индивидуальное задание}

 Фирма может рекламировать свою продукцию с помощью радио (допустимое решение X1), телевидения (допустимое решение X2) и газет (допустимое решение X3). Недельные затраты на рекламу с помощью этих средств равны 200, 900 и 300 денежных единиц соответственно. Фирма может оценить недельный объём сбыта как удовлетворительный (состояние S1), хороший (состояние S2) и отличный (состояние S3). Матрицы переходных вероятностей для каждого из трёх средств массовой информации имеют вид

$P_1 = 
 \begin{pmatrix}
0,4 & 0,5 & 0,1 \\
0,1 & 0,7 & 0,2 \\
0,1 & 0,7 & 0,2 \\
\end{pmatrix}$
$P_2 = 
 \begin{pmatrix}
0,7 & 0,2 & 0,1 \\
0,3 & 0,6 & 0,1 \\ 
0,1 & 0,7 & 0,2 \\ 
\end{pmatrix}$
$P_3 = 
 \begin{pmatrix}
0,2 & 0,5 & 0,3 \\
0 & 0,7 & 0,3 \\ 
0 & 0,2 & 0,8 \\ 
\end{pmatrix}$

а соответствующие им недельные доходы (в денежных единицах) заданыматрицами

$R_1 = 
 \begin{pmatrix}
 400 & 520 & 600 \\ 
300 & 400 & 700 \\ 
200 & 250 & 500 \\ 
\end{pmatrix}$
$R_2 = 
 \begin{pmatrix}
1000 & 1300 & 1600 \\ 
800 & 1000 & 1700 \\ 
600 & 700 & 1100 \\ 
\end{pmatrix}$
$R_3 = 
\begin{pmatrix}
400 & 530 & 710 \\ 
350 & 450 & 800 \\ 
250 & 400 & 650 \\ 
\end{pmatrix}$




в которых не учтены затраты на рекламу, при этом необходимо учитывать коэффициент дисконтирования (если он задан). Найти оптимальную стратегию рекламы для последующих N недель и при бесконечном горизонте планирования.

\section{Ход работы}

\subsection{Решения и стратегии}

Состояния:

\begin{itemize}
	\item 1 -- удовлетворительно;
	\item 2 -- хорошо;
	\item 3 -- отлично;
\end{itemize}

Для данных состояний имеется три решения:

\begin{itemize}
	\item $X_1$ -- рекламировать на радио;
	\item $X_2$ -- рекламировать в газете;
	\item $X_3$ -- рекламировать на телевиденье;
\end{itemize}



\subsection{Модель динамического программирования с конечным числом этапов}

В нашем случае число этапов – 3 (недели), число состояний для каждого m = 3 (удовлетворительный, хороший, отличный).

Вычислим значения:


$$   v_i^k = \sum_{j=1}^m  p_{ij}^k * r_{ij}^k $$

$ v_1^1 = 0.4*400 + 0.5*520 + 0.1*600 = 480$

$ v_2^1 = 0.1*300 + 0.7*400 + 0.2*700 = 450$

$ v_3^1 = 0.1*200 + 0.7*250 + 0.2*500 = 295$

$ v_1^2 = 0.7*1000 + 0.2*1300 + 0.1*1600 = 1120$

$ v_2^2 = 0.3*800 + 0.6*1000 + 0.1*1700 = 1010$

$ v_3^2 = 0.1*600 + 0.7*700 + 0.2*1100 = 770$


$ v_1^3 = 0.2*400 + 0.5*530 + 0.3*710 = 558$

$ v_2^3 = 0*350 + 0.7*450 + 0.3*800 = 555$

$ v_3^3 = 0*250 + 0.2*400 + 0.8*650 = 600$

Запишем это в таблицу:

\begin{table}[h!]
	\centering
	\bgroup
	\def\arraystretch{1}
	\begin{tabular}{ | c | c | c | c | }
		\hline
		i & $ v_i^1 $ & $ v_i^1 $  & $ v_i^1 $ \\ \hline
		1 & 480 & 1120 & 558 \\ \hline
		2 & 450 & 1010 & 555 \\ \hline
		3 & 295 & 770 & 600 \\ \hline
	\end{tabular}
	\egroup
\end{table}

С учетом затрат на каждую стратегию (200, 900, 300):

\begin{table}[h!]
	\centering
	\bgroup
	\def\arraystretch{1}
	\begin{tabular}{ | c | c | c | c | c | c | }
	\hline
		  & \multicolumn{3}{c|}{$v_i^k$ }&   \multicolumn{2}{c|}{оптимальное решение} \\ \hline
		i & $ v_i^1 $ & $ v_i^1 $  & $ v_i^1 $ & $ f_3(i) $ & k \\ \hline
		1 & \textcolor{red}{280} & 220 & 258 & 280 & 1\\ \hline
		2 & 250 & 110 & \textcolor{red}{255} & 255 & 3\\ \hline
		3 & 95 & -130 & \textcolor{red}{300} & 300 & 3\\ \hline
	\end{tabular}
	\egroup
\end{table}




\begin{table}[h!]
	\centering
	\bgroup
	\def\arraystretch{1}
	\begin{tabular}{ | c | c | c | c | c | c | }
	\hline
	\multicolumn{6}{| c |}{Этап 3 } \\ \hline
		  & \multicolumn{3}{c|}{$v_i^k$ }&   \multicolumn{2}{c|}{оптимальное решение} \\ \hline
		i & k = 1 & k = 2  &  k = 3 & $ f_3(i) $ & k \\ \hline
		1 & \textcolor{red}{280} & 220 & 258 & 280 & 1\\ \hline
		2 & 250 & 110 & \textcolor{red}{255} & 255 & 3\\ \hline
		3 & 95 & -130 & \textcolor{red}{300} & 300 & 3\\ \hline
		
			\multicolumn{6}{| c |}{Этап 2} \\ \hline
		  & \multicolumn{3}{c|}{$v_i^k + p_{i1}^k * f_3(1) + p_{i2}^k * f_3(2) + p_{i3}^k * f_3(3)$ }&   \multicolumn{2}{c|}{оптимальное решение} \\ \hline
		i & k = 1 &  k = 2  & k = 3 & $ f_3(i) $ & k \\ \hline
		1 & \textcolor{red}{349.5} & -403 & 231.5 & 349.5 & 1\\ \hline
		2 & \textcolor{red}{316.5} & -523 & 223.5 & 316.5 & 1\\ \hline
		3 & 161.5 & -763.5 & \textcolor{red}{291} & 291 & 3\\ \hline
		
			\multicolumn{6}{| c |}{Этап 1} \\ \hline
		  & \multicolumn{3}{c|}{$v_i^k + p_{i1}^k * f_2(1) + p_{i2}^k * f_2(2) + p_{i3}^k * f_2(3)$} &   \multicolumn{2}{c|}{оптимальное решение} \\ \hline
		i & k = 1 &  k = 2  & k = 3 & $ f_3(i) $ & k \\ \hline
		1 & \textcolor{red}{476.65} & -965.95 & 246.95 & 476.65 & 1\\ \hline
		2 & \textcolor{red}{431.2} & -1099.15 & 232.35 & 431.2 & 1\\ \hline
		3 & 276.2 & -1348.8 & \textcolor{red}{287.1} & 287.1 & 3\\ \hline
	\end{tabular}
	\egroup
\end{table}

Оптимальное решение показывает, что в 1-ый и 2-ой месяцы фирме следует рекламировать свою продукцию по радио в случае удовлетворительного и хорошего объема продаж и рекламировать в газетах , при условии, что уровень недельного объема продаж будет отличным.

В 3-ем месяце фирме следует рекламировать свою продукцию по радио при удовлетворительном недельном объеме продаж, в остальных случаях (хорошем и отличном) в газетах.

Суммарный ожидаемый доход за 3 месяца составит 476.65   при удовлетворительном  уровне продаж в 1-ый месяц, 431.2 при хорошем уровне и 287.1 - при удовлетворительном уровне продаж в 1-ый месяц.


Все расчеты проводились в Matlab:
\begin{lstlisting}[language={matlab}, caption={скрипт}, basicstyle=\ttfamily]
% матрицы переходных вероятностей 
p111 = 0.4; p112 = 0.5; p113 = 0.1;
p121 = 0.1; p122 = 0.7; p123 = 0.2;
p131 = 0.1; p132 = 0.7; p133 = 0.2;

p211 = 0.7; p212 = 0.2; p213 = 0.1;
p221 = 0.3; p222 = 0.6; p223 = 0.1;
p231 = 0.1; p232 = 0.7; p233 = 0.2;

p311 = 0.2; p312 = 0.5; p313 = 0.3;
p321 = 0; p322 = 0.7; p323 = 0.3;
p331 = 0; p332 = 0.2; p333 = 0.8;

% недельные доходы
r111 = 400; r112 = 520; r113 = 600;
r121 = 300; r122 = 400; r123 = 700;
r131 = 200; r132 = 250; r133 = 500;

r211 = 1000; r212 = 1300; r213 = 1600;
r221 = 800; r222 = 1000; r223 = 1700;
r231 = 600; r232 = 700; r233 = 1100;

r311 = 400; r312 = 530; r313 = 710;
r321 = 350; r322 = 450; r323 = 800;
r331 = 250; r332 = 400; r333 = 650;

% недельные затраты
o1 = 200;
o2 = 900;
o3 = 300;


% 3 этап
% радио
pnew111 = p111*r111 + p112*r112 + p113*r113 - o1;
pnew112 = p121*r121 + p122*r122 + p123*r123 - o1;
pnew113 = p131*r131 + p132*r132 + p133*r133 - o1;
% телевиденье
pnew121 = p211*r211 + p212*r212 + p213*r213 - o2;
pnew122 = p221*r221 + p222*r222 + p223*r223 - o2;
pnew123 = p231*r231 + p232*r232 + p233*r233 - o2;
% газета
pnew131 = p311*r311 + p312*r312 + p313*r313 - o3;
pnew132 = p321*r321 + p322*r322 + p323*r323 - o3;
pnew133 = p331*r331 + p332*r332 + p333*r333 - o3;


% 2 этап
% радио
pnew211 = pnew111 + p111*pnew111 + p112*pnew132 + p113*pnew133 - o1;
pnew212 = pnew112 + p121*pnew111 + p122*pnew132 + p123*pnew133 - o1;
pnew213 = pnew113 + p131*pnew111 + p132*pnew132 + p133*pnew133 - o1;
% телевиденье
pnew221 = pnew121 + p211*pnew111 + p212*pnew132 + p213*pnew133 - o2;
pnew222 = pnew122 + p221*pnew111 + p222*pnew132 + p223*pnew133 - o2;
pnew223 = pnew123 + p231*pnew111 + p232*pnew132 + p233*pnew133 - o2;
% газета
pnew231 = pnew131 + p311*pnew111 + p312*pnew132 + p313*pnew133 - o3;
pnew232 = pnew132 + p321*pnew111 + p322*pnew132 + p323*pnew133 - o3;
pnew233 = pnew133 + p331*pnew111 + p332*pnew132 + p333*pnew133 - o3;


% 1 этап
% радио
pnew311 = pnew211 + p111*pnew211 + p112*pnew212 + p113*pnew233 - o1
pnew312 = pnew212 + p121*pnew211 + p122*pnew212 + p123*pnew233 - o1
pnew313 = pnew213 + p131*pnew211 + p132*pnew212 + p133*pnew233 - o1
% телевиденье
pnew321 = pnew221 + p211*pnew211 + p212*pnew212 + p213*pnew233 - o2
pnew322 = pnew222 + p221*pnew211 + p222*pnew212 + p223*pnew233 - o2
pnew323 = pnew223 + p231*pnew211 + p232*pnew212 + p233*pnew233 - o2
% газета
pnew331 = pnew231 + p311*pnew211 + p312*pnew212 + p313*pnew233 - o3
pnew332 = pnew232 + p321*pnew211 + p322*pnew212 + p323*pnew233 - o3
pnew333 = pnew233 + p331*pnew211 + p332*pnew212 + p333*pnew233 - o3
\end{lstlisting}

\subsection{Решение задачи с бесконечным числом этапов методом полного перебора}
В данной задаче принятия решений имеется $3^3$ = 27 стационарных стратегий поведения, представленных в следующей таблице.

\scalebox{0.8}{
	\begin{tabular}{  l  l  l  }

	\begin{tabular}{ | c | c | c | c | c | c | c | }
	\hline
	\multicolumn{6}{| c |}{1}\\ \hline
\multicolumn{3}{| c |}{P} & \multicolumn{3}{c |}{R} \\ \hline
		0,4 & 0,5 & 0,1   & 400 & 520 & 600 \\ \hline
		 0,1 & 0,7 & 0,2   & 300 & 400 & 700\\ \hline
		 0,1 & 0,7 & 0,2    & 200 & 250 & 500  \\ \hline
			 		 	\multicolumn{6}{| c |}{2}\\ \hline
\multicolumn{3}{| c |}{P} & \multicolumn{3}{c |}{R} \\ \hline
		0,4 & 0,5 & 0,1   & 400 & 520 & 600 \\ \hline
		 0,1 & 0,7 & 0,2   & 300 & 400 & 700\\ \hline
			0,1 & 0,7 & 0,2    & 600 & 700 & 1100 \\ \hline	 
				 		 	\multicolumn{6}{| c |}{3}\\ \hline
\multicolumn{3}{| c |}{P} & \multicolumn{3}{c |}{R} \\ \hline
		0,4 & 0,5 & 0,1   & 400 & 520 & 600 \\ \hline
		 0,1 & 0,7 & 0,2   & 300 & 400 & 700\\ \hline		
					 0 & 0,2 & 0,8   & 250 & 400 & 650  \\ \hline
					 	\multicolumn{6}{| c |}{4}\\ \hline

			\multicolumn{3}{| c |}{P} & \multicolumn{3}{c |}{R} \\ \hline
		0,4 & 0,5 & 0,1   & 400 & 520 & 600 \\ \hline
		 0,3 & 0,6 & 0,1  & 800 & 1000 & 1700 \\ \hline
		 0,1 & 0,7 & 0,2    & 200 & 250 & 500  \\ \hline
		 	\multicolumn{6}{| c |}{5}\\ \hline

	\multicolumn{3}{| c |}{P} & \multicolumn{3}{c |}{R} \\ \hline
		0,4 & 0,5 & 0,1   & 400 & 520 & 600 \\ \hline
		 0,3 & 0,6 & 0,1  & 800 & 1000 & 1700 \\ \hline
		0,1 & 0,7 & 0,2    & 600 & 700 & 1100 \\ \hline	 
			\multicolumn{6}{| c |}{6}\\ \hline

		 			\multicolumn{3}{| c |}{P} & \multicolumn{3}{c |}{R} \\ \hline
		0,4 & 0,5 & 0,1   & 400 & 520 & 600 \\ \hline
		 0,3 & 0,6 & 0,1  & 800 & 1000 & 1700 \\ \hline
		 0 & 0,2 & 0,8   & 250 & 400 & 650  \\ \hline
		 	\multicolumn{6}{| c |}{7}\\ \hline

		 \multicolumn{3}{| c |}{P} & \multicolumn{3}{c |}{R} \\ \hline
		0,4 & 0,5 & 0,1   & 400 & 520 & 600 \\ \hline
		 0 & 0,7 & 0,3   & 350 & 450 & 800 \\ \hline
		 0,1 & 0,7 & 0,2    & 200 & 250 & 500  \\ \hline
		 	\multicolumn{6}{| c |}{8}\\ \hline

		 \multicolumn{3}{| c |}{P} & \multicolumn{3}{c |}{R} \\ \hline
		0,4 & 0,5 & 0,1   & 400 & 520 & 600 \\ \hline
		 0 & 0,7 & 0,3   & 350 & 450 & 800 \\ \hline
				0,1 & 0,7 & 0,2    & 600 & 700 & 1100 \\ \hline	 
					\multicolumn{6}{| c |}{9}\\ \hline

		 \multicolumn{3}{| c |}{P} & \multicolumn{3}{c |}{R} \\ \hline
		0,4 & 0,5 & 0,1   & 400 & 520 & 600 \\ \hline
		 0 & 0,7 & 0,3   & 350 & 450 & 800 \\ \hline
		 0 & 0,2 & 0,8   & 250 & 400 & 650  \\ \hline
				\end{tabular}  & \begin{tabular}{ | c |  c | c | c | c | c | c | }
	\hline	 
	
		\multicolumn{6}{| c |}{10}\\ \hline
\multicolumn{3}{| c |}{P} & \multicolumn{3}{c |}{R} \\ \hline
	    0,7 & 0,2 & 0,1  & 1000 & 1300 & 1600 \\ \hline
		 0,3 & 0,6 & 0,1  & 800 & 1000 & 1700 \\ \hline
		0,1 & 0,7 & 0,2    & 600 & 700 & 1100 \\ \hline	 
			 	\multicolumn{6}{| c |}{11}\\ \hline
\multicolumn{3}{| c |}{P} & \multicolumn{3}{c |}{R} \\ \hline
	    0,7 & 0,2 & 0,1  & 1000 & 1300 & 1600 \\ \hline
		 0,3 & 0,6 & 0,1  & 800 & 1000 & 1700 \\ \hline
		 		 0,1 & 0,7 & 0,2    & 200 & 250 & 500  \\ \hline
		 			 	\multicolumn{6}{| c |}{12}\\ \hline
\multicolumn{3}{| c |}{P} & \multicolumn{3}{c |}{R} \\ \hline
	    0,7 & 0,2 & 0,1  & 1000 & 1300 & 1600 \\ \hline
		 0,3 & 0,6 & 0,1  & 800 & 1000 & 1700 \\ \hline
		 		 0 & 0,2 & 0,8   & 250 & 400 & 650  \\ \hline
		\multicolumn{6}{| c |}{13}\\ \hline

	 \multicolumn{3}{| c |}{P} & \multicolumn{3}{c |}{R} \\ \hline
	    0,7 & 0,2 & 0,1  & 1000 & 1300 & 1600 \\ \hline
		 0,1 & 0,7 & 0,2   & 300 & 400 & 700\\ \hline
		 0,1 & 0,7 & 0,2    & 200 & 250 & 500  \\ \hline
		\multicolumn{6}{| c |}{14}\\ \hline

	\multicolumn{3}{| c |}{P} & \multicolumn{3}{c |}{R} \\ \hline
	    0,7 & 0,2 & 0,1  & 1000 & 1300 & 1600 \\ \hline
		 0,1 & 0,7 & 0,2   & 300 & 400 & 700\\ \hline
		0,1 & 0,7 & 0,2    & 600 & 700 & 1100 \\ \hline	 
		\multicolumn{6}{| c |}{15}\\ \hline

	\multicolumn{3}{| c |}{P} & \multicolumn{3}{c |}{R} \\ \hline
	    0,7 & 0,2 & 0,1  & 1000 & 1300 & 1600 \\ \hline
		 0,1 & 0,7 & 0,2   & 300 & 400 & 700\\ \hline
		 0 & 0,2 & 0,8   & 250 & 400 & 650  \\ \hline
		\multicolumn{6}{| c |}{16}\\ \hline

	\multicolumn{3}{| c |}{P} & \multicolumn{3}{c |}{R} \\ \hline
	    0,7 & 0,2 & 0,1  & 1000 & 1300 & 1600 \\ \hline
		 0 & 0,7 & 0,3   & 350 & 450 & 800 \\ \hline
		 0,1 & 0,7 & 0,2    & 200 & 250 & 500  \\ \hline
		\multicolumn{6}{| c |}{17}\\ \hline

	\multicolumn{3}{| c |}{P} & \multicolumn{3}{c |}{R} \\ \hline
	    0,7 & 0,2 & 0,1  & 1000 & 1300 & 1600 \\ \hline
		 0 & 0,7 & 0,3   & 350 & 450 & 800 \\ \hline
		0,1 & 0,7 & 0,2    & 600 & 700 & 1100 \\ \hline	 
		\multicolumn{6}{| c |}{18}\\ \hline

	\multicolumn{3}{| c |}{P} & \multicolumn{3}{c |}{R} \\ \hline
	    0,7 & 0,2 & 0,1  & 1000 & 1300 & 1600 \\ \hline
		 0 & 0,7 & 0,3   & 350 & 450 & 800 \\ \hline
		 0 & 0,2 & 0,8   & 250 & 400 & 650  \\ \hline
				\end{tabular}  & \begin{tabular}{ | c |  c | c | c | c | c | c | }
	\hline	 
		 	\multicolumn{6}{| c |}{19}\\ \hline
\multicolumn{3}{| c |}{P} & \multicolumn{3}{c |}{R} \\ \hline
	    0,2 & 0,5 & 0,3  & 400 & 530 & 710 \\ \hline
		 0 & 0,7 & 0,3   & 350 & 450 & 800 \\ \hline
		 0 & 0,2 & 0,8   & 250 & 400 & 650  \\ \hline
			 	\multicolumn{6}{| c |}{20}\\ \hline
\multicolumn{3}{| c |}{P} & \multicolumn{3}{c |}{R} \\ \hline
	    0,2 & 0,5 & 0,3  & 400 & 530 & 710 \\ \hline
		 0 & 0,7 & 0,3   & 350 & 450 & 800 \\ \hline
		 		 0,1 & 0,7 & 0,2    & 200 & 250 & 500  \\ \hline
		 		 	\multicolumn{6}{| c |}{21}\\ \hline
\multicolumn{3}{| c |}{P} & \multicolumn{3}{c |}{R} \\ \hline
	    0,2 & 0,5 & 0,3  & 400 & 530 & 710 \\ \hline
		 0 & 0,7 & 0,3   & 350 & 450 & 800 \\ \hline
		 		0,1 & 0,7 & 0,2    & 600 & 700 & 1100 \\ \hline	 
		 		 	\multicolumn{6}{| c |}{22}\\ \hline
\multicolumn{3}{| c |}{P} & \multicolumn{3}{c |}{R} \\ \hline
	    0,2 & 0,5 & 0,3  & 400 & 530 & 710 \\ \hline
		 0,1 & 0,7 & 0,2   & 300 & 400 & 700\\ \hline
		 0,1 & 0,7 & 0,2    & 200 & 250 & 500  \\ \hline
		 		 	\multicolumn{6}{| c |}{23}\\ \hline
\multicolumn{3}{| c |}{P} & \multicolumn{3}{c |}{R} \\ \hline
	    0,2 & 0,5 & 0,3  & 400 & 530 & 710 \\ \hline
		 0,1 & 0,7 & 0,2   & 300 & 400 & 700\\ \hline
		0,1 & 0,7 & 0,2    & 600 & 700 & 1100 \\ \hline	 
		 		 	\multicolumn{6}{| c |}{24}\\ \hline
\multicolumn{3}{| c |}{P} & \multicolumn{3}{c |}{R} \\ \hline
	    0,2 & 0,5 & 0,3  & 400 & 530 & 710 \\ \hline
		 0,1 & 0,7 & 0,2   & 300 & 400 & 700\\ \hline
		 		 0 & 0,2 & 0,8   & 250 & 400 & 650  \\ \hline
		 		 	\multicolumn{6}{| c |}{25}\\ \hline
\multicolumn{3}{| c |}{P} & \multicolumn{3}{c |}{R} \\ \hline
	    0,2 & 0,5 & 0,3  & 400 & 530 & 710 \\ \hline
		 0,3 & 0,6 & 0,1  & 800 & 1000 & 1700 \\ \hline
		 0,1 & 0,7 & 0,2    & 200 & 250 & 500  \\ \hline
		 		 	\multicolumn{6}{| c |}{26}\\ \hline
\multicolumn{3}{| c |}{P} & \multicolumn{3}{c |}{R} \\ \hline
	    0,2 & 0,5 & 0,3  & 400 & 530 & 710 \\ \hline
		 0,3 & 0,6 & 0,1  & 800 & 1000 & 1700 \\ \hline
		0,1 & 0,7 & 0,2    & 600 & 700 & 1100 \\ \hline	 
		
		 		 	\multicolumn{6}{| c |}{27}\\ \hline
\multicolumn{3}{| c |}{P} & \multicolumn{3}{c |}{R} \\ \hline
	    0,2 & 0,5 & 0,3  & 400 & 530 & 710 \\ \hline
		 0,3 & 0,6 & 0,1  & 800 & 1000 & 1700 \\ \hline
		 		 0 & 0,2 & 0,8   & 250 & 400 & 650  \\ \hline
	\end{tabular}
\end{tabular}	}


Результаты вычислений $v_i^s$ приведены в таблице.


 \begin{tabular}{| c | c | c | c | }
	\hline	 
	s & i=1 & i=2 & i=3 \\ \hline
	1 & 480 & 450 & 295 \\ \hline
	2 & 480 & 450 & 770 \\ \hline
	3 & 480 & 450 & 600 \\ \hline
	
	4 & 480 & 1010 & 295 \\ \hline
	5 & 480 & 1010 & 770 \\ \hline
	6 & 480 & 1010 & 600 \\ \hline
	
	7 & 480 & 555 & 295 \\ \hline
	8 & 480 & 555 & 770 \\ \hline
	9 & 480 & 555 & 600 \\ \hline
	
	10 & 1120 & 1010 & 770 \\ \hline
	11 & 1120 & 1010 & 295 \\ \hline
	12 & 1120 & 1010 & 600 \\ \hline
	
	13 & 1120 & 450 & 295 \\ \hline
	14 & 1120 & 450 & 770 \\ \hline
	15 & 1120 & 450 & 600 \\ \hline
	
	16 & 1120 & 555 & 295 \\ \hline
	17 & 1120 & 555 & 770 \\ \hline
	18 & 1120 & 555 & 600 \\ \hline
	
	19 & 558 & 555 & 600 \\ \hline
	20 & 558 & 555 & 295 \\ \hline
	21 & 558 & 555 & 770 \\ \hline
	
	22 & 558 & 450 & 295 \\ \hline
	23 & 558 & 450 & 770 \\ \hline
	24 & 558 & 450 & 600 \\ \hline
	
	25 & 558 & 1010 & 295 \\ \hline
	26 & 558 & 1010 & 770 \\ \hline
	27 & 558 & 1010 & 600 \\ \hline
	\end{tabular}
	
	Стационарные вероятности находятся из уравнени:

$$ \pi^s * P = \pi^s $$
$$ \pi_1 + \pi_2  + ... + \pi_m = 1 $$


	(Отметим, что одно из первых трех уравнений избыточно.) Решение системы будет:
	
	$$ 0.4*\pi_1 +  0.1* \pi_2 + 0.1* \pi_3 = \pi_1 $$

	$$ 0.5*\pi_1 +  0.7* \pi_2 + 0.7* \pi_3 = \pi_1 $$

	$$ 0.1*\pi_1 +  0.7* \pi_2 + 0.2* \pi_3 = \pi_1 $$

	$$ \pi_1+ \pi_2 + \pi_3 = 1 $$

(Отметим, что одно из первых трех уравнений избыточно.) Решение системы будет:

$\pi_1^1 = 0.14$  

$\pi_2^1 = 0.67$  

$\pi_3^1 = 0.18 $

$ E^! = 421.8$
	
	Результаты вычисления для всех стационарных стратегий приведены в следующей таблице.
	
	 \begin{tabular}{| c | c | c | c | c | }
	\hline	 
	s & $\pi_1$ & $\pi_2$ & $\pi_3$ & $E_s$\\ \hline
	1 & 0.14 & 0.67 & 0.18   & 421.8\\ \hline
	2 & 0.14 & 0.67 & 0.18 & 507.3\\ \hline
	3 & 0.07 & 0.4 & 0.48 & 501.6\\ \hline
	
	4 & 0.31 & 0.58 & 0.1 & 764.1\\ \hline
	5 & 0.31 & 0.58 & 0.1 & 811.6\\ \hline
	6 & 0.3 & 0.4 & 0.3 & 728\\ \hline
	
	7 & 0.04 & 0.69 & 0.26 & 461.57\\ \hline
	8 & 0.04 & 0.69 & 0.26 & 585.07\\ \hline
	9 & 0 & 0.4 & 0.6 & 582\\ \hline
	
	10 & 0.46 & 0.42 & 0.1 & \textcolor{red}{1016.4}\\ \hline
	11 & 0.46 & 0.42 & 0.1 & 968.9\\ \hline
	12 & 0.3 & 0.3 & 0.4 & 920\\ \hline
	
	13 & 0.25 & 0.575 & 0.175 & 590.375\\ \hline
	14 & 0.25 & 0.575 & 0.175 & 673.5\\ \hline
	15 & 0.13 & 0.4 & 0.46 & 601.6\\ \hline
	
	16 & 0.08 & 0.66 & 0.26 & 532.6\\ \hline
	17 & 0.08 & 0.66 & 0.26 & 651.6\\ \hline
	18 & 0 & 0.4 & 0.6 & 582\\ \hline
	
	19 & 0 & 0.4 & 0.6 & 582\\ \hline
	20 & 0.03 & 0.69 & 0.27 & 479.34\\ \hline
	21 & 50.03 & 0.69 & 0.27 & 607.59\\ \hline
	
	22 & 0.1 & 0.67 & 0.21 & 387.75\\ \hline
	23 & 0.1 & 0.67 & 0.21 & 487.5\\ \hline
	24 & 0.05 & 0.43 & 0.51 & 527.4\\ \hline
	
	25 & 0.24 & 0.59 & 0.16 & 777.02\\ \hline
	26 & 0.24 & 0.59 & 0.16 & 841.82\\ \hline
	27 & 0.15 & 0.41 & 0.41 & 743.8\\ \hline
	\end{tabular}
	
	Вывод: Из таблицы видно, что стратегия 10 (релкамировать товар на телевиденье при любом объеме недельного сбыта) дает наибольший ожидаемый месячный доход. Следовательно, это и есть оптимальная долгосрочная стратегия (без учета затрат на рекламу).

	
\section{Вывод}

Рассмотренная модель марковских процессов принятия решений позволяет решать задачи принятия решений в условиях риска при заданном одном критерии, либо нескольких, приведенных к одному. Система допущений, используемых в модели и приводящих реальную ситуацию к описанию с помощью
марковских процессов, ограничивает применение метода классом задач принятия решений, в которых можно принять допущение о дискретном времени, зависимости текущего состояния системы только от предшествующего и скачкообразном изменении состояния системы. Несмотря на эти ограничения, модель позволяет решать задачи принятия решений, сводящихся к классической задаче управления запасами.

Основным методом решения марковских задач принятия решений в данной работе является метод линейного программирования. Модель линейного программирования позволяет найти решения задачи за конечное число шагов, при этом не используя сложных и приближенных математических методов. Это облегчает поиск решения.

Решением задачи является вектор решений — стратегия — обеспечивающий оптимальное значение выбранного критерия. В качестве принципа оптимальности выбрана максимизация ожидаемых доходов на заданном числе этапов.

\end{document}