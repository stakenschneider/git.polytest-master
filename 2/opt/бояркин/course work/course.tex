\documentclass[a4paper, 12pt]{article}		% general format
\usepackage{multicol}
%%%% Charset
\usepackage{cmap}							% make PDF files searchable and copyable
\usepackage[utf8x]{inputenc} 				% accept different input encodings
\usepackage[english,russian]{babel}   %% загружает пакет многоязыковой вёрстки
\usepackage{fontspec}      %% подготавливает загрузку шрифтов Open Type, True Type и др.
\defaultfontfeatures{Ligatures={TeX},Renderer=Basic}  %% свойства шрифтов по умолчанию
\setmainfont[Ligatures={TeX,Historic}]{Roboto-Light} %% задаёт основной шрифт документа
\setsansfont{Roboto-Light}  
\usepackage{float}
%%%% Graphics
\usepackage[dvipsnames]{xcolor}			% driver-independent color extensions
\usepackage{graphicx}						% enhanced support for graphics
\usepackage{wrapfig}						% produces figures which text can flow around
\usepackage{hyperref}
%%%% Math
\usepackage{amsmath}						% American Mathematical Society (AMS) math facilities
\usepackage{amsfonts}						% fonts from the AMS
\usepackage{amssymb}						% additional math symbols

%%%% Typograpy (don't forget about cm-super)
\usepackage{microtype}						% subliminal refinements towards typographical perfection
\linespread{1.3}							% line spacing
\usepackage[left=2.5cm, right=1.5cm, top=2.5cm, bottom=2.5cm]{geometry}
\setlength{\parindent}{0pt}					% we don't want any paragraph indentation
\usepackage{parskip}						% some distance between paragraphs

%%%% Tables
\usepackage{tabularx}						% tables with variable width columns
\usepackage{multirow}						% for tabularx
\usepackage{hhline}							% for tabularx
\usepackage{tabu}
\usepackage{longtable}

%%%% Graph
\usepackage{tikz}							% package for creating graphics programmatically
\usetikzlibrary{arrows}						% edges for tikz

%%%% Other
\usepackage{url}							% verbatim with URL-sensitive line breaks
\usepackage{fancyvrb}						% sophisticated verbatim text (with box)

\usepackage{listings}
\usepackage{caption}
\DeclareCaptionFont{white}{\color{white}}
\DeclareCaptionFormat{listing}{\colorbox{gray}{\parbox{\dimexpr\textwidth-1.72\fboxsep\relax}{#1#2#3}}}
\captionsetup[lstlisting]{format=listing,labelfont=white,textfont=white,margin=0pt}
\lstset{language=C,
	basicstyle=\footnotesize,
	keepspaces=true,
	tabsize=4,               
	frame=single,                           % Single frame around code
	rulecolor=\color{black},
	captionpos=b,
	showstringspaces=false,	
	abovecaptionskip=-0.9pt,
	xleftmargin=3.4pt,
	xrightmargin=2.6pt,
	breaklines=true,
	postbreak=\raisebox{0ex}[0ex][0ex]{\ensuremath{\color{black}\hookrightarrow\space}},
	xleftmargin=3.2pt,
	literate={а}{{\selectfont\char224}}1
	{~}{{\textasciitilde}}1
	{б}{{\selectfont\char225}}1
	{в}{{\selectfont\char226}}1
	{г}{{\selectfont\char227}}1
	{д}{{\selectfont\char228}}1
	{е}{{\selectfont\char229}}1
	{ё}{{\"e}}1
	{ж}{{\selectfont\char230}}1
	{з}{{\selectfont\char231}}1
	{и}{{\selectfont\char232}}1
	{й}{{\selectfont\char233}}1
	{к}{{\selectfont\char234}}1
	{л}{{\selectfont\char235}}1
	{м}{{\selectfont\char236}}1
	{н}{{\selectfont\char237}}1
	{о}{{\selectfont\char238}}1
	{п}{{\selectfont\char239}}1
	{р}{{\selectfont\char240}}1
	{с}{{\selectfont\char241}}1
	{т}{{\selectfont\char242}}1
	{у}{{\selectfont\char243}}1
	{ф}{{\selectfont\char244}}1
	{х}{{\selectfont\char245}}1
	{ц}{{\selectfont\char246}}1
	{ч}{{\selectfont\char247}}1
	{ш}{{\selectfont\char248}}1
	{щ}{{\selectfont\char249}}1
	{ъ}{{\selectfont\char250}}1
	{ы}{{\selectfont\char251}}1
	{ь}{{\selectfont\char252}}1
	{э}{{\selectfont\char253}}1
	{ю}{{\selectfont\char254}}1
	{я}{{\selectfont\char255}}1
	{А}{{\selectfont\char192}}1
	{Б}{{\selectfont\char193}}1
	{В}{{\selectfont\char194}}1
	{Г}{{\selectfont\char195}}1
	{Д}{{\selectfont\char196}}1
	{Е}{{\selectfont\char197}}1
	{Ё}{{\"E}}1
	{Ж}{{\selectfont\char198}}1
	{З}{{\selectfont\char199}}1
	{И}{{\selectfont\char200}}1
	{Й}{{\selectfont\char201}}1
	{К}{{\selectfont\char202}}1
	{Л}{{\selectfont\char203}}1
	{М}{{\selectfont\char204}}1
	{Н}{{\selectfont\char205}}1
	{О}{{\selectfont\char206}}1
	{П}{{\selectfont\char207}}1
	{Р}{{\selectfont\char208}}1
	{С}{{\selectfont\char209}}1
	{Т}{{\selectfont\char210}}1
	{У}{{\selectfont\char211}}1
	{Ф}{{\selectfont\char212}}1
	{Х}{{\selectfont\char213}}1
	{Ц}{{\selectfont\char214}}1
	{Ч}{{\selectfont\char215}}1
	{Ш}{{\selectfont\char216}}1
	{Щ}{{\selectfont\char217}}1
	{Ъ}{{\selectfont\char218}}1
	{Ы}{{\selectfont\char219}}1
	{Ь}{{\selectfont\char220}}1
	{Э}{{\selectfont\char221}}1
	{Ю}{{\selectfont\char222}}1
	{Я}{{\selectfont\char223}}1,
	extendedchars=true
}

%галочка
\usepackage{amssymb}% http://ctan.org/pkg/amssymb
\usepackage{pifont}% http://ctan.org/pkg/pifont
\newcommand{\cmark}{\ding{52}}%
\newcommand{\xmark}{\ding{56}}
%------------------------------------------------------------------------------
\renewcommand{\labelenumii}{\theenumii}
\renewcommand{\theenumii}{\theenumi.\arabic{enumii}.}
\begin{document}
\begin{titlepage}
\thispagestyle{empty}

\begin{center}
Санкт-Петербургский политехнический университет Петра Великого\\
Институт Информационных Технологий и Управления\\*
Кафедра компьютерных систем и программных технологий\\*
\hrulefill
\end{center}

\vspace{15em}

\begin{center}
\textsc{\textbf{Курсовая работа}}
\vspace{1em}

Дисциплина: \textbf{Методы оптимизации}
\vspace{2em}

Тема: \textbf{Формулировка и решение задачи выбора оптимального решения с использованием различных математических моделей}
\end{center}

\vspace{16em}

\begin{flushleft}
Выполнил студент гр. 53501/3 \hrulefill С.А. Мартынов \\
\vspace{1.5em}
Руководитель, к.т.н.,доц. \hrulefill А.Г. Сиднев\\
\end{flushleft}

\vspace{\fill}

\begin{center}
Санкт-Петербург \\
2015
\end{center}

\end{titlepage}
\setcounter{page}{2}
%\includepdf[pages=-,pagecommand={},width=\textwidth]{task.pdf}
\tableofcontents
\clearpage

%------------------------------------------------------------------------------
%\input{intro}

\addcontentsline{toc}{chapter}{Введение}
\chapter*{Введение}
В данной работе рассматриваются следующие задачи:
\begin{enumerate}
\item Формализация многокритериальной оптимизационной задачи, методы сведения к однокритериальной, решение с использованием Optimization Toolbox системы Matlab;
\item Поиск оптимальной стратегии принятия решений с использованием марковских моделей;
\item Оптимизация сетей систем массового обслуживания;
\item Решение задачи анализа потокового графа с использованием методики GERT и алгебры потоковых графов.
\end{enumerate}

\chapter{Формализация многокритериальной оптимизационной задачи, методы сведения к однокритериальной, решение с использованием Optimization Toolbox системы Matlab.}

\section{Постановка задачи}

\subsubsection{Вариант 14}

Компания Nakia выпускает под своим брендом телефоны трёх ценовых сегментов:

\begin{enumerate}
	\item LowEnd – розничная цена аппарата 60\$, стоимость производства 30\$
	\item MiddleEnd – розничная цена 300\$, стоимость производства 100\$
	\item HighEnd – цена в розничной сети 2000\$, стоимость производства 220\$
\end{enumerate}

Мощность фабрик компании такова, что достижимые объёмы выпуска: 70 млн дешевых устройств, 30 млн устройств среднего сегмента, 1 млн дорогих телефонов. Розничная сеть может продать не более 80 млн устройств в год.

Доход фирмы от продажи дополнений, можно заранее оценить по формуле $F=(0.1x_1+10x_2+70\sqrt{x_3})^{\frac{3}{2}}$.

При выпуске больших партий дешевых телефонов неизбежен антирекламный эффект, обусловленный относительно высоким процентом брака и падением престижа марки. Ущерб от антирекламы можно оценить по формуле $F=ln(20x_1+3x_2+0.01x_3)$.

По каждому из сегментов компания должна производить не менее половины
максимального объема выпуска.

Необходимо найти годовой объём производства телефонов каждого сегмента
для достижения

\begin{enumerate}
	\item Максимизации оборота
	\item Минимизации затрат на производство
	\item Максимизации средней цены телефона
	\item Максимизации выручки от продажи дополнений
	\item Минимизации антирекламного эффекта
\end{enumerate}

\subsection{Решение}

\subsection{Обозначения}

Для решения задачи будем использовать следующие обозначения:

\begin{itemize}
	\item $x_1$ -- количество произведенных дешевых телефонов (в миллионах).
	\item $x_2$ -- количество произведенных телефонов среднего сегмента (в миллионах).
	\item $x_3$ -- количество произведенных дорогих телефонов (в миллионах).
\end{itemize}

\subsection{Критерии}

Введем следующие функции для определения критериев:

\begin{equation*}
\begin{cases}
\text{$F_1=60x_1+300x_2+2000x_3$} \\
\text{$F_2=(0.1x_1+10x_2+70\sqrt{x_3})^{\frac{3}{2}}$} \\
\text{$F_3=30x_1+100x_2+220x_3$} \\
\text{$F_4=ln(20x_1+3x_2+0.01x_3)$} \\
\text{$F_5=F_1/(x_1+x_2+x_3)$} \\
\text{$F_6=(F_1+F_2)-(F_3+F_4)$} \\
\end{cases}
\end{equation*}

Тогда задача сводится к минимизации или максимизации следующих функций:

\begin{enumerate}
	\item Максимизация оборота -- $F_6\rightarrow max$
	\item Минимизация затрат на производство -- $F_3\rightarrow min$
	\item Максимизация средней цены телефона -- $F_5\rightarrow max$
	\item Максимизация выручки от продажи дополнений -- $F_2\rightarrow max$
	\item Минимизация антирекламного эффекта -- $F_4\rightarrow min$
\end{enumerate}

Значения взяты в миллионах для повышения точности вычислений в MATLAB, особенно для сверток. Кроме того, такое решение подразумевается исходя из задания.

\textbf{* Функция $F_1$ используется только для вычисления критерия $F_6$, поэтому не будет впоследствии расчитываться отдельно.}

\subsection{Ограничения}

Формализуем ограничения, приведенные в формулировке задания:

\begin{equation*}
\begin{cases}
\text{$x_1\leq 70$} \\
\text{$-x_1\leq -35$} \\
\text{$x_2\leq 30$} \\
\text{$-x_2\leq -15$} \\
\text{$x_3\leq 1$} \\
\text{$-x_3\leq -0.5$} \\
\text{$x_1+x_2+x_3\leq 80$} \\
\end{cases}
\end{equation*}

\subsection{Поиск оптимумов частных критериев}

Разработаем программу для MATLAB, которая решает задачу, в соответствии с ограничениями, для каждого из критериев (Приложение 1).

Результат решения задачи:

\lstinputlisting{listings/p1.m1.log}

Результирующие значения в формате таблицы:

\begin{table}[h!]
	\centering
	\bgroup
	\captionsetup{singlelinecheck = false, format= hang, justification=raggedleft, font=footnotesize, labelsep=space}
	\caption{Результирующие значения поиска оптимумов}
	\def\arraystretch{1}
	\begin{tabular}{ | m{1.2cm} | m{1.2cm} | m{1.2cm} | m{1.2cm} | m{1.6cm} | m{1.6cm} | m{1.6cm} | m{1.6cm} | m{1.6cm} | }
		\hline
		& $x_1, Mill$ & $x_2, Mill$ & $x_3, Mill$ & $F_2, Mill\$$ & $F_3, Mill\$$ & $F_4, Mill\$$ & $F_5, Mill\$$ & $F_6, Mill\$$  \\ \hline
		F2 max & $49$ & $30$ & $1$ & $\underline{7258.939}$ & $4689.998$ & $6.975$ & $174.250$ & $16501.961$  \\ \hline
		F3 min & $35$ & $15$ & $0.5$ & $2892.251$ & $\underline{2660.000}$ & $6.613$ & $150.495$ & $7825.638$  \\ \hline
		F4 min & $35$ & $15$ & $0.528$ & $2921.812$ & $2666.229$ & $\underline{6.613}$ & $151.530$ & $7905.533$  \\ \hline
		F5 max & $35$ & $30$ & $1$ & $7218.316$ & $4270.000$ & $6.672$ & $\underline{198.485}$ & $16041.644$  \\ \hline
		F6 max & $49$ & $30$ & $1$ & $7258.939$ & $4690.000$ & $6.975$ & $174.250$ & $\underline{16501.964}$  \\
		\hline
	\end{tabular}
	\egroup
\end{table}

\textbf{* Функция $F_1$ используется только для вычисления критерия $F_6$, поэтому не расчитывается отдельно.}

\subsection{Аддитивная свертка критериев}

Для использования метода аддитивной свертки необходимо выполнить нормировку критериев, с тем чтобы сделать их значения соизмеримыми, а единицы измерения безразмерными. Нормировка производится делением функции критерия на модуль ее минимума или максимума.

\begin{equation*}
\text{$f_i(x)=F_i(x)/|F_i^{extr}|$}
\end{equation*}

Формула аддитивной свертки имеет вид:

\begin{equation*}
\text{$F_a(x)=\sum_{i=1}^{N}\lambda_if_i(x), 0<\lambda_i<1,\sum_{i}^{}\lambda_i=1,$}
\end{equation*}

где $f_i(x)$ - критерии оптимальности, $N$ – их общее число, а $\lambda_i$ -- коэффициенты важности. Примем коэффициенты важности равными $\lambda_2=0.15, \lambda_3=0.15, \lambda_4=0.05, \lambda_5=0.05, \lambda_6=0.6$. Коэффициент при $F_6$ очевидно наибольший, так как итоговый оборот интересует прежде всего.

Решение задачи при помощи аддитивной свертки (Приложение 2).

Результат решения при помощи аддитивной свертки:

\lstinputlisting{listings/p1.m2.log}

Стоит отметить, что при больших значениях минимума или максима функций критерии оптимальности становятся очень маленькими, и поэтому MATLAB выдает не совсем корректные результаты. Это еще раз подтверждает правильность взятия значений $x1, x2, x3$ в миллионах.

\begin{table}[h!]
	\centering
	\bgroup
	\captionsetup{singlelinecheck = false, format= hang, justification=raggedleft, font=footnotesize, labelsep=space}
	\caption{Результирующие значения метода аддитивной свертки}
	\def\arraystretch{1}
	\begin{tabular}{ | m{1.5cm} | m{1.9cm} | m{2.2cm} | m{2.2cm} | }
		\hline
		& result, Mill\$ & proportion, \% & difference, \% \\ \hline
		F2 max & 7218.316 & 99.44 & 0.56 \\ \hline
		F3 min & 4270.000 & 160.526 & 60.526 \\ \hline
		F4 min & 6.672 & 100.893 & 0.893 \\ \hline
		F5 max & 174.25 & 100 & 0 \\ \hline
		F6 max & 16501.96 & 97.211 & 2.789 \\ \hline
		Mean & & & \textbf{12.95} \\
		\hline
	\end{tabular}
	\egroup
\end{table}

\textbf{* Функция $F_1$ используется только для вычисления критерия $F_6$, поэтому не расчитывается отдельно.}

\subsection{Мультипликативная свертка критериев}

Формула мультипликативной свертки имеет вид:

\begin{equation*}
\text{$F_m(x)=\prod_{i=1}^{N}f_i(x)^{\lambda_i}, 0<\lambda_i<1,\sum_{i}^{}\lambda_i=1,$}
\end{equation*}

где $f_i(x)$ - критерии оптимальности, $N$ – их общее число, а $\lambda_i$ -- коэффициенты важности. Коэффициенты важности оставим равными $\lambda_2=0.15, \lambda_3=0.15, \lambda_4=0.05, \lambda_5=0.05, \lambda_6=0.6$.

Решение задачи при помощи аддитивной свертки (Приложение 3).

Результат решения при помощи мультипликативной свертки:

\lstinputlisting{listings/p1.m3.log}

По результирующему значению мультипликативной свертки можно заметить, что она лучше справляется в ситуации с очень маленькими критериями оптимальности, чем адаптивная свертка.

\begin{table}[h!]
	\centering
	\bgroup
	\captionsetup{singlelinecheck = false, format= hang, justification=raggedleft, font=footnotesize, labelsep=space}
	\caption{Результирующие значения метода мультипликативной свертки}
	\def\arraystretch{1}
	\begin{tabular}{ | m{1.5cm} | m{1.9cm} | m{2.2cm} | m{2.2cm} | }
		\hline
		& result, Mill\$ & proportion, \% & difference, \% \\ \hline
		F2 max & 7218.315 & 99.44 & 0.56 \\ \hline
		F3 min & 4270.004 & 160.526 & 60.526 \\ \hline
		F4 min & 6.672 & 100.893 & 0.893 \\ \hline
		F5 max & 198.485 & 100 & 0 \\ \hline
		F6 max & 16041.646 & 97.211 & 2.789 \\ \hline
		Mean & & & \textbf{12.95} \\
		\hline
	\end{tabular}
	\egroup
\end{table}

Результат аналогичен аддитивной свертке.

\textbf{* Функция $F_1$ используется только для вычисления критерия $F_6$, поэтому не расчитывается отдельно.}

\subsection{Максимин или минимакс}

Максиминную свертку представим в следующем виде:

\begin{equation*}
\text{$C_i(a)=min w_iC_i(a)$}
\end{equation*}

Решение $a^*$ является наилучшим, если для всех a выполняется условие:

\begin{equation*}
\text{$C(a^*)\geq C(a)$}
\end{equation*}

или

\begin{equation*}
\text{$a^*=arg max C(a)=arg max min w_iC_i(a)$}
\end{equation*}

Решение задачи в среде MATLAB (Приложение 4).

Результат решения задачи:

\lstinputlisting{listings/p1.m4.log}

Результат решения задачи в виде таблицы:

\begin{table}[h!]
	\centering
	\bgroup
	\captionsetup{singlelinecheck = false, format= hang, justification=raggedleft, font=footnotesize, labelsep=space}
	\caption{Результирующие значения метода максимин}
	\def\arraystretch{1}
	\begin{tabular}{ | m{1.5cm} | m{1.9cm} | m{2.2cm} | m{2.2cm} | }
		\hline
		& result, Mill\$ & proportion, \% & difference, \% \\ \hline
		F2 max & 5364.414 & 73.901 & 26.099 \\ \hline
		F3 min & 3599.419 & 135.317 & 35.317 \\ \hline
		F4 min & 6.646 & 100.503 & 0.503 \\ \hline
		F5 max & 187.004 & 94.216 & 5.784 \\ \hline
		F6 max & 12846.607 & 77.849 & 22.151 \\ \hline
		Mean & & & \textbf{17.97} \\
		\hline
	\end{tabular}
	\egroup
\end{table}

\textbf{* Функция $F_1$ используется только для вычисления критерия $F_6$, поэтому не расчитывается отдельно.}

\subsection{Метод последовательных уступок}

Для метода последовательных уступок данные целевые функции не очень подходят, ввиду их нелинейности. Введем три новых критерия, которые хорошо иллюстрируют метод последовательных уступок:

\begin{equation*}
\begin{cases}
\text{$F_7=30x_1+200x_2+1780x_3$} \\
\text{$F_8=x_1+10x_3$} \\
\text{$F_9=x_1+2x_2$} \\
\end{cases}
\end{equation*}

Тогда задача сводится к минимизации или максимизации следующих функций:

\begin{enumerate}
	\item Максимизация оборота (без антирекламного эффекта и продажи дополнений) -- $F_7\rightarrow max$
	\item Максимизация доли дешевых и дорогих телефонов -- $F_8\rightarrow max$
	\item Минимизация доли средних и дешевых телефонов -- $F_9\rightarrow min$
\end{enumerate}

Расположим критерии в порядке значимости:

\begin{equation*}
\text{$F_7>F_8>F_9$}
\end{equation*}

Для решения задачи была выбрана уступка равная $15\%$.

Решение задачи для критерия $F_7$ (Приложение 5).

Результат решения задачи для критерия $F_7$:

\lstinputlisting{listings/p1.m5p1.log}

Результаты решения для критерия $F_7$ в виде таблицы:

\begin{table}[h!]
	\centering
	\bgroup
	\captionsetup{singlelinecheck = false, format= hang, justification=raggedleft, font=footnotesize, labelsep=space}
	\caption{Результирующие значения метода последовательных уступок для $F_7$}
	\def\arraystretch{1}
	\begin{tabular}{ | m{1.2cm} | m{1.9cm} | m{2.2cm} | m{2.2cm} | }
		\hline
		& result, Mill\$ & proportion, \% & difference, \% \\ \hline
		F7 & 9250 & 100 & 0 \\ \hline
		F8 & 59 & - & - \\ \hline
		F9 & 109 & - & - \\
		\hline
	\end{tabular}
	\egroup
\end{table}

Максимум целевой функции $F_7$ равен $9250$. Для расчета максимума $F_8$ будет добавлено новое ограничение:

\begin{equation*}
\text{$F_7<=9250*0.85$}
\end{equation*}

\begin{equation*}
\text{$F_7<=7862.5$}
\end{equation*}

Решение задачи для критерия $F_8$ (Приложение 5).

Результат решения задачи для критерия $F_8$:

\lstinputlisting{listings/p1.m5p2.log}

Результаты решения для критерия $F_8$ в виде таблицы:

\begin{table}[h!]
	\centering
	\bgroup
	\captionsetup{singlelinecheck = false, format= hang, justification=raggedleft, font=footnotesize, labelsep=space}
	\caption{Результирующие значения метода последовательных уступок для $F_8$}
	\def\arraystretch{1}
	\begin{tabular}{ | m{1.2cm} | m{1.9cm} | m{2.2cm} | m{2.2cm} | }
		\hline
		& result, Mill\$ & proportion, \% & difference, \% \\ \hline
		F7 & 7862.5 & 85 & 15 \\ \hline
		F8 & 67.162 & 100 & 0 \\ \hline
		F9 & 100.838 & - & - \\
		\hline
	\end{tabular}
	\egroup
\end{table}

Максимум целевой функции $F_8$ равен $67.162$. Для расчета максимума $F_9$ будет добавлено новое ограничение:

\begin{equation*}
\text{$F_8<=67.162*0.85$}
\end{equation*}

\begin{equation*}
\text{$F_8<=57.088$}
\end{equation*}

Решение задачи для критерия $F_9$ (Приложение 5).

Результат решения задачи для критерия $F_9$:

\lstinputlisting{listings/p1.m5p3.log}

Результаты решения для критерия $F_9$ в виде таблицы:

\begin{table}[h!]
	\centering
	\bgroup
	\captionsetup{singlelinecheck = false, format= hang, justification=raggedleft, font=footnotesize, labelsep=space}
	\caption{Результирующие значения метода последовательных уступок для $F_9$}
	\def\arraystretch{1}
	\begin{tabular}{ | m{1.2cm} | m{1.9cm} | m{2.2cm} | m{2.2cm} | }
		\hline
		& result, Mill\$ & proportion, \% & difference, \% \\ \hline
		F7 & 7862.5 & 85 & 15 \\ \hline
		F8 & 57.088 & 85 & 15 \\ \hline
		F9 & 93.786 & 100 & 0 \\
		\hline
	\end{tabular}
	\egroup
\end{table}

Отличие результатов целевых функций от максимальных или минимальных значений не превышают принятое значение уступки $15\%$.

\subsection{Метод достижения цели (fgoalattain)}

Функция fgoalattain решает задачу достижения цели, которая является одной из формулировок задач для векторной оптимизации. Аргументы fgoalattain схожи с функцией fgoalattain, за исключением добавления целевых значений и весов. Кроме того, одновременно ищутся оптимальные значения для всех целевых  функций, а не для одной.

Решение задачи в среде MATLAB (Приложение 6).

Результат решения задачи:

\lstinputlisting{listings/p1.m6.log}

Результаты решения для критерия в виде таблицы:

\begin{table}[h!]
	\centering
	\bgroup
	\captionsetup{singlelinecheck = false, format= hang, justification=raggedleft, font=footnotesize, labelsep=space}
	\caption{Результирующие значения метода достижения цели}
	\def\arraystretch{1}
	\begin{tabular}{ | m{1.5cm} | m{1.9cm} | m{2.2cm} | m{2.2cm} | }
		\hline
		& result, Mill\$ & proportion, \% & difference, \% \\ \hline
		F2 max & 5038.049 & 69.405 & 30.595 \\ \hline
		F3 min & 3473.833 & 130.595 & 30.595 \\ \hline
		F4 min & 6.641 & 100.429 & 0.429 \\ \hline
		F5 max & 184.559 & 92.984 & 7.016 \\ \hline
		F6 max & 12269.075 & 74.349 & 25.651 \\ \hline
		Mean & & & \textbf{18.86} \\
		\hline
	\end{tabular}
	\egroup
\end{table}

\textbf{* Функция $F_1$ используется только для вычисления критерия $F_6$, поэтому не расчитывается отдельно.}

\subsection{Введение метрики в пространстве критериев}

Для перехода к однокритериальной задаче оптимизации методом введения метрики в пространстве целевых функций необходимо определить координаты идеальной точки $a_i=(f_1^*,f_2^*,...,f_1^N)$, где $f_i=min(f_i(x))$. Данные оптимальные значения уже известны из предыдущих пунктов работы и равняются:

\begin{equation*}
\text{$a=[7258.939, 2660, 6.613, 198.485, 16501.964]$}
\end{equation*}

Введем в пространстве критериев метрику в виде евклидова расстояния:

\begin{equation*}
\text{$p(y,a)=(\sum_{i=1}^{N}(a_i-y_i)^2)^{\frac{1}{2}}$}
\end{equation*}

Тогда за целевую функцию (обобщенный критерий), с учётом необходимости нормировки, можно взять выражение:

\begin{equation*}
\text{$f=\sum_{i=1}^{N}(\frac{a_i-f_i}{f_i^*})^2=\sum_{i=1}^{N}(1-\frac{f_i}{f_i^*})^2$}
\end{equation*}

Решение задачи в среде MATLAB (Приложение 7).

Результат решения задачи:

\lstinputlisting{listings/p1.m7.log}

Результаты решения для критерия в виде таблицы:

\begin{table}[h!]
	\centering
	\bgroup
	\captionsetup{singlelinecheck = false, format= hang, justification=raggedleft, font=footnotesize, labelsep=space}
	\caption{Результирующие значения метода введения метрики в пространстве критериев}
	\def\arraystretch{1}
	\begin{tabular}{ | m{1.5cm} | m{1.9cm} | m{2.2cm} | m{2.2cm} | }
		\hline
		& result, Mill\$ & proportion, \% & difference, \% \\ \hline
		F2 max & 5589.230 & 76.998 & 23.002 \\ \hline
		F3 min & 3684.450 & 138.513 & 38.513 \\ \hline
		F4 min & 6.650 & 100.553 & 0.553 \\ \hline
		F5 max & 188.602 & 95.021 & 4.979 \\ \hline
		F6 max & 13241.479 & 80.242 & 19.758 \\ \hline
		Mean & & & \textbf{17.361} \\
		\hline
	\end{tabular}
	\egroup
\end{table}

\textbf{* Функция $F_1$ используется только для вычисления критерия $F_6$, поэтому не расчитывается отдельно.}

\subsection{Оценка Парето-оптимальности полученных решений}

Выделим результаты решения задачи различными методами в отдельную таблицу. Метод последовательных уступок из-за нелинейности некоторых целевых функций не попадает в таблицу.

\begin{table}[h!]
	\centering
	\bgroup
	\captionsetup{singlelinecheck = false, format= hang, justification=raggedleft, font=footnotesize, labelsep=space}
	\caption{Оценка Парето-оптимальности полученных решений}
	\def\arraystretch{1}
	\begin{tabular}{ | m{2.2cm} | m{1.05cm} | m{1.05cm} | m{1.05cm} | m{1.35cm} | m{1.35cm} | m{1.35cm} | m{1.35cm} | m{1.35cm} | m{1.4cm} | }
		\hline
		Метод & $x_1, Mill$ & $x_2, Mill$ & $x_3, Mill$ & $F_2, Mill\$$ & $F_3, Mill\$$ & $F_4, Mill\$$ & $F_5, Mill\$$ & $F_6, Mill\$$ & Средняя разница \% \\ \hline
		Аддитивная свертка & 35 & 30 & 1 & 7218.316 & 4270.000 & 6.672 & 198.485 & 16041.644 & \textbf{12.95} \\ \hline
		Мультиплика- тивная свертка & 35 & 30 & 1 & 7218.315 & 4270.004 & 6.672 & 198.485 & 16041.646 & \textbf{12.95} \\ \hline
		Минимакс & 35 & 23.294 & 1 & 5364.414 & 3599.419 & 6.646 & 187.004 & 12846.607 & \textbf{17.97} \\ \hline
		Метод последовательных уступок & - &- & - & - & - & - & - & - & \textbf{-} \\ \hline
		Метод достижения цели & 35 & 22.038 & 1 & 5038.049 & 3473.833 & 6.641 & 184.559 & 12269.075 & \textbf{18.86} \\ \hline
		Введение метрики в пространстве критериев & 35 & 24.145 & 1 & 5589.230 & 3684.450 & 6.650 & 188.602 & 13241.479 & \textbf{17.361} \\
		\hline
	\end{tabular}
	\egroup
\end{table}

\textbf{* Функция $F_1$ используется только для вычисления критерия $F_6$, поэтому не расчитывается отдельно.}

Парето-оптимальными по соотношению доходов к расходам можно назвать только аддитивную и мультипликативную светку, так как при их использовании получается наибольший оборот ($F_6$), а оборот как раз таки и характеризует разницу между доходами и расходами. Для этих методов наибольший оборот получается благодаря введению коэффициентов значимости.

\subsection{Решение задачи стохастического программирования}

Рассмотрим задачу стохастического программирования на основе задачи однокритериальной оптимизации, которая была получена из исходной методом введения метрики в пространстве критериев.

Преобразуем последнее ограничение системы:

\begin{equation*}
\text{$x_1+x_2+x_3\leq 80$}
\end{equation*}

в вероятностное, тогда:

\begin{equation*}
\text{$P\{\alpha_1x_1+\alpha_2x_2+\alpha_3x_3\leq 80\}\geq \alpha$}
\end{equation*}

где все $a_i$ нормально распределены и имеют следующие математические ожидания и дисперсии:

\begin{equation*}
\text{$M[\alpha_1]=1, M[\alpha_2]=1, M[\alpha_3]=1$}
\end{equation*}

\begin{equation*}
\text{$\sigma[\alpha_1]=0.5, \sigma[\alpha_2]=0.5, \sigma[\alpha_3]=0.5$}
\end{equation*}

где СКО равняется половине математического ожидания. По таблице функции нормального распределения находим коэффициенты $K_\alpha$:

\begin{equation*}
\text{$K_{0.5}=0, K_{0.6}=0.2533, K_{0.7}=0.5244, K_{0.8}=0.8416, K_{0.9}=1.2816$}
\end{equation*}

Таким образом, вероятностное ограничение становится эквивалентно детерминированному неравенству:

\begin{equation*}
\text{$x_1+x_2+x_3+K_\alpha \sqrt{0.5x_1^2+0.5x_2^2+0.5x_3^2}\leq 80$}
\end{equation*}

Решение задачи в среде MATLAB (Приложение 8).

Результат решения задачи:

\lstinputlisting{listings/p1.m8.log}

Результаты решения для критерия в виде таблицы:

\begin{table}[h!]
	\centering
	\bgroup
	\captionsetup{singlelinecheck = false, format= hang, justification=raggedleft, font=footnotesize, labelsep=space}
	\caption{Решение задачи стохастического программирования}
	\def\arraystretch{1}
	\begin{tabular}{ | m{0.7cm} | m{1.0cm} | m{1.05cm} | m{1.05cm} | m{1.05cm} | m{1.35cm} | m{1.35cm} | m{1.35cm} | m{1.35cm} | m{1.35cm} | m{1.4cm} | }
		\hline
		$\alpha$ & $K_\alpha$ & $x_1, Mill$ & $x_2, Mill$ & $x_3, Mill$ & $F_2, Mill\$$ & $F_3, Mill\$$ & $F_4, Mill\$$ & $F_5, Mill\$$ & $F_6, Mill\$$ & Средняя разница \% \\ \hline
		det & det & 35 & 24.145 & 1 & 5589.230 & 3684.450 & 6.650 & 188.602 & 13241.479 & \textbf{17.361} \\ \hline
		0.1 & -1.282 & 35 & 24.145 & 1 & 5589.230 & 3684.450 & 6.650 & 188.602 & 13241.479 & \textbf{17.361} \\ \hline
		0.2 & -0.842 & 35 & 24.145 & 1 & 5589.230 & 3684.450 & 6.650 & 188.602 & 13241.479 & \textbf{17.361} \\ \hline
		0.3 & -0.524 & 35 & 24.145 & 1 & 5589.230 & 3684.450 & 6.650 & 188.602 & 13241.479 & \textbf{17.361} \\ \hline
		0.4 & -0.253 & 35 & 24.145 & 1 & 5589.230 & 3684.450 & 6.650 & 188.602 & 13241.479 & \textbf{17.361} \\ \hline
		0.5 & 0 & 35 & 24.145 & 1 & 5589.230 & 3684.450 & 6.650 & 188.602 & 13241.479 & \textbf{17.361} \\ \hline
		0.6 & 0.253 & 35 & 24.145 & 1 & 5589.230 & 3684.450 & 6.650 & 188.602 & 13241.479 & \textbf{17.361} \\ \hline
		0.7 & 0.524 & 35 & 24.145 & 1 & 5589.230 & 3684.450 & 6.650 & 188.602 & 13241.479 & \textbf{17.361} \\ \hline
		0.8 & 0.842 & 35 & 20.002 & 1 & 4523.611 & 3270.208 & 6.633 & 180.361 & 11347.389 & \textbf{20.259} \\ \hline
		0.9 & 1.282 & 34.155 & 14.355 & 0 & 1781.719 & 2460.145 & 6.588 & 131.020 & 5670.774 & \textbf{36.595} \\
		\hline
	\end{tabular}
	\egroup
\end{table}

\textbf{* Функция $F_1$ используется только для вычисления критерия $F_6$, поэтому не расчитывается отдельно.}

Увеличение доверительной вероятности $\alpha$ приводит к ухудшению результатов решения. Неизменность результатов на промежутке $\alpha=[0.1, 0.7]$ объясняется выбором ограничения.

\section{Вывод}

Можно заметить, что аддитивная и мультипликативная свертка выдают одинаковый, наиболее оптимальный результат. Наилучший результат этих методов обусловлен наличием коэффициентов значимости.

Методы, не подразумевающие введение весовых коэффициентов показывают похожий результат, который в целом хуже, чем у аддитивной и мультипликативной свертки.

\chapter{Поиск оптимальной стратегии принятия решений с использованием марковских моделей}
\section{Постановка задачи}
\textbf{Вариант:} 12, решить задачу методом итераций по стратегиям для $N=\infty$\\\\
%\textbf{Примечание:} задание из книги Г. Вагнера «Основы исследования операций», т. 3, стр. 183 – 184
Крупная фирма, производящая моющие средства и пользующаяся широкой известностью в связи с успехами в исследованиях по созданию новых продуктов и их рекламированию, выпустила на рынок новый высококачественный стиральный порошок, названный LYE. Руководитель, возглавляющий производство этого продукта, совместно с отделом рекламы разрабатывает специальную рекламную кампанию по сбыту порошка, для которой принят девиз «Порошок LYE нужен всем!» Как и все продукты фирмы, новый продукт в течение первого полугодия будет иметь высокий уровень сбыта. Руководитель полагает, что с вероятностью 0,8 этот уровень сбыта сохранится и в последующем полугодии при условии проведения особой рекламной кампании и что эта вероятность составит всего 0,5, если такую кампанию не проводить. В случае, если уровень сбыта снизится до среднего, у руководителя имеются две возможности. Он может дать указание о проведении исследований с целью улучшения качества продукта. При этом условии с вероятностью 0,7 уровень сбыта к началу следующего полугодия повысится до первоначального высокого значения. С другой стороны, можно ничего не предпринимать в отношении улучшения качества продукта. Тогда с вероятностью 0,6 в начале последующего полугодия уровень сбыта останется средним, однако вследствие изменений потребительских вкусов он может вновь подняться до высокого значения лишь с вероятностью 0,4.

Если сбыт нового стирального порошка начинается на высоком уровне при обычной рекламе, то прибыли в течение полугодия равны 19 единицам в случае, когда этот уровень сохраняется, и равны 13, если уровень сбыта падает. При проведении специальной рекламной кампании соответствующие показатели равны 4,5 и 2 единицам. Если начальный уровень сбыта окажется средним и при этом проводятся исследования с целью улучшения качества продукции, то прибыли составят 11 единиц в случае, когда уровень сбыта поднимается до высокого, и 9 единиц в противном случае. При сохранении продукта в неизменном виде соответствующие прибыли равны 13 и 3 единицам. Предположим, что одна и та же проблема принятия решений относительно сбыта стирального порошка LYE повторяется через каждые полгода в течение бесконечного планового периода.


\section{Метод итераций по стратегиям}
Для начала выпишем все известные параметры задачи. 

Система может быть в двух состояних:
\begin{enumerate}
\item хороший сбыт($S_1$);
\item средний сбыт($S_2$).
\end{enumerate}  
Организация может предпринять следующие действия(далее стратегии):
\begin{enumerate}
\item всегда улучшать сбыт($X_1$)
\begin{itemize}
\item при $S_1$ - создание специальной рекламы($D_1$);
\item при $S_2$ - проведение исследований($D_2$). 
\end{itemize}
\item улучшать сбыт только при хорошем сбыте($X_2$)
\begin{itemize}
\item при $S_1$ - создание специальной рекламы($D_1$);
\item при $S_2$ - ничего не делать($D_3$). 
\end{itemize}
\item улучшать сбыт только при среднем сбыте($X_3$)
\begin{itemize}
\item при $S_1$ - ничего не делать($D_3$);
\item при $S_2$ - проведение исследований($D_2$). 
\end{itemize}
\item всегда ничего не делать($X_4$)
\begin{itemize}
\item при $S_1$ - ничего не делать($D_3$);
\item при $S_2$ - ничего не делать($D_3$). 
\end{itemize}
\end{enumerate}

На основе данной информации составим матрицы переходных вероятностей $P_1, P_2, P_3, P_4$ соответсвующие стратегиям $X_1, X_2, X_3, X_4$.

\begin{equation*}
P_1=\begin{pmatrix}
0.8 & 0.2\\
0.7 & 0.3
\end{pmatrix}\\
P_2=\begin{pmatrix}
0.8 & 0.2\\
0.4 & 0.6
\end{pmatrix}\\
P_3=\begin{pmatrix}
0.5 & 0.5\\
0.7 & 0.3
\end{pmatrix}\\
P_4=\begin{pmatrix}
0.5 & 0.5\\
0.4 & 0.6
\end{pmatrix}
\end{equation*}
%Строки означают попытки улучшить сбыт. Первая строка - специальная реклама, а вторая - улучшение качества продукта.
%
%Столбцы означают состояние сбытая. Первый столбец - хороший сбыт, второй - средний сбыт.
%
%То есть, если провести специальную рекламу, система с вероятностью в 0.8 останется в состоянии хорошого сбыта, и с верояностью в 0.2 перейдет в состояние среднего сбыта. По аналогии при улучшении качества продукта.
%
%Составим матрицу переходных вероятностей, если не предпринимать никаких действий.

%Если сравнивать $P_1$ и $P_2$, то у $P_1$ вероятность того что система будет в состоянии хорошего сбыта, выше чем у $P_2$.

Также составим матрицы доходов $R_1, R_2, R_3, R_4$.
\begin{equation*}
R_1=\begin{pmatrix}
4.5 & 2\\
11 & 9
\end{pmatrix}\\
R_2=\begin{pmatrix}
4.5 & 2\\
13 & 3
\end{pmatrix}\\
R_3=\begin{pmatrix}
19 & 13\\
11 & 9
\end{pmatrix}\\
R_4=\begin{pmatrix}
19 & 13\\
13 & 3
\end{pmatrix}
\end{equation*}
Множество допустимых стратегий $G=\{X_1, X_2, X_3, X_4\}$.
\subsection{Этап(1) оценивания параметров}
Выбираем стратегию $\tau$ - $X_4$.  Тогда, матрицы переходных вероятностей и доходов будут следующими:
\begin{equation*}
P=\begin{pmatrix}
0.5 & 0.5\\
0.4 & 0.6
\end{pmatrix}\\
R=\begin{pmatrix}
19 & 13\\
13 & 3
\end{pmatrix}
\end{equation*}
Учитывая, что  $F_{\tau}(2)=0$, получаем систему линейных алгебраических уравнений:

\begin{equation*}
\begin{cases}
E_{\tau}-(1-0.5)*F_{\tau}(1)=16\\
E_{\tau}-(1-0.4)*F_{\tau}(1)=7
\end{cases}
\end{equation*}
\begin{lstlisting}[language={matlab}, caption={Скрпит для решения системы уравнений},basicstyle=\ttfamily]
syms Et Ft
eqn1 = Et - (1-0.5)*Ft == 16;
eqn2 = Et - (1-0.4)*Ft == 7;
[A, B] = equationsToMatrix([eqn1, eqn2], [Et, Ft])
X = linsolve(A, B)
\end{lstlisting}
В результате выполнения скрипта matlab, было получено единственное решение:
\begin{equation*}
E_{\tau}=61; F_{\tau}(1)=90
\end{equation*}

\subsection{Этап(1) улучшения стратегии}
Для каждого состояния $S_j$, где $j$ от $1$ до $m$, найдем допустимое решение, на котором достигается:
\begin{equation*}
max(v_j(X_i)+\sum_{k=1}^mp_{jk}(X_i)F_{\tau}(k)
\end{equation*}


\tabulinesep = 1mm
\begin{longtabu} to \textwidth {|X[1, c , m ] |X[3, c , m ] | X[3,c , m ]|X[3,c , m ]|X[3,c , m ]| X[2,c , m ]|X[1,c , m ]|}\firsthline\hline

\multirow{2}{*}{$S_j$} & \multicolumn{4}{c|}{$\varphi_i=v_j(X_i)+p_{j1}(X_i)F_i(1)$} & \multirow{2}{*}{$max\varphi_i$} & \multirow{2}{*}{$X_{j}$}\\ \cline{2-5}
&i=1&i=2&i=3&i=4&&\\ \hline
1&4+0.8*(90)= 76&4+0.8*(90)= 76&16+0.5*(90)= 61&16+0.5*(90)= 61&76&$X_1$, $X_2$\\ \hline
2&10.4+0.7*(90)= 73.4&7+0.4*(90)= 40&10.4+0.7*(90)= 73.4&7+0.4*(90)= 40&73.4&$X_1$, $X_3$\\ \hline
\end{longtabu}


\subsection{Этап(2) оценивания параметров}
Выбираем стратегию $\tau$ - $X_3$.  Тогда, матрицы переходных вероятностей и доходов будут следующими:
\begin{equation*}
P=\begin{pmatrix}
0.5 & 0.5\\
0.7 & 0.3
\end{pmatrix}\\
R=\begin{pmatrix}
19 & 13\\
11 & 9
\end{pmatrix}
\end{equation*}
Учитывая, что  $F_{\tau}(2)=0$, получаем систему линейных алгебраических уравнений:

\begin{equation*}
\begin{cases}
E_{\tau}-(1-0.5)*F_{\tau}(1)=16\\
E_{\tau}-(1-0.7)*F_{\tau}(1)=10.4
\end{cases}
\end{equation*}
\begin{lstlisting}[language={matlab}, caption={Скрпит для решения системы уравнений},basicstyle=\ttfamily]
syms Et Ft
eqn1 = Et - (1-0.5)*Ft == 16;
eqn2 = Et - (1-0.7)*Ft == 10.4;
[A, B] = equationsToMatrix([eqn1, eqn2], [Et, Ft])
X = linsolve(A, B)
\end{lstlisting}
В результате выполнения скрипта matlab, было получено единственное решение:
\begin{equation*}
E_{\tau}=2; F_{\tau}(1)=-28
\end{equation*}

\subsection{Этап(2) улучшения стратегии}
Для каждого состояния $S_j$, где $j$ от $1$ до $m$, найдем допустимое решение, на котором достигается:
\begin{equation*}
max(v_j(X_i)+\sum_{k=1}^mp_{jk}(X_i)F_{\tau}(k)
\end{equation*}


\tabulinesep = 1mm
\begin{longtabu} to \textwidth {|X[1, c , m ] |X[3, c , m ] | X[3,c , m ]|X[3,c , m ]|X[3,c , m ]| X[2,c , m ]|X[1,c , m ]|}\firsthline\hline

\multirow{2}{*}{$S_j$} & \multicolumn{4}{c|}{$\varphi_i=v_j(X_i)+p_{j1}(X_i)F_i(1)$} & \multirow{2}{*}{$max\varphi_i$} & \multirow{2}{*}{$X_{j}$}\\ \cline{2-5}
&i=1&i=2&i=3&i=4&&\\ \hline
1&4+0.8*(-28)= -18.4&4+0.8*(-28)= -18.4&16+0.5*(-28)= 2&16+0.5*(-28)= 2&2&$X_3$, $X_4$\\ \hline
2&10.4+0.7*(-28)= -9.2&7+0.4*(-28)= -4.2&10.4+0.7*(-28)= -9.2&7+0.4*(-28)= -4.2&-4.2&$X_2$, $X_4$\\ \hline
\end{longtabu}

Так как $t\neq\tau$, то снова переходим к этапу оценивания параметров.


\subsection{Этап(3) оценивания параметров}
Выбираем стратегию $\tau$ - $X_2$.  Тогда, матрицы переходных вероятностей и доходов будут следующими:
\begin{equation*}
P=\begin{pmatrix}
0.8 & 0.2\\
0.4 & 0.6
\end{pmatrix}\\
R=\begin{pmatrix}
4.5 & 2\\
13 & 3
\end{pmatrix}
\end{equation*}
Учитывая, что  $F_{\tau}(2)=0$, получаем систему линейных алгебраических уравнений:

\begin{equation*}
\begin{cases}
E_{\tau}-(1-0.8)*F_{\tau}(1)=4\\
E_{\tau}-(1-0.4)*F_{\tau}(1)=7
\end{cases}
\end{equation*}
\begin{lstlisting}[language={matlab}, caption={Скрпит для решения системы уравнений}, basicstyle=\ttfamily]
syms Et Ft
eqn1 = Et - (1-0.8)*Ft == 4;
eqn2 = Et - (1-0.4)*Ft == 7;
[A, B] = equationsToMatrix([eqn1, eqn2], [Et, Ft])
X = linsolve(A, B)
\end{lstlisting}
В результате выполнения скрипта matlab, было получено единственное решение:
\begin{equation*}
E_{\tau}=5/2; F_{\tau}(1)=-15/2
\end{equation*}


\subsection{Этап(3) улучшения стратегии}
Для каждого состояния $S_j$, где $j$ от $1$ до $m$, найдем допустимое решение, на котором достигается:
\begin{equation*}
max(v_j(X_i)+\sum_{k=1}^mp_{jk}(X_i)F_{\tau}(k)
\end{equation*}


\tabulinesep = 1mm
\begin{longtabu} to \textwidth {|X[1, c , m ] |X[3, c , m ] | X[3,c , m ]|X[3,c , m ]|X[3,c , m ]| X[2,c , m ]|X[1,c , m ]|}\firsthline\hline

\multirow{2}{*}{$S_j$} & \multicolumn{4}{c|}{$\varphi_i=v_j(X_i)+p_{j1}(X_i)F_i(1)$} & \multirow{2}{*}{$max\varphi_i$} & \multirow{2}{*}{$X_{j}$}\\ \cline{2-5}
&i=1&i=2&i=3&i=4&&\\ \hline
1&4+0.8*(-15/2)= -2&4+0.8*(-15/2)= -2&16+0.5*(-15/2)= 12.25&16+0.5*(-15/2)= 12.25&12.25&$X_3$, $X_4$\\ \hline
2&10.4+0.7*(-15/2)= 5.15&7+0.4*(-15/2)= 4&10.4+0.7*(-15/2)= 5.15&7+0.4*(-15/2)= 4&5.15&$X_1$, $X_3$\\ \hline
\end{longtabu}

Так как $t\neq\tau$, то снова переходим к этапу оценивания параметров.


\subsection{Этап(4) оценивания параметров}
Выбираем стратегию $\tau$ - $X_1$.  Тогда, матрицы переходных вероятностей и доходов будут следующими:
\begin{equation*}
P=\begin{pmatrix}
0.8 & 0.2\\
0.7 & 0.3
\end{pmatrix}\\
R=\begin{pmatrix}
4.5 & 2\\
11 & 9
\end{pmatrix}
\end{equation*}
Учитывая, что  $F_{\tau}(2)=0$, получаем систему линейных алгебраических уравнений:

\begin{equation*}
\begin{cases}
E_{\tau}-(1-0.8)*F_{\tau}(1)=4\\
E_{\tau}-(1-0.7)*F_{\tau}(1)=10.4
\end{cases}
\end{equation*}
\begin{lstlisting}[language={matlab}, caption={Скрпит для решения системы уравнений}, basicstyle=\ttfamily]
syms Et Ft
eqn1 = Et - (1-0.8)*Ft == 4;
eqn2 = Et - (1-0.7)*Ft == 10.4;
[A, B] = equationsToMatrix([eqn1, eqn2], [Et, Ft])
X = linsolve(A, B)
\end{lstlisting}
В результате выполнения скрипта matlab, было получено единственное решение:
\begin{equation*}
E_{\tau}=-44/5; F_{\tau}(1)=-64
\end{equation*}

\subsection{Этап(4) улучшения стратегии}
Для каждого состояния $S_j$, где $j$ от $1$ до $m$, найдем допустимое решение, на котором достигается:
\begin{equation*}
max(v_j(X_i)+\sum_{k=1}^mp_{jk}(X_i)F_{\tau}(k)
\end{equation*}


\tabulinesep = 1mm
\begin{longtabu} to \textwidth {|X[1, c , m ] |X[3, c , m ] | X[3,c , m ]|X[3,c , m ]|X[3,c , m ]| X[2,c , m ]|X[1,c , m ]|}\firsthline\hline

\multirow{2}{*}{$S_j$} & \multicolumn{4}{c|}{$\varphi_i=v_j(X_i)+p_{j1}(X_i)F_i(1)$} & \multirow{2}{*}{$max\varphi_i$} & \multirow{2}{*}{$X_{j}$}\\ \cline{2-5}
&i=1&i=2&i=3&i=4&&\\ \hline
1&4+0.8*(-64)= -47.2&4+0.8*(-64)= -47.2&16+0.5*(-64)= -16&16+0.5*(-64)= -16&-16&$X_3$, $X_4$\\ \hline
2&10.4+0.7*(-64)= -34.4&7+0.4*(-64)= -18.6&10.4+0.7*(-64)= -34.4&7+0.4*(-64)= -18.6&-18.6&$X_2$, $X_4$\\ \hline
\end{longtabu}

Итак в этапе(2) и данном, были найдены оптимальные стратегии. То есть

\begin{equation*}
\tau=((X_3 \textit{ или } X_4), (X_2 \textit{ или } X_4))^T
\end{equation*}
В каждом из двух состояний, имеется два варианта дальнейших действий.

Если подвести итоги, то данное решение означает что:
\begin{itemize}
\item В состоянии хорошего сбыта($S_1$) - ничего не делать($D_3$);
\item В состоянии среднего сбыта($S_2$) - ничего не делать($D_3$).
\end{itemize}
И судя по данным итогам, ничего не делать($D_3$) является лучшей стратегией. Подобный исход можно объяснить тем, что при попытках увеличения сбыта, компания тратит на это деньги и соответственно доход снижается.



\chapter{Поиск оптимальных параметров сети систем массового обслуживания}
\section{Постановка задачи}
\textbf{Вариант:} задача 4, вариант 144.

\tabulinesep = 1mm
\begin{longtabu} to \textwidth {|X[c , m ] |X[4,c , m ] | X[c , m ]|X[c , m ]|X[c , m ]| X[c , m ]|X[4,c , m ]|}\firsthline\hline

№ вар&$Q=\{q_{ij}\}_{\begin{matrix}i=\overline{0,n}\\j=\overline{0,n}\end{matrix}}$&$ca_0$&$\lambda_0$&$L_r$&$\mu$&$\{cs_j\}$\\ \hline
144&$\begin{array}{c|c|c|c|c}0& 0.2& 0.3& 0.2& 0.3\\ \hline 0.1&0&0.2&0.6&0.1\\ \hline 0.6&0.2&0&0.1&0.1\\ \hline 0&0.5&0.1&0&0.4\\ \hline 0.5&0.3&0.1&0.1&0 \end{array}$&0.16&8&-&10&$\begin{array}{c|c|c|c}0.04& 0.04& 0.04& 0.04	\end{array}$\\ \hline
\end{longtabu}

Найти:
\begin{equation*}
min L(\mu)=\sum_{j=1}^{n}L_j
\end{equation*}
При условии:
\begin{equation*}
\sum_{j=1}^{n}\mu_j=\mu
\end{equation*}

\section{Решение}
Вычислим мощность $\mu > \mu_j^1, cs$ и $ca$ для каждой станции.

Скорость прихода задач в узел j: $\lambda_j=\lambda_{0j}+\sum_{i=0}^nq_{ij}\lambda_i, j=0, ..., n$
\begin{equation*}
Q =
 \begin{pmatrix}
  0& 0.2& 0.3& 0.2& 0.3 \\
  0.1&0&0.2&0.6&0.1 \\
  0.6&0.2&0&0.1&0.1  \\
  0&0.5&0.1&0&0.4 \\
  0.5&0.3&0.1&0.1&0
 \end{pmatrix}
\end{equation*}
$\lambda_0=8$\\
$\lambda_1=0.2\lambda_0+0\lambda_1+0.2\lambda_2+0.5\lambda_3+0.3\lambda_4$\\
$\lambda_2=0.3\lambda_0+0.2\lambda_1+0\lambda_2+0.1\lambda_3+0.1\lambda_4$\\
$\lambda_3=0.2\lambda_0+0.6\lambda_1+0.1\lambda_2+0\lambda_3+0.1\lambda_4$\\
$\lambda_4=0.3\lambda_0+0.1\lambda_1+0.1\lambda_2+0.4\lambda_3+0\lambda_4$
\begin{lstlisting}[language={matlab}, caption={Код Matlab}, basicstyle=\ttfamily]
A = [1 0 0 0 0;
 0.2 -1 0.2 0.5 0.3;
 0.3 0.2 -1 0.1 0.1;
 0.2 0.6 0.1 -1 0.1;
 0.3 0.1 0.1 0.4 -1];
b=[8; 0; 0; 0; 0];
lambdaj=A\b
\end{lstlisting}
\begin{lstlisting}[language={matlab}, caption={Результат}, basicstyle=\ttfamily]
lambdaj =
    8.0000
    9.1128
    5.7833
    8.3697
    7.2375
\end{lstlisting}
Проверим полученный результат:
\begin{lstlisting}[language={matlab}, caption={Проверка}, basicstyle=\ttfamily]
>> [ 0 0.1 0.6 0 0.5] * lambdaj

ans =
    8.0000 
\end{lstlisting}
Как и ожидалось, была получена $\lambda_0$.\\Вычислим $ca_j$. Для этого, сперва найдем все $\lambda_{ij}$, где $\lambda_{ij} = \lambda_i*q_{ij}$.
\begin{lstlisting}[language={matlab}, caption={Код Matlab}, basicstyle=\ttfamily]
N=5;
Q = [0 0.2 0.3 0.2 0.3;
 0.1 0 0.2 0.6 0.1;
 0.6 0.2 0 0.1 0.1;
 0 0.5 0.1 0 0.4;
 0.5 0.3 0.1 0.1 0];
lambdaij=[];
for i = 1:N
 for j = 1:N
 lambdaij(i,j) = lambdaj(i)*Q(i, j);
 end
end
 lambdaij
\end{lstlisting}
\begin{lstlisting}[language={matlab}, caption={Результат}, basicstyle=\ttfamily]
lambdaij =

         0    1.6000    2.4000    1.6000    2.4000
    0.9113         0    1.8226    5.4677    0.9113
    3.4700    1.1567         0    0.5783    0.5783
         0    4.1849    0.8370         0    3.3479
    3.6188    2.1713    0.7238    0.7238         0
\end{lstlisting}

\begin{equation*}
\lambda =
 \begin{pmatrix}
  0  &  1.6   & 2.4   & 1.6    &2.4 \\
  0.91&    0   & 1.82  &  5.47  &  0.91 \\
  3.47 &   1.16 &   0   & 0.58   & 0.58  \\
  0    &4.18    &0.84    &0       &3.35 \\
  3.62  &  2.17  &  0.72  &  0.72  &  0
 \end{pmatrix}
\end{equation*}
Решим уравнения по формулам:
\begin{equation*}
ca_j=\frac{\lambda_{0j}}{\lambda_j}ca_{0j}+\sum_{i=1}^{n}\frac{\lambda_{ij}}{\lambda_j}ca_{ij}=\sum_{i=0}^{n}\frac{\lambda_{ij}}{\lambda_j}ca_{ij}
\end{equation*}

\begin{equation*}
cd_{ij}=q_{ij}cd_{ij}+1-q_{ij}
\end{equation*}

\begin{lstlisting}[language={matlab}, caption={Код Matlab}, basicstyle=\ttfamily]
caA = 0;
 caB = [0 0 0 0 0];
for j = 1:N
 for i = 1:N
 caA(j,i) = lambdaij(i,j)/lambdaj(j)*Q(i, j);
 caB(j)=caB(j)+lambdaij(i,j)/lambdaj(j)*(1-Q(i,j));
 end
end
 caA = caA-eye(5);
 caA(1,:) = [1 0 0 0 0];
 caA;
 caA^-1;
 caB = -caB';
 caB(1)=0.49;
 caj = (caA^-1)*caB
\end{lstlisting}
\begin{lstlisting}[language={matlab}, caption={Результат}, basicstyle=\ttfamily]
caj =
    0.1600
    0.9486
    0.8902
    0.9461
    0.9049
\end{lstlisting}
Вычислим $L_j$ и $P_j$
\begin{equation*}
p_j=\frac{\lambda_j}{\mu_jm_j}
\end{equation*}
\begin{equation*}
L_j(\lambda_j, ca_j, \mu_j, cs_j)=\frac{(\frac{\lambda_j}{\mu_j})^2(ca_j+cs_j)*g(\lambda_j, ca_j, \mu_j, cs_j)}{2(1-\frac{\lambda_j}{\mu_j})}+\frac{\lambda_j}{\mu_j}
\end{equation*}
\begin{equation*}
PI_j(\mu_j)=-V_j\frac{\partial L_j(\mu_j)}{\partial(\mu_j)}
\end{equation*}
Где:
\begin{equation*}
\lambda_j=9.1128, ca_0=0.16, cs_1=0.04
\end{equation*}
Для этого был написан скрипт matlab.
\begin{lstlisting}[language={matlab}, caption={Код Matlab}, basicstyle=\ttfamily]
for i = 2:N
 [Lj(i-1), Pj(i-1)] = params(lambdaj(i), caj(i), m(i-1), cs(i-1));
end
 Lj
 Pj
 L = sum(Lj)

function [ fLj, fPj ] = params( fl, fca, fm, fcs )
 Lj = '(l/m)^2*(ca+csj)*exp(-2*(1-l/m)*(1-ca)^2/(3*(l/m)*(ca+csj)))/(2*(1-l/m))';
 syms m;
 syms l;
 syms ca;
 syms csj;
 fLj = subs(Lj,l, fl);
 fLj = subs(fLj,m, fm);
 fLj = subs(fLj,ca, fca);
 fLj = subs(fLj,csj, fcs);
 fLj = vpa(fLj);
 
 Pj = '-((l)/(l-m)^2)';
 fPj = subs(Pj,l, fl);
 fPj = subs(fPj,m, fm);
 fPj = -1*vpa(fPj);
 end
\end{lstlisting}
\begin{lstlisting}[language={matlab}, caption={Результат}, basicstyle=\ttfamily]
Lj =
[ 4.6255839976370720273840674915939, 0.36659806895060537891974885952995, 2.1179331569743248761799734854413, 0.89370855816402942385738205100272]
 
 
Pj =
[ 11.576845733984487216317150244183, 0.32525589689516798523377343071108, 3.1492188330322032096928472738867, 0.94838805064508015837789082867662] 

L = 
8.0038237817260317063411718875679
\end{lstlisting}
Воспользуемся следующей формулой:
\begin{equation*}
PI_j(\mu_j,(\lambda_j+\varepsilon_j))=max\{PI_j(\mu_j), j\in J_0\}
\end{equation*}
Чем выше загрузка узла $J$, тем больше $PI_j=-\frac{\partial L(\mu_j)}{\partial\mu_j}$\\
Для узла $\textbf{M/M/1}$ имеем $PI_j=-\frac{\partial L(\mu_j)}{\partial\mu_j}=\frac{\lambda_j}{(\lambda_j-\mu_j)^2}$.
Благодаря расчётам по этой формуле, можно понять в каких узла нужно увеличивать или уменьшать интенсивность входного потока.\\
Далее, применим следующий алгоритм:
\begin{figure}[H]
  \centering
  \includegraphics[width=.8\textwidth]{img/alg}
\end{figure}
Как значение $\Delta$ возьмем 0.5.\\По приведенному выше алгоритму определяем множества $J_0, J_1, J_2$. Результаты приведены в таблице \ref{tbl_1}.


\tabulinesep = 1mm
\begin{longtabu} to \textwidth {|X[ c , m ] |X[c , m ] | X[ c , m ]|X[ c , m ]|X[ c , m ]|}\firsthline\hline
№&$J_0$&$J_1$&$J_2$&Действия\\ \hline 
1&1,2,3,4&-&-&$\begin{array}{c} J_1 \leftarrow 1 \\ J_2 \leftarrow 2 \end{array}$ \\ \hline
2&1,2,3,4&1&2&$\begin{array}{c} J_1 \leftarrow 1 \\ J_2 \leftarrow 2 \end{array}$ \\ \hline
3&1,2,3,4&1&2&$\begin{array}{c} J_1 \leftarrow 3 \\ J_2 \leftarrow 2 \end{array}$ \\ \hline
4&1,2,3,4&1,3&2&$\begin{array}{c} J_1 \leftarrow 1 \\ J_2 \leftarrow 2 \end{array}$ \\ \hline
5&1,2,3,4&1,3&2&$\begin{array}{c} J_1 \leftarrow 3 \\ J_2 \leftarrow 4 \end{array}$ \\ \hline
6&1,2,3,4&1,3&2,4&$\begin{array}{c} J_1 \leftarrow 1 \\ J_2 \leftarrow 2 \end{array}$ \\ \hline
7&1,2,3,4&1,3&2,4&$\begin{array}{c} J_0 \leftarrow J_0-1\\ J_0 \leftarrow J_0-2\\ J_1 \leftarrow 2\\ J_2 \leftarrow 1 \end{array}$\\ \hline
8&3,4&1,2,3&1,2,4&$\begin{array}{c} J_1 \leftarrow 4 \\ J_2 \leftarrow 3 \end{array}$ \\ \hline
9&3,4&1,2,3,4&1,2,3,4&$\begin{array}{c} J_0 \leftarrow J_0-3\\ J_0 \leftarrow J_0-4\\ J_1 \leftarrow 3\\ J_2 \leftarrow 4 \end{array}$\\ \hline
10&-&1,2,3,4&1,2,3,4&- \\ \hline
\caption{Формирование множеств J}
\label{tbl_1}
\end{longtabu}
Результаты перераспределения мощностей представлено в таблице \ref{tbl_2}.

\tabulinesep = 1mm
\begin{longtabu} to \textwidth {|X[ c , m ] |X[c , m ] | X[ c , m ]|X[ c , m ]|X[ c , m ]| X[ c , m ]|}\firsthline\hline
№&$\mu_1$&$\mu_2$&$\mu_3$&$\mu_4$&Действия\\ \hline 
1&10&10&10&10&$\begin{array}{c} \mu_2-\Delta \\ \mu_1+\Delta  \end{array}$ \\ \hline
2&10.5&9.5&10&10&$\begin{array}{c} \mu_2-\Delta \\ \mu_1+\Delta  \end{array}$ \\ \hline
3&11&9&10&10&$\begin{array}{c} \mu_2-\Delta \\ \mu_3+\Delta  \end{array}$ \\ \hline
4&11&8.5&10.5&10&$\begin{array}{c} \mu_2-\Delta \\ \mu_1+\Delta  \end{array}$ \\ \hline
5&11.5&8&10.5&10&$\begin{array}{c} \mu_4-\Delta \\ \mu_3+\Delta  \end{array}$ \\ \hline
6&11.5&8&11&9.5&$\begin{array}{c} \mu_2-\Delta \\ \mu_1+\Delta  \end{array}$ \\ \hline
7&12&7.5&11&9.5&$\begin{array}{c} \mu_1-\Delta \\ \mu_2+\Delta  \end{array}$ \\ \hline
8&11.5&8&11&9.5&$\begin{array}{c} \mu_3-\Delta \\ \mu_4+\Delta  \end{array}$ \\ \hline
9&11.5&8&10.5&10&$\begin{array}{c} \mu_4-\Delta \\ \mu_3+\Delta  \end{array}$ \\ \hline
10&11.5&8&11&9.5&- \\ \hline
\caption{Перераспределение мощностей}
\label{tbl_2}
\end{longtabu}

Пересчитаем значения $PI_j, L_j$. Результаты вычисления $PI_j$ приведены в таблице \ref{tbl_3} и \ref{tbl_4}.

\tabulinesep = 1mm
\begin{longtabu} to \textwidth {|X[ c , m ] |X[c , m ] | X[ c , m ]|X[ c , m ]|X[ c , m ]|}\firsthline\hline
№&$PI_1$&$PI_2$&$PI_3$&$PI_4$\\ \hline 
1&	11.5768&0.3252&	3.1492&	0.9483\\ \hline 
2&	4.7354&	0.4186&	3.1492&	0.9483\\ \hline 
3&	2.5586&	0.5589&	3.1492&	0.9483\\ \hline 
4&	2.5586&	0.7835&	1.8443&	0.9483\\ \hline 
5&	1.5990&	1.1769&	1.8443&	0.9483\\ \hline 
6&	1.5990&	1.1769&	1.2098&	1.4138\\ \hline 
7&	1.0931&	1.9623&	1.2098&	1.4138\\ \hline 
8&	1.5990&	1.1769&	1.2098&	1.4138\\ \hline 
9&	1.5990&	1.1769&	1.8443&	0.9483\\ \hline 
10&	1.5990&	1.1769&	1.2098&	1.4138\\ \hline 

\caption{Пересчитанное $PI_j$}
\label{tbl_3}
\end{longtabu}


\tabulinesep = 1mm
\begin{longtabu} to \textwidth {|X[ c , m ] |X[c , m ] | X[ c , m ]|X[ c , m ]|X[ c , m ]| X[ c , m ]|}\firsthline\hline
№&	$L_1$&	$L_2$&	$L_3$&	$L_4$&	L\\ \hline 
1&	4.6255&	0.3665&	2.1179&	0.8937&	8.0038\\ \hline 
2&	2.8172&	0.4381&	2.1179&	0.8937&	6.2669\\ \hline 
3&	1.9765&	0.5347&	2.1179&	0.8937&	5.5229\\ \hline 
4&	1.9765&	0.6709&	1.5434&	0.8937&	5.0845\\ \hline 
5&	1.4944&	0.8743&	1.5434&	0.8937&	4.8059\\ \hline 
6&	1.4944&	0.8743&	1.1930&	1.1491&	4.7109\\ \hline 
7&	1.1840&	1.2051&	1.1930&	1.1491&	4.7313\\ \hline 
8&	1.4944&	0.8743&	1.1930&	1.1491&	4.7109\\ \hline 
9&	1.4944&	0.8743&	1.5434&	0.8937&	4.8059\\ \hline 
10&	1.4944&	0.8743&	1.1930&	1.1491&	4.7109\\ \hline 
\caption{Пересчитанное $L_j$ и $L$}
\label{tbl_4}
\end{longtabu}

\section{Вывод}
В данной работе была произведена оптимизация ССМО, в частности перераспределение мощностей в системе, для уменьшения очереди.\\\\
Начальные значения:\\
$\mu = (10, 10,10, 10)$\\
$L = 8.0038$\\
$L_j= (4.6255, 0.3655, 2.1179, 0.8937)$\\\\
После оптимизации:\\
$\mu = (10.5, 8, 11, 9.5)$\\
$L = 4.7109$\\
$L_j= (1.4944, 0.8743, 1.1930, 1.1491)$\\\\
Как и ожидалась, после оптимизации сети, загруженность системы понизилась.
\chapter{Решение задачи анализа потокового графа с использованием методики GERT и алгебры потоковых графов}
\section{Постановка задачи}
\textbf{Вариант:} 36.\\\\
\textbf{Дано:}
\begin{enumerate}
\item Граф GERT-сети
\begin{figure}[H]
  \centering
  \fbox{\includegraphics[width=.85\textwidth]{img/scheme_0}}
  \caption{Граф GERT-сети}
\end{figure}
\item Каждой дуге-работе ($ij$) поставлены в соответствие следующие данные:
\begin{enumerate}
\item Закон распределения времени выполнения работы. Будем считать его нормальным;
\item Параметры закона распределения (математическое ожидание \textbf{M} и дисперсия \textbf{D});
\item Вероятность $p_{ij}$ выполнения работы, показанная на графе.
\end{enumerate}
\end{enumerate}


\subsection{Задание}
\subsubsection{Часть 1}
Используя методику GERT, изложенную в книге «Методы анализа сетей»\\
Найти:
\begin{enumerate}
\item Вероятность выхода в завершающий узел графа (для всех вариантов узел 6);
\item Производящую функцию длительности процесса от начального узла  до завершающего узла;
\item Математическое ожидание длительности процесса от начального узла  до завершающего узла;
\item Дисперсию ожидание длительности процесса от начального узла  до завершающего узла;
\end{enumerate}
В отчете перечислить все петли всех порядков, обнаруженные на графе, выписать уравнение Мейсона, получить решение для $W_E(S)$ и найти требуемые параметры. Примерно так, как это сделано в примере на стр. 403–409 книги Филипса и Гарсиа «Методы анализа сетей»
\subsubsection{Часть 2}
Повторить пункты задания 2, 3, 4 используя методику анализа потокового графа, основанную на обработке матрицы передач (Branch Transmittance Matrix). 


Для выполнения задания рекомендуется пользоваться следующими источниками:
\begin{enumerate}
\item Филипс и Гарсиа «Методы анализа сетей»
\item Презентация GERT\_\&\_Flowgraph\_Algebra.pdf (выложена в ИНТРАНЕТ)
\item Ren\_The Methodology of Flowgraph.pdf
\end{enumerate}

\section{Решение}

\subsection{Часть 1}
Чтобы определить эквивалентную W-функцию для анализируемой GERT-сети, необходимо замкнуть сеть дугой, исходящей из узла 6 в узел 1 (рис. \ref{pic_1}).
\begin{figure}[H]
  \centering
  \fbox{\includegraphics[width=\textwidth]{img/scheme_1}}
  \caption{Замкнутая GERT-сеть}
  \label{pic_1}
\end{figure}
Далее, выпишем в таблицу данные графа(мат. ожидание, дисперсия, W-функции)

\tabulinesep = 1mm
\begin{longtabu} to \textwidth {|X[8,c , m ] |X[8,c , m ] | X[15,l, m ]| X[15,l, m ]|X[15,l, m ]|X[25,l, m ]|}\firsthline\hline
\textbf{Начало}&\textbf{Конец}&\textbf{Вес ребра}&\textbf{Мат. ожидание}&\textbf{Дисперсия}&\textbf{W-функция}\\ \hline \endfirsthead
1	&2	&1	&20	&9	&$exp(20s+4.5s^2)$	\\ \hline
2	&2	&0.6&30	&16	&$0.6*exp(30s+8s^2)$	\\ \hline
2	&3	&0.4&40	&25	&$0.4*exp(40s+12.5s^2)$	\\ \hline
3	&2	&0.5&28	&16	&$0.5*exp(28s+8s^2)$	\\ \hline
3	&4	&0.5&37	&16	&$0.5*exp(37s+8s^2)$	\\ \hline
4	&5	&1	&30	&25	&$exp(30s+12.5s^2)$	\\ \hline
5	&1	&0.2&30	&16	&$0.2*exp(30s+8s^2)$	\\ \hline
5	&5	&0.1&10	&4	&$0.1*exp(10s+2s^2)$	\\ \hline
5	&6	&0.7&30	&16	&$0.7*exp(30s+8s^2)$	\\ \hline
\caption{Данные анализируемой GERT-сети}
\end{longtabu}
\textbf{Петли первого порядка:}
\begin{itemize}
\item $W_{12}W_{23}W_{34}W_{45}W_{51}$;
\item $W_{12}W_{23}W_{34}W_{45}W_{56}\frac{1}{W_e}$;
\item $W_{22}$;
\item $W_{23}W_{32}$;
\item $W_{55}$;
\end{itemize}
\textbf{Петли второго порядка:}
\begin{itemize}
\item $W_{22}W_{55}$;
\item $W_{55}W_{23}W_{32}$;
\end{itemize}
\textbf{Уравнение Мейсона}:
\begin{multline*}
H = 1 - W_{12}W_{23}W_{34}W_{45}W_{51} - W_{12}W_{23}W_{34}W_{45}W_{56}\frac{1}{W_e} - W_{22} - W_{23}W_{32} - W_{55}\\
 + W_{22}W_{55} + W_{55}W_{23}W_{32} = 0
\end{multline*}
\textbf{Выведем $W_E(S)$:}
\begin{equation*}
1 - W_{12}W_{23}W_{34}W_{45}W_{51} - W_{22} - W_{23}W_{32} - W_{55} + W_{22}W_{55} + W_{55}W_{23}W_{32} = W_{12}W_{23}W_{34}W_{45}W_{56}\frac{1}{W_e}
\end{equation*}
\begin{multline*}
W_E(S) =\\
(W_{12}W_{23}W_{34}W_{45}W_{56})/(1 - W_{12}W_{23}W_{34}W_{45}W_{51} - W_{22} - W_{23}W_{32} - W_{55} + W_{22}W_{55} + W_{55}W_{23}W_{32})
\end{multline*}

Вычислим математическое ожидание и дисперсию: $M_E(s) = 1$ при $s=0$

Так как $W_E(s)=p_E M_E (s)$,  то  $p_E=W_E(0)$, тогда $M_E(s)=\frac{W_E(s)}{p_E} =\frac{W_E(s)}{W_E(0)}$

Вычисляя первую и вторую производные по $s$ функции $M_E(s)$, и полагая $s=0$, находим математическое ожидание:
\begin{equation*}
\mu_{1E}=\frac{\partial M_E(s)}{\partial s}|s=0
\end{equation*}

и дисперсию:
\begin{equation*}
\sigma^2=\mu_{2E}-[\mu_{1E}]^2
\end{equation*}

Вероятность выхода в завершающий узел графа:
\begin{equation*}
p_E=W_E (0)
\end{equation*}
Для решения задачи был написан скрипт matlab, код приведен в приложении 4.

\begin{lstlisting}[language={matlab}, caption={Результат}, basicstyle=\ttfamily]
We =
-(7*exp((s*(91*s + 314))/2))/(5*exp(2*s*(s + 5)) - 3*exp(10*s*(s + 4)) + 30*exp(2*s*(4*s + 15)) - exp((3*s*(15*s + 52))/2) + 10*exp((s*(41*s + 136))/2) + 2*exp((s*(91*s + 314))/2) - 50)
 
We0 =
1
 
m1 =
2845/7
 
m2 =
11938987/49
 
D =
3844962/49
\end{lstlisting}
Были получены следующие результаты:
\begin{enumerate}
\item Вероятность выхода в завершающий узел графа равна 100\% ($p=W_E=1$).
\item Математическое ожидание 406,43.
\item Дисперсия времени выхода процесса в завершающий узел графа 78 468,61.
\end{enumerate}
\subsection{Часть 2}
Алгоритм дальнейших действий основан на:
\begin{itemize}
\item Презентация GERT\_\&\_Flowgraph\_Algebra.pdf (со слайда 56);
\item Ren\_The Methodology of Flowgraph.pdf (со страницы 35).
\end{itemize}
Определим матрицу Q, не забывая про обратную связь.
\begin{equation*}
Q = 
 \begin{pmatrix}
  0 & q_{12} & 0 & 0 & 0 & 0 \\
  0 & q_{22} & q_{23} & 0 & 0 & 0 \\ 
  0 & q_{32} & 0 & q_{34} & 0 & 0 \\ 
  0 & 0 & 0 & 0 & q_{45} & 0 \\ 
  q_{51} & 0 & 0 & 0 & q_{55} & q_{56} \\ 
  w_{61} & 0 & 0 & 0 & 0 & 0 
 \end{pmatrix}
\end{equation*}
Определим матрицу коэффициентов $A=I_6-Q^T$.
\begin{equation*}
A = 
 \begin{pmatrix}
    1&       0&    0&    0&    -q_{51}& -w_{61}\\
 -q_{12}& 1 - q_{22}& -q_{32}&    0&       0&    0\\
    0&    -q_{23}&    1&    0&       0&    0\\
    0&       0& -q_{34}&    1&       0&    0\\
    0&       0&    0& -q_{45}& 1 - q_{55}&    0\\
    0&       0&    0&    0&    -q_{56}&    1
 \end{pmatrix}
\end{equation*}
Находим 
\begin{equation*}
det(A)
\end{equation*}
далее
\begin{equation*}
\frac{\partial det(A)}{\partial w_{61}}
\end{equation*}
\begin{equation*}
det(A | w_{61}=0)
\end{equation*}
Далее можно вывести $W_E(S)$ с помощью формулы:
\begin{equation*}
W_E(S)=-\frac{\frac{\partial det(A)}{\partial w_{61}}}{det(A | w_{61}=0)}
\end{equation*}
Для расчетов, был написан matlab скрипт, код представлен в приложении 5.

\begin{lstlisting}[language={matlab}, caption={Результат}, basicstyle=\ttfamily]
[    1,       0,    0,    0,    -q51, -w61]
[ -q12, 1 - q22, -q32,    0,       0,    0]
[    0,    -q23,    1,    0,       0,    0]
[    0,       0, -q34,    1,       0,    0]
[    0,       0,    0, -q45, 1 - q55,    0]
[    0,       0,    0,    0,    -q56,    1]
 
 
-(q12*q23*q34*q45*q56)/(q22 + q55 + q23*q32 - q22*q55 - q23*q32*q55 + q12*q23*q34*q45*q51 - 1)
\end{lstlisting}
Во второй строчке был получен $W_E(S)$, который полностью(за исключением знаков) совпадает с $W_E(S)$ найденным в части 1.

Далее, имея $W_E(S)$ находим необходимые переменные.

Для расчетов, был использован скрипт из приложения 6.

\begin{lstlisting}[language={matlab}, caption={Результат}, basicstyle=\ttfamily]
We =
-(7*exp((s*(91*s + 314))/2))/(5*exp(2*s*(s + 5)) - 3*exp(10*s*(s + 4)) + 30*exp(2*s*(4*s + 15)) - exp((3*s*(15*s + 52))/2) + 10*exp((s*(41*s + 136))/2) + 2*exp((s*(91*s + 314))/2) - 50)
 
We0 =
1
 
m1 =
2845/7
 
m2 =
11938987/49
 
D =
3844962/49
\end{lstlisting}
Были получены следующие результаты:
\begin{enumerate}
\item Вероятность выхода в завершающий узел графа равна 100\% ($p=W_E=1$).
\item Математическое ожидание 406,43.
\item Дисперсия времени выхода процесса в завершающий узел графа 78 468,61.
\end{enumerate}
Которые полностью совпадает с результатами части 1.

\addcontentsline{toc}{chapter}{Заключение}
\chapter*{Заключение}
В работе были рассмотрены различные математические модели для решения задачи выбора оптимального решения.

При анализе результатов решения многокритериальной задачи можно заметить, что аддитивная и мультипликативная свертка выдают одинаковый, наиболее оптимальный результат. Наилучший результат этих методов обусловлен наличием коэффициентов значимости. Методы, не подразумевающие введение весовых коэффициентов показывают похожий результат, который в целом хуже, чем у аддитивной и мультипликативной свертки.

В процессе поиска оптимальной стратегии принятия решений с использованием марковских моделей были определены оптимальные стратегии для конкретной задачи. Линейное программирование позволяет достаточно легко и быстро находить оптимальную стратегию, однако требует некоторых предварительных преобразований и формализаций перед использованием.

Как и ожидалось, при оптимизации многоканальной замкнутой ССМО алгоритм простых итераций не сходится, поэтому пришлось использовать более надежный алгоритм Ньютона. Модифицированный алгоритм Ньютона чрезвычайно чувствителен к выбору начального приближения, поэтому был использован вариант алгоритма с расчетом матрицы Якоби. Для увеличения производительности данного алгоритма можно подключить пакет MATLAB Parallel Computing Toolbox, который использует MPI для эффективного распараллеливания.

В результате решения задачи анализа потокового графа можно сделать вывод, что при  заданных значениях вероятности, мат. ожидания и дисперсии для каждой дуги исходного графа достаточно легко расчитываются W-функции, которые необходимы для построения формулы Мейсона. После этого из формулы Мейсона по формулам математической статистики достаточно легко расчитывается результирующее мат. ожидание и дисперсия. Решение путем анализа потокового графа показало аналогичные результаты, что подтверждает корректность решения. Однако, метод анализа потокового графа выполняется заметно медленнее, даже на небольшом графе.

Среда MATLAB действительно хорошо подходит для решения оптимизационных задач засчет множества специфичных математических библиотек и пакетов.

%------------------------------------------------------------------------------

%\addcontentsline{toc}{section}{Список литературы}
%\bibliography{thesis}
%\bibliographystyle{ugost2008}

%\nocite{phill}
%\nocite{Ren}

\addcontentsline{toc}{chapter}{Список использованных источников}
\chapter*{Список использованных источников}
\begin{enumerate}
\item Д. Филлипс, А. Гарсиа-Диас. Методы анализа сетей: Пер. с англ. — М.: Мир, 1984.—496 с, ил.
\item Ren, Yu. The methodology of flowgraph models. PhD thesis, The London School of Economics and Political Science (LSE). –– 2011.
\item Сиднев А. Г. Системный анализ. Часть 1 [Электронный ресурс] // Интранет-
портал ИИТУ СПбПУ. 2018. URL: http://intranet.ftk.spbstu.ru/docinfo.php?InfoFtkDocumentID= 1386947 (дата обращения: 22.04.2018)
\item Горбунов В.М. Теория принятия решений: учебное пособие. Томск: Изд-во
Томск. политех. ун-та, 2010.
\item Системный анализ и принятие решений: учебное пособие / Е.Н. Бендерская,
Д.Н. Колесников, В.И. Пахомова и др.; Под ред. Д.Н. Колесникова. 2-е изд.
СПб.: Изд-во СПбГПУ, 2001.
\item Optimization Toolbox: описание функции FGOALATTAIN [Электронный ре-
сурс] // MATLAB.Exponenta, центр компетенций MathWorks. URL:
http://matlab.exponenta.ru/ optimiz/book\_4/2/fgoalattain.php (дата обращения:
17.02.2018).
\item Таха Х. А. Введение в исследование операций: Пер. с англ. 7-е изд. М.: Изда-тельский дом «Вильямс», 2005.
\end{enumerate}


%\bibliography{thesis}
%\bibliographystyle{ugost2008}


\clearpage
\addcontentsline{toc}{chapter}{Приложения}
\setcounter{section}{0}
\section*{Приложение 1} \label{p1:1}
\textbf{Решение задачи методом линейного программирования}
\begin{lstlisting}[language={matlab}, caption={Решение задачи методом линейного программирования}, label={lst:0}, basicstyle={\footnotesize\ttfamily}, breaklines={true}]
clear all;
close all; 
clc;
format long g;

% -F1 -> min
mF1 = @(X) -(60 * X(1) + 300 * X(2) + 2000 * X(3));
% -F2 -> min
mF2 = @(X) -((0.1 * X(1) + 10 * X(2) + 70 * sqrt(X(3))) ^ 1.5);
% F3 -> min
pF3 = @(X) 30 * X(1) + 100 * X(2) + 220 * X(3);
% F4 -> min
pF4 = @(X) log(20 * X(1) + 3 * X(2) + 0.01 * X(3));
% -F5 -> min
mF5 = @(X) -((-mF1(X)) / (X(1) + X(2) + X(3)));
% -F6 -> min
mF6 = @(X) -(((-mF1(X)) + (-mF2(X))) - (pF3(X) + pF4(X)));

A = [1, 0, 0;
-1, 0, 0;
0, 1, 0;
0, -1, 0;
0, 0, 1;
0, 0, -1;
1, 1, 1];

B = [70;
-35;
30;
-15;
1;
-0.5;
80];

Aeq = [];
Beq = [];

S = [35; 15; 0.5];

[x1, result1] = fmincon(mF1, S, A, B, Aeq, Beq);
[x2, result2] = fmincon(mF2, S, A, B, Aeq, Beq);
[x3, result3] = fmincon(pF3, S, A, B, Aeq, Beq);
[x4, result4] = fmincon(pF4, S, A, B, Aeq, Beq);
[x5, result5] = fmincon(mF5, S, A, B, Aeq, Beq);
[x6, result6] = fmincon(mF6, S, A, B, Aeq, Beq);

formatter = '-- F1 max --\n%s\n%s\n\n-- F2 max --\n%s\n%s\n\n-- F3 min --\n%s\n%s\n\n-- F4 min --\n%s\n%s\n\n-- F5 max --\n%s\n%s\n\n-- F6 max --\n%s\n%s\n\n';
s1 = sprintf('x1 = %.3f, x2 = %.3f, x3 = %.3f, x1 + x2 + x3 = %.3f', x1, sum(x1));
s2 = sprintf('!F1 = %.3f, F2 = %.3f, F3 = %.3f, F4 = %.3f, F5 = %.3f, F6 = %.3f', -result1, -mF2(x1), pF3(x1), pF4(x1), -mF5(x1), -mF6(x1));
s3 = sprintf('x1 = %.3f, x2 = %.3f, x3 = %.3f, x1 + x2 + x3 = %.3f', x2, sum(x2));
s4 = sprintf('F1 = %.3f, !F2 = %.3f, F3 = %.3f, F4 = %.3f, F5 = %.3f, F6 = %.3f', -mF1(x2), -result2, pF3(x2), pF4(x2), -mF5(x2), -mF6(x2)); 
s5 = sprintf('x1 = %.3f, x2 = %.3f, x3 = %.3f, x1 + x2 + x3 = %.3f', x3, sum(x3));
s6 = sprintf('F1 = %.3f, F2 = %.3f, !F3 = %.3f, F4 = %.3f, F5 = %.3f, F6 = %.3f', -mF1(x3), -mF2(x3), result3, pF4(x3), -mF5(x3), -mF6(x3)); 
s7 = sprintf('x1 = %.3f, x2 = %.3f, x3 = %.3f, x1 + x2 + x3 = %.3f', x4, sum(x4));
s8 = sprintf('F1 = %.3f, F2 = %.3f, F3 = %.3f, !F4 = %.3f, F5 = %.3f, F6 = %.3f', -mF1(x4), -mF2(x4), pF3(x4), result4, -mF5(x4), -mF6(x4));
s9 = sprintf('x1 = %.3f, x2 = %.3f, x3 = %.3f, x1 + x2 + x3 = %.3f', x5, sum(x5));
s10 = sprintf('F1 = %.3f, F2 = %.3f, F3 = %.3f, F4 = %.3f, !F5 = %.3f, F6 = %.3f', -mF1(x5), -mF2(x5), pF3(x5), pF4(x5), -result5, -mF6(x5)); 
s11 = sprintf('x1 = %.3f, x2 = %.3f, x3 = %.3f, x1 + x2 + x3 = %.3f', x6, sum(x6));
s12 = sprintf('F1 = %.3f, F2 = %.3f, F3 = %.3f, F4 = %.3f, F5 = %.3f, !F6 = %.3f', -mF1(x6), -mF2(x6), pF3(x6), pF4(x6), -mF5(x6), -result6);
fprintf(formatter, s1, s2, s3, s4, s5, s6, s7, s8, s9, s10, s11, s12);
\end{lstlisting}

\section*{Приложение 2} \label{p1:2}
\textbf{Решение задачи при помощи аддитивной свертки}
\begin{lstlisting}[language={matlab}, caption={Решение задачи при помощи аддитивной свертки}, label={lst:0}, basicstyle={\footnotesize\ttfamily}, breaklines={true}]
clear all;
close all; 
clc;
format long g;

% -F1 -> min
mF1 = @(X) -(60 * X(1) + 300 * X(2) + 2000 * X(3));
% -F2 -> min
mF2 = @(X) -((0.1 * X(1) + 10 * X(2) + 70 * sqrt(X(3))) ^ 1.5);
% F3 -> min
pF3 = @(X) 30 * X(1) + 100 * X(2) + 220 * X(3);
% F4 -> min
pF4 = @(X) log(20 * X(1) + 3 * X(2) + 0.01 * X(3));
% -F5 -> min
mF5 = @(X) -((-mF1(X)) / (X(1) + X(2) + X(3)));
% -F6 -> min
mF6 = @(X) -(((-mF1(X)) + (-mF2(X))) - (pF3(X) + pF4(X)));

cF2 = 0.15;
rF2 = 7258.939;
cF3 = 0.15;
rF3 = 2660;
cF4 = 0.05;
rF4 = 6.613;
cF5 = 0.05;
rF5 = 198.485;
cF6 = 0.6;
rF6 = 16501.964;

pFS = @(X) cF2 * (mF2(X) / rF2) + cF3 * (pF3(X) / rF3) + cF4 * (pF4(X) / rF4) + cF5 * (mF5(X) / rF5) + cF6 * (mF6(X) / rF6);

A = [1, 0, 0;
-1, 0, 0;
0, 1, 0;
0, -1, 0;
0, 0, 1;
0, 0, -1;
1, 1, 1];

B = [70;
-35;
30;
-15;
1;
-0.5;
80];

Aeq = [];
Beq = [];

S = [35; 15; 0.5];

[xS, resultS] = fmincon(pFS, S, A, B, Aeq, Beq);

formatter = '-- FS --\n%s\n%s\n%s\n%s\n\n';
s1 = sprintf('FS = %.3f', resultS);
s2 = sprintf('x1 = %.3f, x2 = %.3f, x3 = %.3f, x1 + x2 + x3 = %.3f', xS, sum(xS));
s3 = sprintf('F2 = %.3f, F3 = %.3f, F4 = %.3f, F5 = %.3f, F6 = %.3f', -mF2(xS), pF3(xS), pF4(xS), -mF5(xS), -mF6(xS));
s4 = sprintf('F2/rF2 = %.3f%%, F3/rF3 = %.3f%%, F4/rF4 = %.3f%%, F5/rF5 = %.3f%%, F6/rF6 = %.3f%%', -mF2(xS) / rF2 * 100, pF3(xS) / rF3 * 100, pF4(xS) / rF4 * 100, -mF5(xS) / rF5 * 100, -mF6(xS) / rF6 * 100);
fprintf(formatter, s1, s2, s3, s4);
\end{lstlisting}

\section*{Приложение 3} \label{p1:3}
\textbf{Решение задачи при помощи мультипликативной свертки}
\begin{lstlisting}[language={matlab}, caption={Решение задачи при помощи мультипликативной свертки}, label={lst:0}, basicstyle={\footnotesize\ttfamily}, breaklines={true}]
clear all;
close all; 
clc;
format long g;

% 1 / F1 -> min
mF1 = @(X) 1 / (60 * X(1) + 300 * X(2) + 2000 * X(3));
% 1 / F2 -> min
mF2 = @(X) 1 / ((0.1 * X(1) + 10 * X(2) + 70 * sqrt(X(3))) ^ 1.5);
% F3 -> min
pF3 = @(X) 30 * X(1) + 100 * X(2) + 220 * X(3);
% F4 -> min
pF4 = @(X) log(20 * X(1) + 3 * X(2) + 0.01 * X(3));
% 1 / F5 -> min
mF5 = @(X) 1 / ((1 / mF1(X)) / (X(1) + X(2) + X(3)));
% 1 / F6 -> min
mF6 = @(X) 1 / (((1 / mF1(X)) + (1 / mF2(X))) - (pF3(X) + pF4(X)));

cF2 = 0.15;
rF2 = 7258.939;
cF3 = 0.15;
rF3 = 2660;
cF4 = 0.05;
rF4 = 6.613;
cF5 = 0.05;
rF5 = 198.485;
cF6 = 0.6;
rF6 = 16501.964;

pFS = @(X) nthroot(mF2(X) * rF2, 1 / cF2) * nthroot(pF3(X) / rF3, 1 / cF3) * nthroot(pF4(X) / rF4, 1 / cF4) * nthroot(mF5(X) * rF5, 1 / cF5) * nthroot(mF6(X) * rF6, 1 / cF6);

A = [1, 0, 0;
-1, 0, 0;
0, 1, 0;
0, -1, 0;
0, 0, 1;
0, 0, -1;
1, 1, 1];

B = [70;
-35;
30;
-15;
1;
-0.5;
80];

Aeq = [];
Beq = [];

S = [35; 15; 0.5];

[xS, resultS] = fmincon(pFS, S, A, B, Aeq, Beq);

formatter = '-- FS --\n%s\n%s\n%s\n%s\n\n';
s1 = sprintf('FS = %.3f', resultS);
s2 = sprintf('x1 = %.3f, x2 = %.3f, x3 = %.3f, x1 + x2 + x3 = %.3f', xS, sum(xS));
s3 = sprintf('F2 = %.3f, F3 = %.3f, F4 = %.3f, F5 = %.3f, F6 = %.3f', 1 / mF2(xS), pF3(xS), pF4(xS), 1 / mF5(xS), 1 / mF6(xS));
s4 = sprintf('F2/rF2 = %.3f%%, F3/rF3 = %.3f%%, F4/rF4 = %.3f%%, F5/rF5 = %.3f%%, F6/rF6 = %.3f%%', (1 / mF2(xS)) / rF2 * 100, pF3(xS) / rF3 * 100, pF4(xS) / rF4 * 100, (1 / mF5(xS)) / rF5 * 100, (1 / mF6(xS)) / rF6 * 100);
fprintf(formatter, s1, s2, s3, s4);
\end{lstlisting}

\section*{Приложение 4} \label{p1:4}
\textbf{Решение задачи при помощи функции fminimax}
\begin{lstlisting}[language={matlab}, caption={Решение задачи при помощи функции fminimax}, label={lst:0}, basicstyle={\footnotesize\ttfamily}, breaklines={true}]
clear all;
close all; 
clc;
format long g;

% 1 / F1 -> min
mF1 = @(X) 1 / (60 * X(1) + 300 * X(2) + 2000 * X(3));
% 1 / F2 -> min
mF2 = @(X) 1 / ((0.1 * X(1) + 10 * X(2) + 70 * sqrt(X(3))) ^ 1.5);
% F3 -> min
pF3 = @(X) 30 * X(1) + 100 * X(2) + 220 * X(3);
% F4 -> min
pF4 = @(X) log(20 * X(1) + 3 * X(2) + 0.01 * X(3));
% 1 / F5 -> min
mF5 = @(X) 1 / ((1 / mF1(X)) / (X(1) + X(2) + X(3)));
% 1 / F6 -> min
mF6 = @(X) 1 / (((1 / mF1(X)) + (1 / mF2(X))) - (pF3(X) + pF4(X)));

rF1 = 13940;
rF2 = 7258.939;
rF3 = 2660;
rF4 = 6.613;
rF5 = 198.485;
rF6 = 16501.964;

A = [1, 0, 0;
-1, 0, 0;
0, 1, 0;
0, -1, 0;
0, 0, 1;
0, 0, -1;
1, 1, 1];

B = [70;
-35;
30;
-15;
1;
-0.5;
80];

Aeq = [];
Beq = [];

S = [35; 15; 0.5];

[xS, resultS] = fminimax(@m4f, S, A, B, Aeq, Beq);

formatter = '-- FS --\n%s\n%s\n%s\n%s\n\n';
s1 = sprintf('FS = (%.3f, %.3f, %.3f, %.3f, %.3f)', resultS(2:6));
s2 = sprintf('x1 = %.3f, x2 = %.3f, x3 = %.3f, x1 + x2 + x3 = %.3f', xS, sum(xS));
s3 = sprintf('F2 = %.3f, F3 = %.3f, F4 = %.3f, F5 = %.3f, F6 = %.3f', 1 / mF2(xS), pF3(xS), pF4(xS), 1 / mF5(xS), 1 / mF6(xS));
s4 = sprintf('F2/rF2 = %.3f%%, F3/rF3 = %.3f%%, F4/rF4 = %.3f%%, F5/rF5 = %.3f%%, F6/rF6 = %.3f%%', (1 / mF2(xS)) / rF2 * 100, pF3(xS) / rF3 * 100, pF4(xS) / rF4 * 100, (1 / mF5(xS)) / rF5 * 100, (1 / mF6(xS)) / rF6 * 100);
fprintf(formatter, s1, s2, s3, s4);
\end{lstlisting}

\begin{lstlisting}[language={matlab}, caption={Внутренняя функция для fminimax}, label={lst:0}, basicstyle={\footnotesize\ttfamily}, breaklines={true}]
function result = m4f(X)
	% 1 / F1 -> min
	mF1 = 1 / (60 * X(1) + 300 * X(2) + 2000 * X(3));
	% 1 / F2 -> min
	mF2 = 1 / ((0.1 * X(1) + 10 * X(2) + 70 * sqrt(X(3))) ^ 1.5);
	% F3 -> min
	pF3 = 30 * X(1) + 100 * X(2) + 220 * X(3);
	% F4 -> min
	pF4 = log(20 * X(1) + 3 * X(2) + 0.01 * X(3));
	% 1 / F5 -> min
	mF5 = 1 / ((1 / mF1) / (X(1) + X(2) + X(3)));
	% 1 / F6 -> min
	mF6 = 1 / (((1 / mF1) + (1 / mF2)) - (pF3 + pF4));
	
	rF1 = 13940;
	rF2 = 7258.939;
	rF3 = 2660;
	rF4 = 6.613;
	rF5 = 198.485;
	rF6 = 16501.964;
	
	result(1) = mF1 * rF1;
	result(2) = mF2 * rF2;
	result(3) = pF3 / rF3;
	result(4) = pF4 / rF4;
	result(5) = mF5 * rF5;
	result(6) = mF6 * rF6;
end
\end{lstlisting}

\section*{Приложение 5} \label{p1:5}
\textbf{Решение задачи при помощи метода последовательных уступок}
\begin{lstlisting}[language={matlab}, caption={Решение задачи для критерия F7}, label={lst:0}, basicstyle={\footnotesize\ttfamily}, breaklines={true}]
clear all;
close all; 
clc;
format long g;

% -F7 -> min
mF7 = @(X) -(30 * X(1) + 200 * X(2) + 1780 * X(3));
% -F8 -> min
mF8 = @(X) -(X(1) + 10 * X(3));
% F9 -> min
pF9 = @(X) X(1) + 2 * X(2);

A = [1, 0, 0;
-1, 0, 0;
0, 1, 0;
0, -1, 0;
0, 0, 1;
0, 0, -1;
1, 1, 1];

B = [70;
-35;
30;
-15;
1;
-0.5;
80];

Aeq = [];
Beq = [];

S = [35; 15; 0.5];

[x7, result7] = fmincon(mF7, S, A, B, Aeq, Beq);

formatter = '-- F7 max --\n%s\n%s\n\n';
s1 = sprintf('x1 = %.3f, x2 = %.3f, x3 = %.3f, x1 + x2 + x3 = %.3f', x7, sum(x7));
s2 = sprintf('!F7 = %.3f, F8 = %.3f, F9 = %.3f', -result7, -mF8(x7), pF9(x7)); 
fprintf(formatter, s1, s2);
\end{lstlisting}

\begin{lstlisting}[language={matlab}, caption={Решение задачи для критерия F8}, label={lst:0}, basicstyle={\footnotesize\ttfamily}, breaklines={true}]
clear all;
close all; 
clc;
format long g;

% -F7 -> min
mF7 = @(X) -(30 * X(1) + 200 * X(2) + 1780 * X(3));
% -F8 -> min
mF8 = @(X) -(X(1) + 10 * X(3));
% F9 -> min
pF9 = @(X) X(1) + 2 * X(2);

K = 0.85

rF7 = 9250;

A = [1, 0, 0;
-1, 0, 0;
0, 1, 0;
0, -1, 0;
0, 0, 1;
0, 0, -1;
1, 1, 1;
-30, -200, -1780];

B = [70;
-35;
30;
-15;
1;
-0.5;
80;
-rF7 * K];

Aeq = [];
Beq = [];

S = [35; 15; 0.5];

[x8, result8] = fmincon(mF8, S, A, B, Aeq, Beq);

fprintf('%.3f >= F7 >= %.1f * %.3f\n', rF7, K, rF7);
fprintf('%.3f >= F7 >= %.3f\n\n', rF7, rF7 * K);

formatter = '-- F8 max --\n%s\n%s\n\n';
s1 = sprintf('x1 = %.3f, x2 = %.3f, x3 = %.3f, x1 + x2 + x3 = %.3f', x8, sum(x8));
s2 = sprintf('F7 = %.3f, !F8 = %.3f, F9 = %.3f', -mF7(x8), -result8, pF9(x8)); 
fprintf(formatter, s1, s2);
\end{lstlisting}

\begin{lstlisting}[language={matlab}, caption={Решение задачи для критерия F9}, label={lst:0}, basicstyle={\footnotesize\ttfamily}, breaklines={true}]
clear all;
close all; 
clc;
format long g;

% -F7 -> min
mF7 = @(X) -(30 * X(1) + 200 * X(2) + 1780 * X(3));
% -F8 -> min
mF8 = @(X) -(X(1) + 10 * X(3));
% F9 -> min
pF9 = @(X) X(1) + 2 * X(2);

K = 0.85

rF7 = 9250;
rF8 = 67.162;

A = [1, 0, 0;
-1, 0, 0;
0, 1, 0;
0, -1, 0;
0, 0, 1;
0, 0, -1;
1, 1, 1;
-30, -200, -1780;
-1, 0, -10];

B = [70;
-35;
30;
-15;
1;
-0.5;
80;
-rF7 * K;
-rF8 * K];

Aeq = [];
Beq = [];

S = [35; 15; 0.5];

[x9, result9] = fmincon(pF9, S, A, B, Aeq, Beq);

fprintf('%.3f >= F7 >= %.2f * %.3f\n', rF7, K, rF7);
fprintf('%.3f >= F7 >= %.3f\n\n', rF7, rF7 * K);

fprintf('%.3f >= F8 >= %.2f * %.3f\n', rF8, K, rF8);
fprintf('%.3f >= F8 >= %.3f\n\n', rF8, rF8 * K);

formatter = '-- F9 max --\n%s\n%s\n\n';
s1 = sprintf('x1 = %.3f, x2 = %.3f, x3 = %.3f, x1 + x2 + x3 = %.3f', x9, sum(x9));
s2 = sprintf('F7 = %.3f, F8 = %.3f, !F9 = %.3f', -mF7(x9), -mF8(x9), result9); 
fprintf(formatter, s1, s2);
\end{lstlisting}

\section*{Приложение 6} \label{p1:6}
\textbf{Решение задачи при помощи метода достижения цели (fgoalattain)}
\begin{lstlisting}[language={matlab}, caption={Решение задачи при помощи метода достижения цели (fgoalattain)}, label={lst:0}, basicstyle={\footnotesize\ttfamily}, breaklines={true}]
clear all;
close all; 
clc;
format long g;

% -F1 -> min
mF1 = @(X) -(60 * X(1) + 300 * X(2) + 2000 * X(3));
% -F2 -> min
mF2 = @(X) -((0.1 * X(1) + 10 * X(2) + 70 * sqrt(X(3))) ^ 1.5);
% F3 -> min
pF3 = @(X) 30 * X(1) + 100 * X(2) + 220 * X(3);
% F4 -> min
pF4 = @(X) log(20 * X(1) + 3 * X(2) + 0.01 * X(3));
% -F5 -> min
mF5 = @(X) -((-mF1(X)) / (X(1) + X(2) + X(3)));
% -F6 -> min
mF6 = @(X) -(((-mF1(X)) + (-mF2(X))) - (pF3(X) + pF4(X)));

rF1 = -13940;
rF2 = -7258.939;
rF3 = 2660;
rF4 = 6.613;
rF5 = -198.485;
rF6 = -16501.964;

pFS = @(X) [mF2(X), pF3(X), pF4(X), mF5(X), mF6(X)];

G = [rF2, rF3, rF4, rF5, rF6];

W = abs(G);

A = [1, 0, 0;
-1, 0, 0;
0, 1, 0;
0, -1, 0;
0, 0, 1;
0, 0, -1;
1, 1, 1];

B = [70;
-35;
30;
-15;
1;
-0.5;
80];

Aeq = [];
Beq = [];

S = [35; 15; 0.5];

[xS, resultS] = fgoalattain(pFS, S, G, W, A, B, Aeq, Beq);

formatter = '-- FS --\n%s\n%s\n%s\n%s\n\n';
s1 = sprintf('FS = (%.3f, %.3f, %.3f, %.3f, %.3f)', resultS);
s2 = sprintf('x1 = %.3f, x2 = %.3f, x3 = %.3f, x1 + x2 + x3 = %.3f', xS, sum(xS));
s3 = sprintf('F2 = %.3f, F3 = %.3f, F4 = %.3f, F5 = %.3f, F6 = %.3f', -mF2(xS), pF3(xS), pF4(xS), -mF5(xS), -mF6(xS));
s4 = sprintf('F2/rF2 = %.3f%%, F3/rF3 = %.3f%%, F4/rF4 = %.3f%%, F5/rF5 = %.3f%%, F6/rF6 = %.3f%%', mF2(xS) / rF2 * 100, pF3(xS) / rF3 * 100, pF4(xS) / rF4 * 100, mF5(xS) / rF5 * 100, mF6(xS) / rF6 * 100);
fprintf(formatter, s1, s2, s3, s4);
\end{lstlisting}

\section*{Приложение 7} \label{p1:7}
\textbf{Решение задачи при помощи введения метрики в пространстве критериев}
\begin{lstlisting}[language={matlab}, caption={Решение задачи при помощи введения метрики в пространстве критериев}, label={lst:0}, basicstyle={\footnotesize\ttfamily}, breaklines={true}]
clear all;
close all; 
clc;
format long g;

% -F1 -> min
mF1 = @(X) -(60 * X(1) + 300 * X(2) + 2000 * X(3));
% -F2 -> min
mF2 = @(X) -((0.1 * X(1) + 10 * X(2) + 70 * sqrt(X(3))) ^ 1.5);
% F3 -> min
pF3 = @(X) 30 * X(1) + 100 * X(2) + 220 * X(3);
% F4 -> min
pF4 = @(X) log(20 * X(1) + 3 * X(2) + 0.01 * X(3));
% -F5 -> min
mF5 = @(X) -((-mF1(X)) / (X(1) + X(2) + X(3)));
% -F6 -> min
mF6 = @(X) -(((-mF1(X)) + (-mF2(X))) - (pF3(X) + pF4(X)));

rF1 = -13940;
rF2 = -7258.939;
rF3 = 2660;
rF4 = 6.613;
rF5 = -198.485;
rF6 = -16501.964;

pFS = @(X) (1 - mF2(X) / rF2) ^ 2 + (1 - pF3(X) / rF3) ^ 2 + (1 - pF4(X) / rF4) ^ 2 + (1 - mF5(X) / rF5) ^ 2 + (1 - mF6(X) / rF6) ^ 2;

A = [1, 0, 0;
-1, 0, 0;
0, 1, 0;
0, -1, 0;
0, 0, 1;
0, 0, -1;
1, 1, 1];

B = [70;
-35;
30;
-15;
1;
-0.5;
80];

Aeq = [];
Beq = [];

S = [35; 15; 0.5];

[xS, resultS] = fmincon(pFS, S, A, B, Aeq, Beq);

formatter = '-- FS --\n%s\n%s\n%s\n%s\n\n';
s1 = sprintf('FS = %.3f', resultS);
s2 = sprintf('x1 = %.3f, x2 = %.3f, x3 = %.3f, x1 + x2 + x3 = %.3f', xS, sum(xS));
s3 = sprintf('F2 = %.3f, F3 = %.3f, F4 = %.3f, F5 = %.3f, F6 = %.3f', -mF2(xS), pF3(xS), pF4(xS), -mF5(xS), -mF6(xS));
s4 = sprintf('F2/rF2 = %.3f%%, F3/rF3 = %.3f%%, F4/rF4 = %.3f%%, F5/rF5 = %.3f%%, F6/rF6 = %.3f%%', mF2(xS) / rF2 * 100, pF3(xS) / rF3 * 100, pF4(xS) / rF4 * 100, mF5(xS) / rF5 * 100, mF6(xS) / rF6 * 100);
fprintf(formatter, s1, s2, s3, s4);
\end{lstlisting}

\section*{Приложение 8} \label{p1:8}
\textbf{Решение задачи стохастического программирования}
\begin{lstlisting}[language={matlab}, caption={Решение задачи стохастического программирования}, label={lst:0}, basicstyle={\footnotesize\ttfamily}, breaklines={true}]
clear all;
close all; 
clc;
format long g;

% -F1 -> min
mF1 = @(X) -(60 * X(1) + 300 * X(2) + 2000 * X(3));
% -F2 -> min
mF2 = @(X) -((0.1 * X(1) + 10 * X(2) + 70 * sqrt(X(3))) ^ 1.5);
% F3 -> min
pF3 = @(X) 30 * X(1) + 100 * X(2) + 220 * X(3);
% F4 -> min
pF4 = @(X) log(20 * X(1) + 3 * X(2) + 0.01 * X(3));
% -F5 -> min
mF5 = @(X) -((-mF1(X)) / (X(1) + X(2) + X(3)));
% -F6 -> min
mF6 = @(X) -(((-mF1(X)) + (-mF2(X))) - (pF3(X) + pF4(X)));

rF1 = -13940;
rF2 = -7258.939;
rF3 = 2660;
rF4 = 6.613;
rF5 = -198.485;
rF6 = -16501.964;

pFS = @(X) (1 - mF2(X) / rF2) ^ 2 + (1 - pF3(X) / rF3) ^ 2 + (1 - pF4(X) / rF4) ^ 2 + (1 - mF5(X) / rF5) ^ 2 + (1 - mF6(X) / rF6) ^ 2;

A = [1, 0, 0;
-1, 0, 0;
0, 1, 0;
0, -1, 0;
0, 0, 1;
0, 0, -1;
1, 1, 1];

B = [70;
-35;
30;
-15;
1;
-0.5;
80];

Aeq = [];
Beq = [];

lb = [];
ub = [];

S = [35; 15; 0.5];

O = optimoptions('fmincon', 'Display', 'none');

[xS, resultS] = fmincon(pFS, S, A, B, Aeq, Beq, lb, ub, [], O);

global K;

Ka = icdf('Normal', 0.1 : 0.1 : 0.9, 0, 1);
sizeK = size(Ka, 2);
xV = zeros(3, sizeK);
rV = zeros(3, sizeK);
for i = 1 : 1 : sizeK
K = Ka(i);
fmincon(pFS, S, A, B, Aeq, Beq);

[xV(:, i), rV(:, i)] = fmincon(pFS, S, A(1:6, :), B(1:6, :), Aeq, Beq, lb, ub, @m8f, O);
end

mean = (abs(mF2(xS) / rF2 - 1) + abs(pF3(xS) / rF3 - 1) + abs(pF4(xS) / rF4 - 1) + abs(mF5(xS) / rF5 - 1) + abs(mF6(xS) / rF6 - 1)) / 5 * 100;

formatter = '-- FS --\n%s\n%s\n%s\n%s\n%s\n\n';
s1 = sprintf('FS = %.3f', resultS);
s2 = sprintf('x1 = %.3f, x2 = %.3f, x3 = %.3f, x1 + x2 + x3 = %.3f', xS, sum(xS));
s3 = sprintf('F2 = %.3f, F3 = %.3f, F4 = %.3f, F5 = %.3f, F6 = %.3f', -mF2(xS), pF3(xS), pF4(xS), -mF5(xS), -mF6(xS));
s4 = sprintf('F2/rF2 = %.3f%%, F3/rF3 = %.3f%%, F4/rF4 = %.3f%%, F5/rF5 = %.3f%%, F6/rF6 = %.3f%%', mF2(xS) / rF2 * 100, pF3(xS) / rF3 * 100, pF4(xS) / rF4 * 100, mF5(xS) / rF5 * 100, mF6(xS) / rF6 * 100);
s5 = sprintf('Mean = %.3f%%', mean);
fprintf(formatter, s1, s2, s3, s4, s5);

for i = 1 : 1 : sizeK
mean = (abs(mF2(xV(:, i)) / rF2 - 1) + abs(pF3(xV(:, i)) / rF3 - 1) + abs(pF4(xV(:, i)) / rF4 - 1) + abs(mF5(xV(:, i)) / rF5 - 1) + abs(mF6(xV(:, i)) / rF6 - 1)) / 5 * 100;

formatter = '-- %s --\n%s\n%s\n%s\n%s\n\n';
s1 = sprintf('a = 0.%d, Ka = %.3f', i, Ka(i));
s2 = sprintf('x1 = %.3f, x2 = %.3f, x3 = %.3f, x1 + x2 + x3 = %.3f', xV(:, i), sum(xV(:, i)));
s3 = sprintf('F2 = %.3f, F3 = %.3f, F4 = %.3f, F5 = %.3f, F6 = %.3f', -mF2(xV(:, i)), pF3(xV(:, i)), pF4(xV(:, i)), -mF5(xV(:, i)), -mF6(xV(:, i)));
s4 = sprintf('F2/rF2 = %.3f%%, F3/rF3 = %.3f%%, F4/rF4 = %.3f%%, F5/rF5 = %.3f%%, F6/rF6 = %.3f%%', mF2(xV(:, i)) / rF2 * 100, pF3(xV(:, i)) / rF3 * 100, pF4(xV(:, i)) / rF4 * 100, mF5(xV(:, i)) / rF5 * 100, mF6(xV(:, i)) / rF6 * 100);
s5 = sprintf('Mean = %.3f%%', mean);
fprintf(formatter, s1, s2, s3, s4, s5);
end

\end{lstlisting}

\begin{lstlisting}[language={matlab}, caption={Внутренняя функция для решения задачи стохастического программирования}, label={lst:0}, basicstyle={\footnotesize\ttfamily}, breaklines={true}]
function [c, ceq] = m8f(X)
	global K;
	
	M1 = 1;
	M2 = 1;
	M3 = 1;
	D1 = M1 / 2;
	D2 = M2 / 2;
	D3 = M3 / 2;
	
	c = M1 * X(1) + M2 * X(2) + M3 * X(3) + K * sqrt(D1 * (X(1) ^ 2) + D2 * (X(2) ^ 2) + D3 * (X(3) ^ 2)) - 80;
	ceq = [];
end
\end{lstlisting}

\section*{Приложение 9} \label{p2:9}
\textbf{Поиск оптимальной стратегии при помощи линейного программирования}
\begin{lstlisting}[language={matlab}, caption={Поиск оптимальной стратегии при помощи линейного программирования}, label={lst:0}, basicstyle={\footnotesize\ttfamily}, breaklines={true}]
clear all;
close all; 
clc;
format long g;

f = [0 12.6 8.4 14.7 12.3];

A=[];
b=[];

Aeq = [0.4 -0.9 -0.7 -0.6 -0.5;
-0.3 0.9 0.8 -0.3 -0.4;
-0.1 0 -0.1 0.9 0.9;
1 1 1 1 1];
Beq = [0; 0; 0; 1];

lb = zeros(5, 1);
ub = ones(5, 1);

[w, opt] = linprog(f, A, b, Aeq, Beq, lb, ub)
\end{lstlisting}

\section*{Приложение 10} \label{p3:10}
\textbf{Решение СМО методом простых итераций}
\begin{lstlisting}[language={matlab}, caption={Решение СМО методом простых итераций}, label={lst:0}, basicstyle={\footnotesize\ttfamily}, breaklines={true}]
clear all;
close all; 
clc;
format long g;

p =   [0 0.1 0.6 0.3;
0.7 0 0.2 0.1;
0.9 0 0 0.1;
0.3 0.3 0.4 0];

M = 4;
N = 5;

lam_1_ = 3;

c = [2 2 2 2];
a = [1 1 1 1];

u = [1 1 1 1];

%% Initialize w

A = p' - diag(ones(1, M));
A(4, :) = [1; 1; 1; 1];
b = [0; 0; 0; 1];
w = inv(A) * b;
w = w';


%% Initialize u

for i = 1 : M
u(i) = w(i) / (c(i) * w(1)) * lam_1_;
end

u_last = u;

for iteration = 1 : 5
	%% Find L(N)
	[L_N, lam_1_N] = fl(u, w, c, M, N);
	L_N_1 = fl(u, w, c, M, N - 1);
	
	%% Find u1
	u(1) = lam_1_ / lam_1_N * u(1);
	
	%% Find u
	for i = 2 : M
	u(i) = u(1)  * (L_N(i) - L_N_1(i)) / (L_N(1) - L_N_1(1)) * (c(1) * a(1)) / (c(i) * a(i));
	end
	
	%% Finish iteration
	
	% min_delta = min(abs(u - u_last));
	
	u_last = u;
	
	fprintf('Iteration #%d\nSumm u(i) = %f\nu(i) = [%f, %f, %f, %f]\n\n', iteration, sum(u), u(1), u(2), u(3), u(4));
end

\end{lstlisting}

\begin{lstlisting}[language={matlab}, caption={Итеративная функция решения для ССМО}, label={lst:0}, basicstyle={\footnotesize\ttfamily}, breaklines={true}]
function [L, lumbda1] = fl(u, w, c, M, N)
	P = zeros(M, N + 1, N + 1);
	t = zeros(1, M);
	
	P(:, 1, 1) = 1;
	for r = 1 : N
		%% Step 1
		
		for i = 1 : M
			t(i) = 0;
			for n = 1 : r
				t(i) = t(i) + n / (min(n, c(i)) * u(i)) * P(i, n, r);
			end
		end
		
		%% Step 2
		
		temp = 0;
		for i = 1 : M
		temp = temp + w(i) * t(i) / w(1);
		end
		
		lumbda1 = r / temp;
		
		%% Step 3
		
		for i = 1 : M
			for n = 1 : r
				P(i, n + 1, r + 1) = w(i) / w(1) * lumbda1 / (min(n, c(i)) * u(i)) * P(i, n, r);
			end
			
			P(i, 1, r + 1) = 1;
			for n = 1 : r
				P(i, 1, r + 1) = P(i, 1, r + 1) - P(i, n + 1, r + 1);
			end
		end
	end
	
	L = t;
end
\end{lstlisting}

\section*{Приложение 11} \label{p3:11}
\textbf{Решение СМО методом Ньютона}
\begin{lstlisting}[language={matlab}, caption={Решение СМО методом Ньютона}, label={lst:0}, basicstyle={\footnotesize\ttfamily}, breaklines={true}]
clear all;
close all; 
clc;
format long g;

% delete(gcp);
% distcomp.feature('LocalUseMpiexec', false);
% parpool();

p =   [0 0.1 0.6 0.3;
0.7 0 0.2 0.1;
0.9 0 0 0.1;
0.3 0.3 0.4 0];

M = 4;
N = 5;

lam_1_ = 3;

c = [2 2 2 2];
a = [1 1 1 1];

%% Initialize w

A = p' - diag(ones(1, M));
A(4, :) = [1; 1; 1; 1];
b = [0; 0; 0; 1];
w = inv(A) * b;
w = (1 / w(1)) * w';

%% Error

e = 1e-06;

%% First approximation

u0 = zeros(1, M);
for i = 1 : M
u0(i) = w(i) / (c(i) * w(1)) * lam_1_;
end

fprintf('Start Conditions\ne = %f\nw = [%.6f %.6f %.6f %.6f]\nu0 = [%.6f %.6f %.6f %.6f]\n\n', e, w, u0);

%% Syms u

u = sym('u', [1 M]);

%% Find characteristics

G = gl3(u, w, c, M, N);
[L, Lam] = pl3(u, w, c, M, N, G);
[pL, pLam] = pl3(u, w, c, M, N - 1, G);

% [L, Lam] = fl3(u, w, c, M, N);
% [pL, pLam] = fl3(u, w, c, M, N - 1);

Function = sym(zeros(1, M));
Function(1) = simplify(lam_1_ / Lam(1) * u(1));

for i = 2 : M
% Function(i) = simplify(Function(1) * (L(i) - pL(i)) / (L(1) - pL(1)) * (c(1) * a(1)) / (c(i) * a(i)));
Function(i) = simplify(u(1) * (L(i) - pL(i)) / (L(1) - pL(1)) * (c(1) * a(1)) / (c(i) * a(i)));
end

%% Newton method with jacobian matrix

% uCurrent = u0;
uCurrent = [10 10 10 10];
uPrevious = zeros(1, M);

jaco = jacobian(Function - u);

index = 0;
condition = true;
while condition
	resf = zeros(M, 1);
	% parfor i = 1 : M
	for i = 1 : M
		resf(i) = subs(Function(i) - u(i), u, uCurrent);
	end
	
	rU = double(uCurrent);
	fprintf('%d | u = [%.6f %.6f %.6f %.6f] | F(u) = [%.6f %.6f %.6f %.6f] | S = %.6f\n', index, double(uCurrent), double(resf), c * rU');
	
	uPrevious = uCurrent;
	
	resjaco = zeros(M, M);
	% parfor i = 1 : M
	for i = 1 : M
		for j = 1 : M
			resjaco(i, j) = subs(jaco(i, j), u, uPrevious);
		end
	end
	
	resf = zeros(M, 1);
	% parfor i = 1 : M
	for i = 1 : M
		resf(i) = subs(Function(i) - u(i), u, uPrevious);
	end
	
	uCurrent = uPrevious - (resjaco \ resf)';
	
	condition = false;
	for i = 1 : M
		condition = condition || abs(uCurrent(i) - uPrevious(i)) > e;
	end
	
	index = index + 1;
	
	if (~condition)
	rU = double(uCurrent);
	fprintf('%d | u = [%.6f %.6f %.6f %.6f] | F(u) = [%.6f %.6f %.6f %.6f] | S = %.6f\n', index, double(uCurrent), double(resf), c * rU');
	end
end

resf = zeros(M, 1);
% parfor i = 1 : M
for i = 1 : M
	resf(i) = subs(Function(i) - u(i), u, uCurrent);
end

rU = double(uCurrent);
rL = double(subs(L, u, uCurrent));
rLam = double(subs(Lam, u, uCurrent));
rS = c * rU';
fprintf('\nResult\nu = [%.6f %.6f %.6f %.6f]\nF(u) = [%.6f %.6f %.6f %.6f]\nL(N) = [%.6f %.6f %.6f %.6f]\nlam(N) = [%.6f %.6f %.6f %.6f]\nS = %.6f\n', rU, double(resf), rL, rLam, rS);

\end{lstlisting}

\begin{lstlisting}[language={matlab}, caption={Расчет нормирующей константы}, label={lst:0}, basicstyle={\footnotesize\ttfamily}, breaklines={true}]
function [G] = gl3(u, w, c, M, N)
	G = sym(zeros(M, N + 1));
	G(:, 1) = 1;
	
	for k = 0 : N
		%% Find G(1, k)
		
		tempMul = 1;
		for j = 1 : k
			tempMul = tempMul * min(j, c(1)) * u(1);
		end
		
		G(1, k + 1) = (w(1) ^ k) / tempMul;
	end
	
	for r = 2 : M
		for k = 1 : N
			%% Find G(r, k)
			
			tempSum = 0;
			for h = 0 : k
				Z = 0;
				if (h == 0)
					Z = 1;
				else
					%% Find Z(r, h)
					tempMul = 1;
					for j = 1 : h
						tempMul = tempMul * min(j, c(r)) * u(r);
					end
			
					Z = (w(r) ^ h) / tempMul;
				end
			
				tempSum = tempSum + Z * G(r - 1, k - h + 1);
			end
			
			G(r, k + 1) = tempSum;
		end
	end
	
	G = simplify(G);
end
\end{lstlisting}

\begin{lstlisting}[language={matlab}, caption={Расчет основных характеристик СМО}, label={lst:0}, basicstyle={\footnotesize\ttfamily}, breaklines={true}]
function [L, Lam] = pl3(u, w, c, M, N, G)
	L = sym(zeros(1, M));
	Lam = sym(zeros(1, M));
	
	for i = 1 : M
		tempSum = 0;
		for n = 1 : N
			Z = 0;
			if (n == 0)
				Z = 1;
			else
				%% Find Z(i, n)
				tempMul = 1;
				for j = 1 : n
					tempMul = tempMul * min(j, c(i)) * u(i);
				end
				
				Z = (w(i) ^ n) / tempMul;
			end
			
			tempSum = tempSum + n * Z * G(M - 1, N - n + 1);
		end
		
		L(i) = tempSum / G(M, N + 1);
		Lam(i) = w(i) * G(M, N - 1 + 1) / G(M, N + 1);
	end
	
	L = simplify(L);
	Lam = simplify(Lam);
end
\end{lstlisting}

\section*{Приложение 12} \label{p4:12}
\textbf{Расчет статистических значений для GERT сети}
\begin{lstlisting}[language={matlab}, caption={Расчет статистических значений для GERT сети}, label={lst:0}, basicstyle={\footnotesize\ttfamily}, breaklines={true}]
clear all;
close all; 
clc;
format long g;

syms s;

% W-functions
W12 = 0.5 * exp(10*s + 8 * s^2);
W16 = 0.5 * exp(23 * s + 312.5 * s^2);
W22 = 0.2 * exp(13 * s + 128 * s^2);
W26 = 0.8 * exp(11 * s + 128 * s^2);
W35 = 1 * exp(10 * s + 40.5 * s^2);
W41 = 0.4 * exp(37 * s + 128 * s^2);
W43 = 0.3 * exp(12 * s + 128 * s^2);
W46 = 0.3 * exp(12 * s + 1200.5 * s^2);
W54 = 0.3 * exp(15 * s + 312.5  * s^2);
W55 = 0.5 * exp(19 * s + 8 * s^2);
W56 = 0.2 * exp(42 * s + 40.5 * s^2);

% We(s)
We = (W12 * W26 + W16 - W16 * W22 - W16 * W35 * W54 * W43 - W16 * W55 - W12 * W26 * W55 - W12 * W26 * W35 * W54 * W43 + W16 * W22 * W35 * W54 * W43 + W16 * W22 * W55) / (1 - W35 * W54 * W43 - W22 - W55 + W22 * W55 + W22 * W35 * W54 * W43);
We = simplify(We);

% We(0)
We0 = subs(We, 's', 0);
fprintf('We(0) = %.3f\n', double(We0));

% Me(s)
Me = We / We0;

% me1
me1 = diff(Me, 's', 1);
me1 = subs(me1, 's', 0);
fprintf('me1 = %.3f\n', double(me1));

% me2
me2 = diff(Me, 's', 2);
me2 = subs(me2, 's', 0);
fprintf('me2 = %.3f\n', double(me2));

% de
de = me2 - me1 ^ 2;
fprintf('de = %.3f\n', double(de));

\end{lstlisting}

\section*{Приложение 13} \label{p4:13}
\textbf{Расчет статистических значений при помощи анализа потокового графа}
\begin{lstlisting}[language={matlab}, caption={Расчет статистических значений при помощи анализа потокового графа}, label={lst:0}, basicstyle={\small\ttfamily}, breaklines={true}]
clear all;
close all; 
clc;
format long g;

syms s;
syms W61;

% W-functions
W12 = 0.5 * exp(10*s + 8 * s^2);
W16 = 0.5 * exp(23 * s + 312.5 * s^2);
W22 = 0.2 * exp(13 * s + 128 * s^2);
W26 = 0.8 * exp(11 * s + 128 * s^2);
W35 = 1 * exp(10 * s + 40.5 * s^2);
W41 = 0.4 * exp(37 * s + 128 * s^2);
W43 = 0.3 * exp(12 * s + 128 * s^2);
W46 = 0.3 * exp(12 * s + 1200.5 * s^2);
W54 = 0.3 * exp(15 * s + 312.5  * s^2);
W55 = 0.5 * exp(19 * s + 8 * s^2);
W56 = 0.2 * exp(42 * s + 40.5 * s^2);

Q = [0 W12 0 0 0 W16;
	 0 W22 0 0 0 W26;
  	 0 0 0 0 W35 0;
 	 W41 0 W43 0 0 W46;
 	 0 0 0 W54 W55 W56;
	 W61 0 0 0 0 0];

% Determinate A
A = eye(size(Q, 1)) - Q';
detA = det(A);
dDetA = diff(detA, W61, 1);
detA0 = subs(detA, W61, 0);

% We(s)
We = dDetA / detA0;
We = simplify(We);

% We(0)
We0 = subs(We, 's', 0);
fprintf('We(0) = %.3f\n', double(We0));

% Me(s)
Me = We / We0;

% me1
me1 = diff(Me, 's', 1);
me1 = subs(me1, 's', 0);
fprintf('me1 = %.3f\n', double(me1));

% me2
me2 = diff(Me, 's', 2);
me2 = subs(me2, 's', 0);
fprintf('me2 = %.3f\n', double(me2));

% de
de = me2 - me1 ^ 2;
fprintf('de = %.3f\n', double(de));
\end{lstlisting}

\end{document}
