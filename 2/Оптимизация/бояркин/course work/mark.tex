\chapter{Поиск оптимальной стратегии принятия решений с использованием марковских моделей}
\section{Постановка задачи}

\subsubsection{Вариант 16}

Ежедневно утром производится проверка дорогостоящей машины с целью выявления, находится ли она в исправном состоянии, требует мелкого ремонта или нуждается в серьезном ремонте. Обозначим эти состояния 0, 1, 2 соответственно. Если машина находится в совершенно исправном состоянии, то вероятность того, что она останется в таком же состоянии на начало следующего дня, равна р(0|0), вероятность того, что потребуется мелкий ремонт, равна р(1|0) и вероятность того, что возникает необходимость серьезного ремонта, равна р(2|0). В случае когда машина требует ремонта, фирма может прибегнуть к услугам двух ремонтных фирм, одна из которых (фирма F, гарантирующая качество ремонта) взимает плату М за мелкий ремонт и плату R за крупный. Вторая (фирма Т, не гарантирующая качества ремонта) взымает соответственно плату m и r, где m<М и r<R. Легко себе представить, что качество работ, производимых фирмой F, выше, чем у фирмы Т, что отражается значением вероятности полностью исправного состояния машины на начало следующего за ремонтом дня. Пусть решение d=1 определяет выбор фирмы F и решение d=2 — выбор фирмы Т. Обозначим через р(j | i, d) вероятность перехода машины в состояние j на следующем отрезке (j=0,1,2) при условии, что она находится в состоянии I на текущем отрезке (i=1,2) и принимается решение d(d=1,2).
\\\\
р (0 | 0) = 0.6 (машина осталась исправной)\\
р (1 | 0) = 0.3 (машина требует мелкого ремонта)\\
р (2 | 0) = 0.1 (машина требует крупного ремонта)\\\\
р (0 | 1, 1) = 0.9 [М = 14] (фирма F выполнила мелкий ремонт)\\
р (1 | 1, 1) = 0.1 (фирма F в процессе выполнения мелкого ремонта)\\
р (2 | 1, 1) = 0 (невозможное событие -- если выполнен крупный ремонт, то мелкий не нужен)\\
р (0 | 2, 1) = 0.6 [R = 21] (фирма F выполнила крупный ремонт)\\
р (1 | 2, 1) = 0.3 [R – M = 7] (фирма F выполнила мелкий ремонт, но надо доделать до крупного)\\
р (2 | 2, 1) = 0.1 (фирма F в процессе выполнения крупного ремонта)\\\\
р (0 | 1, 2) = 0.7 [m = 12] (фирма T выполнила мелкий ремонт)\\
р (1 | 1, 2) = 0.2 (фирма T в процессе выполнения мелкого ремонта)\\
р (2 | 1, 2) = 0 (невозможное событие -- если выполнен крупный ремонт, то мелкий не нужен)\\
р (0 | 2, 2) = 0.5 [r = 19] (фирма T выполнила крупный ремонт)\\
р (1 | 2, 2) = 0.4 [r – m = 7] (фирма T выполнила мелкий ремонт, но надо доделать до крупного)\\
р (2 | 2, 2) = 0.1 (фирма T в процессе выполнения крупного ремонта)\\
\\
Найдите оптимальную стратегию и минимальные затраты на отрезке N=$\infty$.

\subsubsection{Доп задание}

Предположите, что фирме F для выполнения крупного ремонта требуется 1 полный рабочий день, а фирме Т — 2 полных рабочих
дня. Считайте далее, что фирма — владелец машины несет потери в размере $c$ единиц за каждый день ее простоя. Покажите, как при этих условиях нужно изменить уравнения.

\section{Решение}

\subsection{Решения и стратегии}

Имеется три состояния машины:

\begin{itemize}
	\item 1 -- исправна;
	\item 2 -- легкая поломка (необходим мелкий ремонт);
	\item 3 -- серьезная поломка (необходим крупный ремонт);
\end{itemize}

Для данных состояний имеется три решения:

\begin{itemize}
	\item $X_1$ -- ничего не делать;
	\item $X_2$ -- выбрать фирму F;
	\item $X_3$ -- выбрать фирму T;
\end{itemize}

Таким образом, возможны следующие стратегии:

\begin{table}[h!]
	\centering
	\bgroup
	\captionsetup{singlelinecheck = false, format= hang, justification=raggedleft, font=footnotesize, labelsep=space}
	\caption{Возможные стратегии}
	\def\arraystretch{1}
	\begin{tabular}{ | m{0.3cm} | m{2.2cm} | m{2.8cm} | m{3.1cm} | }
		\hline
		№ & Исправна & Легкая поломка & Серьезная поломка \\ \hline
		1 & $X_1, (-)$ & $X_2, (F)$ & $X_3, (T)$ \\ \hline
		2 & $X_1, (-)$ & $X_2, (F)$ & $X_2, (F)$ \\ \hline
		3 & $X_1, (-)$ & $X_3, (T)$ & $X_3, (T)$ \\ \hline
		4 & $X_1, (-)$ & $X_3, (T)$ & $X_1, (F)$ \\
		\hline
	\end{tabular}
	\egroup
\end{table}

\subsection{Построение матриц переходных вероятностей и матриц расходов}

$P_1$ и $P_2$ -- матрицы переходных вероятностей для компаний $F$ и $T$ соответственно:
\\

$
P_1=\begin{bmatrix}
0.6 & 0.3 & 0.1 \\
- & - & - \\
- & - & - \\
\end{bmatrix}
$
$
P_2=\begin{bmatrix}
- & - & - \\
0.9 & 0.1 & 0 \\
0.6 & 0.3 & 0.1 \\
\end{bmatrix}
$
$
P_3=\begin{bmatrix}
- & - & - \\
0.7 & 0.2 & 0.1 \\
0.5 & 0.4 & 0.1 \\
\end{bmatrix}
$\\

$R_1$ и $R_2$ -- матрицы расходов для компаний $F$ и $T$ соответственно:
\\

$
R_1=\begin{bmatrix}
0 & 0 & 0 \\
- & - & - \\
- & - & - \\
\end{bmatrix}
$
$
R_2=\begin{bmatrix}
- & - & - \\
14 & 0 & 0 \\
21 & 7 & 0 \\
\end{bmatrix}
$
$
R_3=\begin{bmatrix}
- & - & - \\
12 & 0 & 0 \\
19 & 7 & 0 \\
\end{bmatrix}
$\\

\subsection{Нахождение величин ожидаемого дохода}

Ожидаемый доход вычисляется по формуле:\\

$\nu_i(X_{k})=\sum_{j=1}^{m}p_{i,j}(X_{k})r_{i,j}(X_{k})$\\

Тогда для первого решения (ничего не делать):\\

$\nu_1(X_1)=0$

$\nu_2(X_1)=-$

$\nu_3(X_1)=-$\\

Для второго решения (выбрать фирму F):\\

$\nu_1(X_2)=-$

$\nu_2(X_2)=0.9\cdot 14=12.6$

$\nu_3(X_2)=0.6\cdot 21+0.3\cdot 7=14.7$\\

Тогда для третьего решения (выбрать фирму T):\\

$\nu_1(X_3)=-$

$\nu_2(X_3)=0.7\cdot 12=8.4$

$\nu_3(X_3)=0.5\cdot 19+0.4\cdot 7=12.3$\\

\subsection{Формулировка задачи в виде задачи линейного программирования}

Приведение к задаче линейного программирования производится следующим образом:

\begin{figure}[h!]
	\centering
	\includegraphics[scale = 0.90]{images/p2_1.png}
	\label{image:p2_1}
\end{figure}

Для данной задачи:

\begin{equation*}
\begin{cases}
\text{$0\cdot w_{11}+12.6\cdot w_{22}+8.4\cdot w_{23}+14.7\cdot w_{32}+12.3\cdot w_{33}\rightarrow min$} \\
\text{$(1-0.6)\cdot w_{11}-0.9\cdot w_{22}-0.7\cdot w_{23}-0.6\cdot w_{32}-0.5\cdot w_{33}=0$} \\
\text{$-0.3\cdot w_{11}+(1-0.1)\cdot w_{22}+(1-0.2)\cdot w_{23}-0.3\cdot w_{32}-0.4\cdot w_{33}=0$} \\
\text{$-0.1\cdot w_{11}-0\cdot w_{22}-0.1\cdot w_{23}+(1-0.1)\cdot w_{32}+(1-0.1)\cdot w_{33}=0$} \\
\text{$w_{11}+w_{22}+w_{23}+w_{32}+w_{33}=1$}
\text{$w_{ij}\geq 0, i=\overline{1,2,3}, j=\overline{1,2,3}$}
\end{cases}
\end{equation*}

\subsection{Решение задачи}

Разработаем скрипт для расчета вероятностей $w_{ij}$ в среде MATLAB (Приложение 9).

Результат расчета вероятностей:

\lstinputlisting{listings/p2.l1.log}

Таким образом машина исправна с вероятностью 61.82\%. Для мелкого ремонта следует обращаться в фирму T (28.18\%). Для крупного ремонта следует обращаться в фирму F (10\%).

\subsection{Дополнительное задание}

Предположите, что фирме F для выполнения крупного ремонта требуется 1 полный рабочий день, а фирме Т — 2 полных рабочих
дня. Считайте далее, что фирма — владелец машины несет потери в размере $c$ единиц за каждый день ее простоя. Покажите, как при этих условиях нужно изменить уравнения.

Для решения задачи нет смысла определять новые состояния, как это указано в методических указаниях. Достаточно просто изменить матрицы расходов для компаний $F$ и $T$ следующим образом:\\

$
R_1=\begin{bmatrix}
0 & 0 & 0 \\
- & - & - \\
- & - & - \\
\end{bmatrix}
$
$
R_2=\begin{bmatrix}
- & - & - \\
14 & 0 & 0 \\
21 + c & 7 + c & 0 \\
\end{bmatrix}
$
$
R_3=\begin{bmatrix}
- & - & - \\
12 & 0 & 0 \\
19 + 2\cdot c & 7 + 2\cdot c & 0 \\
\end{bmatrix}
$\\

Дальше задача решается аналогичным образом.

\section{Вывод}

Были определены оптимальные стратегии для конкретной задачи. Машина исправна с вероятностью 61.82\%. Для мелкого ремонта следует обращаться в фирму T (28.18\%). Для крупного ремонта следует обращаться в фирму F (10\%).

Линейное программирование позволяет достаточно легко и быстро решать подобные задачи, однако требует некоторых предварительных преобразований перед использованием.